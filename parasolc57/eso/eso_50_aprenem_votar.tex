\begin{news}
{3} %columnes
{Aprenem a votar}
{La nostra escola ha passat a formar part del col·lectiu de 40 escoles i instituts que a tot Catalunya portaran a terme l’experiència d’innovació pedagògica “Aprenem a votar”. 
}
{ESO}
{33} %pagesof

\noindent\includegraphics[width=5.5cm,keepaspectratio]{eso/img/aprenem_votar.jpg}

El programa es desenvoluparà durant el primer trimestre del curs 2010-2011 i té com a finalitat que l’alumnat de 4t d’ESO aprengui a exercir lliurement el seu dret al vot de cara al futur, a partir d’un procés d’informació de la mecànica electoral i d’una simulació de les eleccions.

El projecte està promogut pel Departament de Didàctica de les Ciències Socials i té el suport del Departament d’Educació de la Generalitat de Catalunya.


\bf Estructura del projecte  

\rm
El projecte es divideix en cinc fases:

Unitat 1. Elegim. Es tractarà de conèixer els diferents elements del sistema democràtic català, i del Parlament en particular, i descriure’n el funcionament general.

Unitat 2. Ens informem. Es tractarà de reconèixer i utilitzar de manera crítica les diverses fonts d’informació que es posaran a disposició dels nois i noies durant la campanya electoral.

Unitat 3. Opinem.  L’alumnat podrà opinar sobre els diferents partits i coalicions catalans, tenint en compte les preocupacions i les opcions que presentaran durant les eleccions.

Unitat 4. Participem. Incrementarà la informació sobre les diferents maneres de participar en un sistema democràtic.


Unitat 5. Votem. Es tractarà d’aprendre a organitzar una votació tot respectant els criteris que fan possible la celebració d’unes eleccions lliures i democràtiques. L’alumnat participarà en una simulació el divendres anterior al dia de les eleccions al Parlament de Catalunya. 


En una segona part d’aquest programa, es reflexionarà sobre els resultats oficials de les eleccions al Parlament en comparació amb el vot estudiantil de quart d’ESO

Departament de Ciències Socials – Educació Secundària Obligatòria

\end{news}
