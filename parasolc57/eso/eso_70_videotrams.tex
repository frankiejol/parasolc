\begin{news}
{2} %columnes
{VideoTrams}
{El grup de guionistes, format per quatre persones i reforçat per part de l'equip de direcció, van desenvolupar una història basada en el tema principal triat per la classe:“Embarassos a l'adolescència” }
{ESO}
{39} %pagesof


Tik, Tok, s'ha acabat el temps, el nostre vídeo ja està enviat a l'altra escola (és un treball que fem conjuntament amb les escoles de TRAMS). Ha estat un treball en grup complicat que alhora hem sabut tirar endavant. Aquest vídeo ha estat un treball transversal, ja que en fer-lo  hem treballat temes de les àrees de tutoria i ciutadania, d'educació audiovisual, d'informàtica i música. 

En els primers moments,  vam dividir-nos en grups segons la funció que volíem realitzar en el videotrams. El grup de guionistes, format per quatre persones i reforçat per part de l'equip de direcció, van desenvolupar una història basada en el tema principal triat per la classe: “Embarassos a l'adolescència”. Un cop exercida la feina dels guionistes, van començar els càstings per triar els actors i les actrius que portarien a terme els papers principals de la història.



Els guionistes, juntament amb l'equip directiu,  van donar el seu punt de vista sobre la història perquè els actors i actrius l'enfoquessin com ells creien. Després d'assajos, va arribar l'hora que el grup de decorats pensés com poder adaptar les diferents aules de l'escola d'acord amb l'escenari. Un cop tot adaptat i amb el guió après va ser l'hora de posar-se a gravar. 

\noindent\includegraphics[width=9cm,keepaspectratio]{eso/img/videotrams_3.jpg}

Els càmeres se les van empescar per introduir diferents plans  al llarg de tota la història.  Quan va estar gravat, l'equip d'edició de vídeo amb la col·laboració de la Lluïsa  van editar-lo i enviar-lo a l'escola corresponent. El resultat? Un vídeo que mostra un treball excepcional de grup en una activitat més lúdica  del que és habitual.

\authorandplace{Esteve Pérez i Ariadna Peiró}{3r d'ESO}

\end{news}

\newssep
