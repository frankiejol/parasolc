\begin{news}
{2} %columnes
{Les colònies de 1r i 2n d'ESO a Camprodon}
{El dia 4 de maig els nens i les nenes de 1r i 2n d'ESO vam marxar de colònies a Camprodon}
{ESO}
{033} %pagesof

%\noindent\includegraphics[width=9cm,keepaspectratio]{eso/img/JACOB_DAHLGREN.jpg}


El primer dia ens vam dividir: els nens i les nenes de 2n van anar a visitar el monestir de Ripoll, i els de primer vam anar a visitar el Molí i el Monestir de St. Joan de les Abadesses.

Allà ens van explicar que el molí, en el segle VII aproximadament, era molt important, ja que bàsicament menjaven pa. Aquest molí era hidràulic,  funcionava amb la força de l'aigua, per tant estava contruït a prop d'un riu. Com que no sempre el riu baixava amb la mateixa força, desviaven l'aigua per una canal fins a una bassa. Aquesta aigua, quan la necessitaven, la feien baixar per una mena de pendent que feia voltar una pedra enorme que triturava el blat i el convertia en farina, aquesta farina s'havia de dur a les abadesses del monestir.

Després vam anar a visitar el Monestir, que el va fer construir Guifré el Pilós al segle IX. Allà hi vivien les monges i l' abadessa que era la filla de Guifré el Pilós, i es deia Emma.  

Pel fet de ser allà el monestir, es van repoblar les  terres dels voltants.
 El nom que té el monestir té l'origen  que uns segles més tard el Comte de Tallaferro va intentar prendre a l' abadessa Inguilverra el monestir, i com que no ho va aconseguir, va anar a parlar amb el Papa i les va acusar de molts “pecats”, encara que tot i així no es  va sortir amb la seva.

Llavors va fer un tracte amb el Papa : ell es quedava amb el monestir,  però el Papa rebia una part de les riqueses. Aleshores oficialment van fer fora   les monges i  les abadesses amb l'argument que  eren unes Meretrius de Venus. Per aquest motiu el nom que té el monestir és com una mena d'insult. 

Durant la Guerra civil el monestir el van fer servir de polvorí. 

Al cap de molts segles hi va haver terratrèmol que va destruir una gran part del Monestir. 
El van tornar a reconstruir a l'època romànica, incorporant motius decoratius propis de l'art romànic i també del gòtic.

Aquesta visita m`ha semblat molt interessant ja que fins ara, quan veia un monestir, no em podia imaginar que tingués tanta història al seu darrera. Ara imagino les persones, les guerres, els interessos..., i no tan sols un simple convent on vivien monges. Darrera de tota edificació històrica, tal com diu el seu nom: HI HA UNA HISTÒRIA.


\authorandplace{Marta Ortega}{1r ESO}

\end{news}
