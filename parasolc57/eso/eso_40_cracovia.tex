\begin{news}
{2} %columnes
{Cracòvia 2010}
{}
{ESO}
{035} %pagesof

\noindent\includegraphics[width=9cm,keepaspectratio]{eso/img/polonia_P5030075.JPG}

Tots sabíem on anàvem. Si més no, ho havíem decidit nosaltres. Polska, Kraków, Polònia, Cracòvia. Aquesta era la nostra destinació, un món ple d’homes i de dones rossos amb els ulls blaus, un món plujós, poc acostumat a veure un dia totalment assolellat i de viure les caloroses temperatures mediterrànies, un idioma impossible d’entendre, una moneda estranya. Sabíem que hauríem de fer quatre hores d’avió més l’espera a l’aeroport, i també que valdria la pena perquè ens esperava un viatge inoblidable. 

Vam arribar a la nostra destinació, cansats però amb ganes d’instal·lar-nos a l’hotel i començar a viure aquells cinc dies del viatge de fi de curs amb intensitat.

Durant aquests cinc dies, vam passejar per els carrers humits de Cracòvia, vam observar encuriosits els racons més especials i bonics d’aquella petita ciutat, vam provar el menjar típic, com per exemple el formatge fumat, per alguns molt bo, per altres no tant. També visitàrem Zakopane un petit poble hivernal, amb unes vistes precioses. Vam viure l’experiència esgarrifant de endinsar-nos en el Camp de Concentració d’Auschwitz, visitar unes mines de sal sota terra, on vam quedar totalment bocabadats al veure que hi havia escultures totalment de sal a 135 m de profunditat. 


Tampoc podrem oblidar les nits perdudes pel barri jueu on es trobava el nostre hotel, anant a un parc d’atraccions o simplement passejant, i tornant cap a l’hotel, pels carrers que dia a dia ja s’anaven fent nostres, mentre que la pluja ens mullava (factor que ens va acompanyar durant tot el viatge, com era previst), però ens era igual, perquè l’important era que estàvem junts, i volíem aprofitar tots els minuts d’aquells dies per fer d’aquell viatge que tancaria el nostre cicle a l’Escola Solc, un record inesborrable.

\authorandplace{Rosa Gras}{4t ESO}

\end{news}

\newssep
