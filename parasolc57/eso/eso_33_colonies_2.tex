\begin{news}
{2} %columnes
{Camprodon' 10}
{}
{ESO}
{033} %pagesof


Quan vaig arribar em van venir al cap molts records, bons i dolents, i és que l'any anterior ja havíem anat a aquella mateixa casa. Els monitors eren els mateixos menys una noia, que era nova. L'Ignasi ens va rebre molt bé. Per començar vam anar a deixar les motxilles a la sala polivalent, després vam anar als porxos a escoltar les normes de la casa. Encara que ja les sabíem de l'any anterior, les vam escoltar molt atents, i després vam tornar  a la sala polivalent a buscar les motxilles per distribuir-nos  les habitacions.

 Jo vaig estar molt bé amb la gent que em va tocar a l'habitació, ja que són persones amb les quals m'hi trobo bé. A nosaltres ens va tocar una habitació que estava al passadís de tots els nens de la nostra classe. Ens van deixar un temps per fer-nos el llit, i després l'Ignasi va xiular amb el seu xiulet. Això volia dir que havíem d'anar tots a la sala polivalent, i així ho vam fer. 

\columntitle{lines}
{Quan vaig arribar em van venir al cap molts records, bons i dolents, i és que l'any anterior ja havíem anat a aquella mateixa casa}

Un cop els 63 vam ser allà dins, l'Ignasi ens va proposar un joc molt divertit: un es tapava els ulls amb un mocador, llavors una persona (tant de primer com de segon d' ESO) sortia i el dels ulls tapats havia d'endevinar qui era aquella persona només tocant-li la cara i els peus. Després vam fer un altre joc: dues persones (normalment un era el més alt de la classe i l'altre el més baixet) havien de portar una patata amb el front fins a l'altra banda de la sala. Quan vam acabar aquests jocs, va arribar l'hora d'anar a sopar. 


El menjar era molt bo. Cada dia parava i desparava taula un grup de gent diferent, perquè així el treball quedava més ben repartit. 

El dimecres al matí, l'Ignasi i els tres monitors ens van explicar el que faríem durant aquell dia. Faríem diferents activitats;  tir amb arc, fer un recorregut amb bicicleta, rocòdrom i jocs d'enginy. A mi la que em va agradar menys va ser el rocòdrom. Aquestes quatre activitats les vam fer durant tot el dia.

{\noindent\includegraphics[width=9cm,keepaspectratio]{eso/img/Camprodon.JPG}}

A la nit no vam poder fer el joc que tenien preparat ja que nevava. Però per substituir-lo ens van posar una pel·lícula. 

Al dia següent va tocar l'excursió per la muntanya. Vam fer 17quilòmetres en total 8 hores caminant. Gràcies a aquesta excursió em vaig adonar que tots tenim més resistència del que ens pensem. Quan vam arribar a la casa, la Davínia i l'Àlex van jugar a pedra, paper o tisora per decidir qui anava primer a dutxar-se, si les noies o els nois, i vam guanyar les noies. 

Les noies vam anar a berenar xocolata calenta amb melindros, i els nois se'n van anar a dutxar. 

Quan tots vam estar dutxats, vam anar a sopar per agafar forces; a la nit hi havia discoteca! Quan va començar la disco em vaig quedar molt sorpresa perquè, després de la caminada que havíem fet,  la gent ballava com boja. Allà va ser on vaig saber com és la gent de veritat. Hi ha gent que em va sorprendre molt, perquè mai no m'hauria imaginat que arribarien a ballar tant. 

La veritat és que jo també em vaig sorprendre  perquè mai havia ballat tant com aquella nit. 

I al dia següent vam tornar cap a casa.

\authorandplace{Andrea Ibáñez}{2n ESO}

\end{news}

\newssep
