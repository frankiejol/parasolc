\begin{news}
{3} %columnes
{Sortida a la fundació Miró.  Exposició MURALS}
%index: Fundació Miró: Exposició Murals
{ \noindent\includegraphics[width=9cm,keepaspectratio]{eso/img/JACOB_DAHLGREN.jpg}
\noindent\includegraphics[width=9cm,keepaspectratio]{eso/img/UTR_CREW.jpg}}
{ESO}
{31} %pagesof

Els nois i noies de 2n d’ESO,el dia 9 d’abril, vam anar a la Fundació Joan Miró per visitar l’exposició ‘MURALS’.

En arribar vam esmorzar i després ens van separar en dos grups. Cada grup tenia un guia, el del segon grup es deia David.

Abans de començar,  en David ens va explicar que els primers murals que es van pintar van ser els dibuixos de les coves, a la prehistòria; i que aquests ens han permès estudiar la vida d’aquella època.
També ens va explicar que Joan Miró va pintar murals, i que durant el gòtic el nombre de murals va disminuir a causa de la construcció d’edificis amb grans vitralls.

El primer mural que vàrem veure va ser pintat per unes dones de Diadji Bine Gan de Ga. Allà les dones són les que ho controlen tot i pinten per decorar les seves cases.
Els colors que utilitzen són els colors típics de Marraqueix (grocs, ocres, marrons...), també pinten amb colors vius i ho fan de manera senzilla i detallada.
Pinten els murals amb les mans i els dits,i fabriquen els colors barrejant aigua amb determinats pigments.

El segon mural treballava la geometria i la precisió. A diferència de l’anterior, aquest no era improvisat, cada figura havia estat mesurada i feta utilitzant el regle i el nivell.
El pintor pinta amb una precisió perfecte un tema abstracte.
Cada figura mesura la seva alçada. Amb aquesta obra Lothar Götz, vol representar un record de la seva infantesa.

El tercer mural ens va donar la impressió que allò no era un tipus d’art. Era un mural fet amb plantes (heures). L’artista mexicà només treballa amb plantes com a protesta ecologista.
L’obra s’anomena ‘Contemplant la invasió’, on l’espectador assegut en un banc i a través d’un vidre, contempla la invasió d’heura, que ocupa tot el vidre i que anirà enfilant-se per l’edifici.
L’obra s’havia començat a preparar amb un any d’antelació.

El quart mural interpretava un petit temple budista. L’artista va demanar a la Fundació que pintessin la sala de color vermell i que tapessin el terra amb moqueta vermella. Ell va crear l’obra utilitzant unes plantilles amb forats i pols d’arròs. (Utilitzant una tècnica semblant al puntillisme).

La cinquena obra que ens van ensenyar era un graffiti tradicional. En David ens va explicar que els primera graffitis eren signatures (TAG) i que el pintor és de Singapur, on valoren molt el civisme, per tant si enxampen algú fent un graffiti el penalitzen amb una multa.

El sisè mural era un homenatge al graffiti, cada dibuix representa diferents sentiments, emocions, opinions...
El dibuix d'un puny representa l'esforç de fer un graffiti. Unes càmeres de vigilància dibuixades a un racó representen la vigilància de la ciutat, i el cor al centre és l'emoció i la por que sent l'artista quan pinta un graffiti per la ciutat.

El següent mural era un dibuix fet amb pintura blanca sobre un fons pintat prèviament en negre (com si fos una pissarra). A un costat hi havia representades les pors del pintor i a l'altra banda dibuixos de platets voladors i abduccions alienígenes, una por molt americana.

A continuació Núria i Tono. Són els artistes del vuitè mural.
Representen el significat del graffiti.
Els seus graffitis s'integren a les superfícies on els pinten. Aquests dos artistes han creat també graffitis lumínics. Pinten de forma legal, amb permís. I normalment la gent els conserva, els graffitis. El seu treball dins l'exposició representa figures geomètriques pintades amb pinzell amb diferents gradacions de color.

L'avantpenúltim mural és un paisatge tan en el fons com en el primer pal. Una dent de lleó. L'artista l'ha dibuixat en blanc i negre amb la intenció  que cadascú mentalment li posi els colors que vulgui.

El desè mural és una cobra de collage realitzada per Ludovica Gioscia. Per fer aquesta obra , l'artista va haver de calcular-ho tot. La mida dels papers, roba, colors, ...
És un treball molt elaborat i fet de capes.

La darrera obra estava formada per dianes penjades que omplien tot el pany de paret.
L'artista ha volgut jugar amb la norma de no poder tocar els murals. El fet de posar-hi dianes ens convida a atacar l'obra d'art amb dards.

La meva opinió sobre aquesta exposició és que és molt original, ja que sovint no trobem que el graffiti sigui entès com un tipus d'art dins d'un museu.

							Davínia Vilagrassa 2n ESO


\end{news}
