\begin{news}
{3} %columnes
{Viatge 4T D’ESO 2010}
%index: 4t d'ESO a Polònia
{
\noindent\includegraphics[width=18cm,keepaspectratio]{eso/img/polonia_P5050116.JPG}
}
{ESO}
{036} %pagesof


Diumenge 2 de maig al matí, per no dir la matinada, ens trobàvem arrossegant les maletes i els nervis per l’aeroport del Prat

%\noindent\includegraphics[width=9cm,keepaspectratio]{eso/img/polonia_P5050106.JPG}
A poc a poc deixàvem enrere la bella Barcelona i les no tan belles matemàtiques, literatures i d’altres, i fèiem camí cap a Polònia; però abans vam parar a Düsseldorf, una ciutat alemanya amb un aeroport preciós on ens vam acomiadar dels euros per quatre dies. Havent dinat vam embarcar per segona vegada, ara ja sí cap a Cracòvia, una ciutat grisa i castigada encara que preciosa.

Era dilluns però no ens feia mandra llevar-nos a quarts de vuit; no teníem per endavant un dia laboral ni d’escola, sinó una visita a la ciutat on ens allotjàvem: el centre, Kazimierz (el barri jueu), l’antic Ghetto, el pujolet de Babel (que no tenia res a veure amb la mítica Torre de Babel) on per sobre la ciutat regnava la Catedral, la Universitat on es formà Copèrnic, etc. Vam dinar pel centre, uns asseguraven un bon àpat a alguna pizzeria (el que fa la globalització) i d’altres s’atrevien amb la gastronomia de la zona. 

A la tarda vàrem visitar les mines de sal de Wieliczka, a 15km de Cracòvia, on hi vam arribar amb l’autocar. Vam fer un recorregut que ens portava per diferents cambres, fins i tot hi havia una capella, la de Sta.Kinga, feta de la mateixa sal de la mina que els propis miners havien construït.

Eren quarts de vuit quan arribàvem a l’hotel, havíem de ser puntuals, el sopar se servia a les vuit. 

L’endemà al matí l’autocar ens duia fins a Zakopane, la capital hivernal de Polònia, un lloc molt turístic però tot i això força rural. Pràcticament vam passar el dia  allà, vam aprofitar per comprar alguns souvenirs i la majoria vam tastar els famosos formatges fumats de la zona, molts ens vam sorprendre, no eren deliciosos com ens els havien  pintat però eren força bons. 

Dimecres 5 ens vam llevar més d’hora, vam canviar la molla Cracòvia per la freda Auschwitz. Una visita guiada ens portava pels blocs i carrers de la ciutat d’extermini nazi, passant pels crematoris que encara continuaven drets i les cambres de gas. 

Vam estar a Auschwitz I i a Auschwitz II-Birkenau, de 90 i 147 hectàrees respectivament. 

Vam tornar a Cracòvia a l’hora de dinar i a la tarda vam estar passejant i fent les últimes compres, els últims berenars o cafès amb zlotis. 

De sobte, es va fer dijous i havíem de passar altre cop per l’aeroport de Düsseldorf i acabar de gastar les últimes monedes poloneses; així que vam tornar als entrepans de l’avió i a les revistes alemanyes indesxifrables. A quarts de vuit del vespre arribàvem a Barcelona deixant enrere Polònia, els seus carrers i places, la fredor general de la gent, la delicada i valenta Cracòvia, i el nostre primer i últim viatge com a classe de 4t d’ESO. 


\authorandplace{Júlia Oliver}{4t d’ESO}

\end{news}
