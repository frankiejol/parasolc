\begin{news}
{2} %columnes
{Redacció sobre les colònies a Fenals}
%index: Colònies d'ESO: Fenals i Camprodón
{Les colònies de tercer d’ESO són un mite entre tots els alumnes de la secundària; i és que només començar-la ja penses en elles, tothom parla que són uns dies únics i genials; doncs bé, a la fi aquest any hem arribat  tercer!}
{ESO}
{18} %pagesof

\noindent\includegraphics[width=9cm,keepaspectratio]{eso/img/colonies_1.jpg}

Després de tot l’any esperant,  arribà el dia quatre de maig i amb pluja inclosa vam marxar a Fenals. Des del primer dia que el temps no ens va acompanyar, però això no va resultar cap problema i és que vam realitzar sense cap entrebanc la majoria d’activitats programades.  Colònies de platja passades per aigua, encara que ha de quedar clar que van ser igual o més divertides. Van passar volant i vam gaudir al màxim cada dia que anava passant. Quatre dies que van marcar la unió entre nosaltres; i en definitiva, les colònies de tercer són les millors! Què més dóna que faci fred mentre fem submarinisme, que més dóna que plogui si tenim el pòquer i el pictionary! Si la cosa més important i fonamental de les colònies és la unió que es crea entre nosaltres!

Les nostres últimes colònies a la Solc han estat perfectes; un bon hotel, bona gent i activitats divertides... Ha estat una estada diferent de totes les que hem fet fins ara i és que l’objectiu final d’aquestes era crear un gran vincle d’amistat entre nosaltres i alhora descobrir diferents activitats marítimes, la majoria, per passar quatre dies desconnectats del món exterior.

\authorandplace{Anna Peiró}{3r ESO}

\end{news}

\newssep
