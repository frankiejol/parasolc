\begin{news}
{2} %columnes
{Sant Jordi}
{\noindent\includegraphics[width=18cm,keepaspectratio]{eso/img/st_jordi_bates.jpg}}
{ESO}
{500} %pagesof

{\noindent\includegraphics[width=9cm,keepaspectratio]{eso/img/st_jordi_taulell.jpg}}

El dia de St.Jordi,  l'alumnat de 3r d'ESO ens vam encarregar de preparar-ho tot.
Alguns alumnes van venir una hora abans per muntar les parades abans que arribessin els pares per comprar alguns llibres. Al cap d'una hora van arribar els altres alumnes, i un grup de nosaltres se'n va anar a muntar les activitats per als més petits. 
A la parada, van anar venint tots els alumnes, primer els més petits i després els més grans, per comprar-nos molts llibres i polseres.

Els nens de parvulari s'ho van passar molt bé picant les pinyates que els vam preparar. Els vam tapar els ulls i els vam donar un pal perquè les trenquessin. 

{\noindent\includegraphics[width=9cm,keepaspectratio]{eso/img/st_jordi_pinguins.jpg}}

%{\noindent\includegraphics[width=9cm,keepaspectratio]{eso/img/st_jordi_carablanca.jpg}}

%{\noindent\includegraphics[width=9cm,keepaspectratio]{eso/img/st_jordi_taulell.jpg}}


Els nois i noies de 2n de primària també es van divertir molt amb les pinyates, però també fent la prova dels “filipinos”, que consisteix en passar un fil per un “filipino” i anar menjant el fil fins arribar al “filipino” i menjar-se'l.
Els alumnes de 3r i 4t de primària van fer molts jocs variats com el dels “filipinos”, llençar anelles a un pal, buscar un caramel amb la boca dins d'un bol amb farina, portar ous en una cullera,...

Volem agrair a tot l'alumnat, al professorat i a tots els pares i mares que van contribuir , comprant llibres,  al nostre futur viatge de 4t d’ESO   

\authorandplace{Artur Martínez}{3r ESO}
\end{news}
