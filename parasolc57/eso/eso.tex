\definecolor{color}{rgb}{0.1 , 0.1 , 0.1}

\begin{news}
{2} %columnes
{Redacció sobre les colònies a Fenals}
%index: Colònies d'ESO: Fenals i Camprodón
{Les colònies de tercer d’ESO són un mite entre tots els alumnes de la secundària; i és que només començar-la ja penses en elles, tothom parla que són uns dies únics i genials; doncs bé, a la fi aquest any hem arribat  tercer!}
{ESO}
{18} %pagesof

\noindent\includegraphics[width=9cm,keepaspectratio]{eso/img/colonies_1.jpg}

Després de tot l’any esperant,  arribà el dia quatre de maig i amb pluja inclosa vam marxar a Fenals. Des del primer dia que el temps no ens va acompanyar, però això no va resultar cap problema i és que vam realitzar sense cap entrebanc la majoria d’activitats programades.  Colònies de platja passades per aigua, encara que ha de quedar clar que van ser igual o més divertides. Van passar volant i vam gaudir al màxim cada dia que anava passant. Quatre dies que van marcar la unió entre nosaltres; i en definitiva, les colònies de tercer són les millors! Què més dóna que faci fred mentre fem submarinisme, que més dóna que plogui si tenim el pòquer i el pictionary! Si la cosa més important i fonamental de les colònies és la unió que es crea entre nosaltres!

Les nostres últimes colònies a la Solc han estat perfectes; un bon hotel, bona gent i activitats divertides... Ha estat una estada diferent de totes les que hem fet fins ara i és que l’objectiu final d’aquestes era crear un gran vincle d’amistat entre nosaltres i alhora descobrir diferents activitats marítimes, la majoria, per passar quatre dies desconnectats del món exterior.

\authorandplace{Anna Peiró}{3r ESO}

\end{news}

\newssep
\definecolor{color}{rgb}{0.1 , 0.1 , 0.1}

\begin{news}
{2} %columnes
{Les colònies de 1r i 2n d'ESO a Camprodon}
{El dia 4 de maig els nens i les nenes de 1r i 2n d'ESO vam marxar de colònies a Camprodon}
{ESO}
{033} %pagesof

%\noindent\includegraphics[width=9cm,keepaspectratio]{eso/img/JACOB_DAHLGREN.jpg}


El primer dia ens vam dividir: els nens i les nenes de 2n van anar a visitar el monestir de Ripoll, i els de primer vam anar a visitar el Molí i el Monestir de St. Joan de les Abadesses.

Allà ens van explicar que el molí, en el segle VII aproximadament, era molt important, ja que bàsicament menjaven pa. Aquest molí era hidràulic,  funcionava amb la força de l'aigua, per tant estava contruït a prop d'un riu. Com que no sempre el riu baixava amb la mateixa força, desviaven l'aigua per una canal fins a una bassa. Aquesta aigua, quan la necessitaven, la feien baixar per una mena de pendent que feia voltar una pedra enorme que triturava el blat i el convertia en farina, aquesta farina s'havia de dur a les abadesses del monestir.

Després vam anar a visitar el Monestir, que el va fer construir Guifré el Pilós al segle IX. Allà hi vivien les monges i l' abadessa que era la filla de Guifré el Pilós, i es deia Emma.  

Pel fet de ser allà el monestir, es van repoblar les  terres dels voltants.
 El nom que té el monestir té l'origen  que uns segles més tard el Comte de Tallaferro va intentar prendre a l' abadessa Inguilverra el monestir, i com que no ho va aconseguir, va anar a parlar amb el Papa i les va acusar de molts “pecats”, encara que tot i així no es  va sortir amb la seva.

Llavors va fer un tracte amb el Papa : ell es quedava amb el monestir,  però el Papa rebia una part de les riqueses. Aleshores oficialment van fer fora   les monges i  les abadesses amb l'argument que  eren unes Meretrius de Venus. Per aquest motiu el nom que té el monestir és com una mena d'insult. 

Durant la Guerra civil el monestir el van fer servir de polvorí. 

Al cap de molts segles hi va haver terratrèmol que va destruir una gran part del Monestir. 
El van tornar a reconstruir a l'època romànica, incorporant motius decoratius propis de l'art romànic i també del gòtic.

Aquesta visita m`ha semblat molt interessant ja que fins ara, quan veia un monestir, no em podia imaginar que tingués tanta història al seu darrera. Ara imagino les persones, les guerres, els interessos..., i no tan sols un simple convent on vivien monges. Darrera de tota edificació històrica, tal com diu el seu nom: HI HA UNA HISTÒRIA.


\authorandplace{Marta Ortega}{1r ESO}

\end{news}
\definecolor{color}{rgb}{0.1 , 0.1 , 0.1}

\begin{news}
{2} %columnes
{Camprodon' 10}
{}
{ESO}
{033} %pagesof


Quan vaig arribar em van venir al cap molts records, bons i dolents, i és que l'any anterior ja havíem anat a aquella mateixa casa. Els monitors eren els mateixos menys una noia, que era nova. L'Ignasi ens va rebre molt bé. Per començar vam anar a deixar les motxilles a la sala polivalent, després vam anar als porxos a escoltar les normes de la casa. Encara que ja les sabíem de l'any anterior, les vam escoltar molt atents, i després vam tornar  a la sala polivalent a buscar les motxilles per distribuir-nos  les habitacions.

 Jo vaig estar molt bé amb la gent que em va tocar a l'habitació, ja que són persones amb les quals m'hi trobo bé. A nosaltres ens va tocar una habitació que estava al passadís de tots els nens de la nostra classe. Ens van deixar un temps per fer-nos el llit, i després l'Ignasi va xiular amb el seu xiulet. Això volia dir que havíem d'anar tots a la sala polivalent, i així ho vam fer. 

\columntitle{lines}
{Quan vaig arribar em van venir al cap molts records, bons i dolents, i és que l'any anterior ja havíem anat a aquella mateixa casa}

Un cop els 63 vam ser allà dins, l'Ignasi ens va proposar un joc molt divertit: un es tapava els ulls amb un mocador, llavors una persona (tant de primer com de segon d' ESO) sortia i el dels ulls tapats havia d'endevinar qui era aquella persona només tocant-li la cara i els peus. Després vam fer un altre joc: dues persones (normalment un era el més alt de la classe i l'altre el més baixet) havien de portar una patata amb el front fins a l'altra banda de la sala. Quan vam acabar aquests jocs, va arribar l'hora d'anar a sopar. 


El menjar era molt bo. Cada dia parava i desparava taula un grup de gent diferent, perquè així el treball quedava més ben repartit. 

El dimecres al matí, l'Ignasi i els tres monitors ens van explicar el que faríem durant aquell dia. Faríem diferents activitats;  tir amb arc, fer un recorregut amb bicicleta, rocòdrom i jocs d'enginy. A mi la que em va agradar menys va ser el rocòdrom. Aquestes quatre activitats les vam fer durant tot el dia.

{\noindent\includegraphics[width=9cm,keepaspectratio]{eso/img/Camprodon.JPG}}

A la nit no vam poder fer el joc que tenien preparat ja que nevava. Però per substituir-lo ens van posar una pel·lícula. 

Al dia següent va tocar l'excursió per la muntanya. Vam fer 17quilòmetres en total 8 hores caminant. Gràcies a aquesta excursió em vaig adonar que tots tenim més resistència del que ens pensem. Quan vam arribar a la casa, la Davínia i l'Àlex van jugar a pedra, paper o tisora per decidir qui anava primer a dutxar-se, si les noies o els nois, i vam guanyar les noies. 

Les noies vam anar a berenar xocolata calenta amb melindros, i els nois se'n van anar a dutxar. 

Quan tots vam estar dutxats, vam anar a sopar per agafar forces; a la nit hi havia discoteca! Quan va començar la disco em vaig quedar molt sorpresa perquè, després de la caminada que havíem fet,  la gent ballava com boja. Allà va ser on vaig saber com és la gent de veritat. Hi ha gent que em va sorprendre molt, perquè mai no m'hauria imaginat que arribarien a ballar tant. 

La veritat és que jo també em vaig sorprendre  perquè mai havia ballat tant com aquella nit. 

I al dia següent vam tornar cap a casa.

\authorandplace{Andrea Ibáñez}{2n ESO}

\end{news}

\newssep
\definecolor{color}{rgb}{0.1 , 0.1 , 0.1}

\begin{news}
{3} %columnes
{Jornada Esportiva}
{El dissabte dia 15 de maig,  es va fer una jornada esportiva en què es jugaven una sèrie de partits de futbol entre diferents escoles}
{ESO}
{039} %pagesof


La classe de 3r d’ESO, a més, vam organitzar un servei de mini-bar per finançar el viatge de fi de curs  ( quan acabem l'escola) i fer més amè aquell dia. La nostra oferta no era molt extensa però sí de qualitat, ja que els productes els havíem fet nosaltres, intentant que el menjar ajudés a suavitzar i gaudir d’un dia calorós d’escola. També, per facilitar la compra d’aquests productes, vam sortir a la pista a oferir-los . Hi havia : entrepans d'embotit i de formatge, refrescos diversos , cafès i tallats.

El servei de mini-bar el teníem instal.lat a l’entrada del menjador; a més, disposàvem de la cuina per fer els entrepans. Per tal d’organitzar-nos millor, vam fer grups de la classe que havíem de cobrir torns d’una hora i quart. D’aquesta manera, cadascú venia al torn que li tocava.

Des de la classe de 3r d’ESO de l’Escola Solc, us donem les gràcies per la vostra col·laboració i comprensió. 

\authorandplace{Raquel Madrenas}{3r ESO}


\end{news}
\definecolor{color}{rgb}{0.1 , 0.1 , 0.1}

\begin{news}
{3} %columnes
{Sortida a la fundació Miró.  Exposició MURALS}
%index: Fundació Miró: Exposició Murals
{ \noindent\includegraphics[width=9cm,keepaspectratio]{eso/img/JACOB_DAHLGREN.jpg}
\noindent\includegraphics[width=9cm,keepaspectratio]{eso/img/UTR_CREW.jpg}}
{ESO}
{31} %pagesof

Els nois i noies de 2n d’ESO,el dia 9 d’abril, vam anar a la Fundació Joan Miró per visitar l’exposició ‘MURALS’.

En arribar vam esmorzar i després ens van separar en dos grups. Cada grup tenia un guia, el del segon grup es deia David.

Abans de començar,  en David ens va explicar que els primers murals que es van pintar van ser els dibuixos de les coves, a la prehistòria; i que aquests ens han permès estudiar la vida d’aquella època.
També ens va explicar que Joan Miró va pintar murals, i que durant el gòtic el nombre de murals va disminuir a causa de la construcció d’edificis amb grans vitralls.

El primer mural que vàrem veure va ser pintat per unes dones de Diadji Bine Gan de Ga. Allà les dones són les que ho controlen tot i pinten per decorar les seves cases.
Els colors que utilitzen són els colors típics de Marraqueix (grocs, ocres, marrons...), també pinten amb colors vius i ho fan de manera senzilla i detallada.
Pinten els murals amb les mans i els dits,i fabriquen els colors barrejant aigua amb determinats pigments.

El segon mural treballava la geometria i la precisió. A diferència de l’anterior, aquest no era improvisat, cada figura havia estat mesurada i feta utilitzant el regle i el nivell.
El pintor pinta amb una precisió perfecte un tema abstracte.
Cada figura mesura la seva alçada. Amb aquesta obra Lothar Götz, vol representar un record de la seva infantesa.

El tercer mural ens va donar la impressió que allò no era un tipus d’art. Era un mural fet amb plantes (heures). L’artista mexicà només treballa amb plantes com a protesta ecologista.
L’obra s’anomena ‘Contemplant la invasió’, on l’espectador assegut en un banc i a través d’un vidre, contempla la invasió d’heura, que ocupa tot el vidre i que anirà enfilant-se per l’edifici.
L’obra s’havia començat a preparar amb un any d’antelació.

El quart mural interpretava un petit temple budista. L’artista va demanar a la Fundació que pintessin la sala de color vermell i que tapessin el terra amb moqueta vermella. Ell va crear l’obra utilitzant unes plantilles amb forats i pols d’arròs. (Utilitzant una tècnica semblant al puntillisme).

La cinquena obra que ens van ensenyar era un graffiti tradicional. En David ens va explicar que els primera graffitis eren signatures (TAG) i que el pintor és de Singapur, on valoren molt el civisme, per tant si enxampen algú fent un graffiti el penalitzen amb una multa.

El sisè mural era un homenatge al graffiti, cada dibuix representa diferents sentiments, emocions, opinions...
El dibuix d'un puny representa l'esforç de fer un graffiti. Unes càmeres de vigilància dibuixades a un racó representen la vigilància de la ciutat, i el cor al centre és l'emoció i la por que sent l'artista quan pinta un graffiti per la ciutat.

El següent mural era un dibuix fet amb pintura blanca sobre un fons pintat prèviament en negre (com si fos una pissarra). A un costat hi havia representades les pors del pintor i a l'altra banda dibuixos de platets voladors i abduccions alienígenes, una por molt americana.

A continuació Núria i Tono. Són els artistes del vuitè mural.
Representen el significat del graffiti.
Els seus graffitis s'integren a les superfícies on els pinten. Aquests dos artistes han creat també graffitis lumínics. Pinten de forma legal, amb permís. I normalment la gent els conserva, els graffitis. El seu treball dins l'exposició representa figures geomètriques pintades amb pinzell amb diferents gradacions de color.

L'avantpenúltim mural és un paisatge tan en el fons com en el primer pal. Una dent de lleó. L'artista l'ha dibuixat en blanc i negre amb la intenció  que cadascú mentalment li posi els colors que vulgui.

El desè mural és una cobra de collage realitzada per Ludovica Gioscia. Per fer aquesta obra , l'artista va haver de calcular-ho tot. La mida dels papers, roba, colors, ...
És un treball molt elaborat i fet de capes.

La darrera obra estava formada per dianes penjades que omplien tot el pany de paret.
L'artista ha volgut jugar amb la norma de no poder tocar els murals. El fet de posar-hi dianes ens convida a atacar l'obra d'art amb dards.

La meva opinió sobre aquesta exposició és que és molt original, ja que sovint no trobem que el graffiti sigui entès com un tipus d'art dins d'un museu.

							Davínia Vilagrassa 2n ESO


\end{news}
\definecolor{color}{cmyk}{0.5, 0, 1, 0.5}

\begin{news}
{3} %columnes
{Viatge 4T D’ESO 2010}
%index: 4t d'ESO a Polònia
{
\noindent\includegraphics[width=18cm,keepaspectratio]{eso/img/polonia_P5050116.JPG}
}
{ESO}
{036} %pagesof


Diumenge 2 de maig al matí, per no dir la matinada, ens trobàvem arrossegant les maletes i els nervis per l’aeroport del Prat

%\noindent\includegraphics[width=9cm,keepaspectratio]{eso/img/polonia_P5050106.JPG}
A poc a poc deixàvem enrere la bella Barcelona i les no tan belles matemàtiques, literatures i d’altres, i fèiem camí cap a Polònia; però abans vam parar a Düsseldorf, una ciutat alemanya amb un aeroport preciós on ens vam acomiadar dels euros per quatre dies. Havent dinat vam embarcar per segona vegada, ara ja sí cap a Cracòvia, una ciutat grisa i castigada encara que preciosa.

Era dilluns però no ens feia mandra llevar-nos a quarts de vuit; no teníem per endavant un dia laboral ni d’escola, sinó una visita a la ciutat on ens allotjàvem: el centre, Kazimierz (el barri jueu), l’antic Ghetto, el pujolet de Babel (que no tenia res a veure amb la mítica Torre de Babel) on per sobre la ciutat regnava la Catedral, la Universitat on es formà Copèrnic, etc. Vam dinar pel centre, uns asseguraven un bon àpat a alguna pizzeria (el que fa la globalització) i d’altres s’atrevien amb la gastronomia de la zona. 

A la tarda vàrem visitar les mines de sal de Wieliczka, a 15km de Cracòvia, on hi vam arribar amb l’autocar. Vam fer un recorregut que ens portava per diferents cambres, fins i tot hi havia una capella, la de Sta.Kinga, feta de la mateixa sal de la mina que els propis miners havien construït.

Eren quarts de vuit quan arribàvem a l’hotel, havíem de ser puntuals, el sopar se servia a les vuit. 

L’endemà al matí l’autocar ens duia fins a Zakopane, la capital hivernal de Polònia, un lloc molt turístic però tot i això força rural. Pràcticament vam passar el dia  allà, vam aprofitar per comprar alguns souvenirs i la majoria vam tastar els famosos formatges fumats de la zona, molts ens vam sorprendre, no eren deliciosos com ens els havien  pintat però eren força bons. 

Dimecres 5 ens vam llevar més d’hora, vam canviar la molla Cracòvia per la freda Auschwitz. Una visita guiada ens portava pels blocs i carrers de la ciutat d’extermini nazi, passant pels crematoris que encara continuaven drets i les cambres de gas. 

Vam estar a Auschwitz I i a Auschwitz II-Birkenau, de 90 i 147 hectàrees respectivament. 

Vam tornar a Cracòvia a l’hora de dinar i a la tarda vam estar passejant i fent les últimes compres, els últims berenars o cafès amb zlotis. 

De sobte, es va fer dijous i havíem de passar altre cop per l’aeroport de Düsseldorf i acabar de gastar les últimes monedes poloneses; així que vam tornar als entrepans de l’avió i a les revistes alemanyes indesxifrables. A quarts de vuit del vespre arribàvem a Barcelona deixant enrere Polònia, els seus carrers i places, la fredor general de la gent, la delicada i valenta Cracòvia, i el nostre primer i últim viatge com a classe de 4t d’ESO. 


\authorandplace{Júlia Oliver}{4t d’ESO}

\end{news}
\definecolor{color}{cmyk}{0.5, 0, 1, 0.5}

\begin{news}
{2} %columnes
{Cracòvia 2010}
{}
{ESO}
{035} %pagesof

\noindent\includegraphics[width=9cm,keepaspectratio]{eso/img/polonia_P5030075.JPG}

Tots sabíem on anàvem. Si més no, ho havíem decidit nosaltres. Polska, Kraków, Polònia, Cracòvia. Aquesta era la nostra destinació, un món ple d’homes i de dones rossos amb els ulls blaus, un món plujós, poc acostumat a veure un dia totalment assolellat i de viure les caloroses temperatures mediterrànies, un idioma impossible d’entendre, una moneda estranya. Sabíem que hauríem de fer quatre hores d’avió més l’espera a l’aeroport, i també que valdria la pena perquè ens esperava un viatge inoblidable. 

Vam arribar a la nostra destinació, cansats però amb ganes d’instal·lar-nos a l’hotel i començar a viure aquells cinc dies del viatge de fi de curs amb intensitat.

Durant aquests cinc dies, vam passejar per els carrers humits de Cracòvia, vam observar encuriosits els racons més especials i bonics d’aquella petita ciutat, vam provar el menjar típic, com per exemple el formatge fumat, per alguns molt bo, per altres no tant. També visitàrem Zakopane un petit poble hivernal, amb unes vistes precioses. Vam viure l’experiència esgarrifant de endinsar-nos en el Camp de Concentració d’Auschwitz, visitar unes mines de sal sota terra, on vam quedar totalment bocabadats al veure que hi havia escultures totalment de sal a 135 m de profunditat. 


Tampoc podrem oblidar les nits perdudes pel barri jueu on es trobava el nostre hotel, anant a un parc d’atraccions o simplement passejant, i tornant cap a l’hotel, pels carrers que dia a dia ja s’anaven fent nostres, mentre que la pluja ens mullava (factor que ens va acompanyar durant tot el viatge, com era previst), però ens era igual, perquè l’important era que estàvem junts, i volíem aprofitar tots els minuts d’aquells dies per fer d’aquell viatge que tancaria el nostre cicle a l’Escola Solc, un record inesborrable.

\authorandplace{Rosa Gras}{4t ESO}

\end{news}

\newssep
\definecolor{color}{cmyk}{0.5, 0, 1, 0.5}

\begin{news}
{3} %columnes
{Aprenem a votar}
{La nostra escola ha passat a formar part del col·lectiu de 40 escoles i instituts que a tot Catalunya portaran a terme l’experiència d’innovació pedagògica “Aprenem a votar”. 
}
{ESO}
{33} %pagesof

\noindent\includegraphics[width=5.5cm,keepaspectratio]{eso/img/aprenem_votar.jpg}

El programa es desenvoluparà durant el primer trimestre del curs 2010-2011 i té com a finalitat que l’alumnat de 4t d’ESO aprengui a exercir lliurement el seu dret al vot de cara al futur, a partir d’un procés d’informació de la mecànica electoral i d’una simulació de les eleccions.

El projecte està promogut pel Departament de Didàctica de les Ciències Socials i té el suport del Departament d’Educació de la Generalitat de Catalunya.


\bf Estructura del projecte  

\rm
El projecte es divideix en cinc fases:

Unitat 1. Elegim. Es tractarà de conèixer els diferents elements del sistema democràtic català, i del Parlament en particular, i descriure’n el funcionament general.

Unitat 2. Ens informem. Es tractarà de reconèixer i utilitzar de manera crítica les diverses fonts d’informació que es posaran a disposició dels nois i noies durant la campanya electoral.

Unitat 3. Opinem.  L’alumnat podrà opinar sobre els diferents partits i coalicions catalans, tenint en compte les preocupacions i les opcions que presentaran durant les eleccions.

Unitat 4. Participem. Incrementarà la informació sobre les diferents maneres de participar en un sistema democràtic.


Unitat 5. Votem. Es tractarà d’aprendre a organitzar una votació tot respectant els criteris que fan possible la celebració d’unes eleccions lliures i democràtiques. L’alumnat participarà en una simulació el divendres anterior al dia de les eleccions al Parlament de Catalunya. 


En una segona part d’aquest programa, es reflexionarà sobre els resultats oficials de les eleccions al Parlament en comparació amb el vot estudiantil de quart d’ESO

Departament de Ciències Socials – Educació Secundària Obligatòria

\end{news}
\definecolor{color}{cmyk}{0.5, 0, 1, 0.5}

\begin{news}
{2} %columnes
{VideoTrams}
{El grup de guionistes, format per quatre persones i reforçat per part de l'equip de direcció, van desenvolupar una història basada en el tema principal triat per la classe:“Embarassos a l'adolescència” }
{ESO}
{39} %pagesof


Tik, Tok, s'ha acabat el temps, el nostre vídeo ja està enviat a l'altra escola (és un treball que fem conjuntament amb les escoles de TRAMS). Ha estat un treball en grup complicat que alhora hem sabut tirar endavant. Aquest vídeo ha estat un treball transversal, ja que en fer-lo  hem treballat temes de les àrees de tutoria i ciutadania, d'educació audiovisual, d'informàtica i música. 

En els primers moments,  vam dividir-nos en grups segons la funció que volíem realitzar en el videotrams. El grup de guionistes, format per quatre persones i reforçat per part de l'equip de direcció, van desenvolupar una història basada en el tema principal triat per la classe: “Embarassos a l'adolescència”. Un cop exercida la feina dels guionistes, van començar els càstings per triar els actors i les actrius que portarien a terme els papers principals de la història.



Els guionistes, juntament amb l'equip directiu,  van donar el seu punt de vista sobre la història perquè els actors i actrius l'enfoquessin com ells creien. Després d'assajos, va arribar l'hora que el grup de decorats pensés com poder adaptar les diferents aules de l'escola d'acord amb l'escenari. Un cop tot adaptat i amb el guió après va ser l'hora de posar-se a gravar. 

\noindent\includegraphics[width=9cm,keepaspectratio]{eso/img/videotrams_3.jpg}

Els càmeres se les van empescar per introduir diferents plans  al llarg de tota la història.  Quan va estar gravat, l'equip d'edició de vídeo amb la col·laboració de la Lluïsa  van editar-lo i enviar-lo a l'escola corresponent. El resultat? Un vídeo que mostra un treball excepcional de grup en una activitat més lúdica  del que és habitual.

\authorandplace{Esteve Pérez i Ariadna Peiró}{3r d'ESO}

\end{news}

\newssep
\definecolor{color}{cmyk}{0.5, 0, 1, 0.5}

\begin{news}
{3} %columnes
{Concert a l'Escola Sant Gervasi}
{Els nois i noies de 1r d’ESO hem anat a l’escola Sant Gervasi de Mollet del Vallès a fer un concert conjuntament amb els de 1r d’ESO d’aquesta escola}
{ESO}
{60} %pagesof


Quan vàrem arribar, els nois i noies de l’escola Sant Gervasi ens van donar la benvinguda i vàrem esmorzar junts al seu pati.

Després d’esmorzar vàrem anar a la sala d’actes, on cada grup va assajar la seva part per al concert.

Més tard varen venir nens i nenes de 6è de primària, per poder escoltar i participar del concert.

Un cop acabat el concert, vàrem anar a dinar junts i a jugar una estona. Aquesta estona junts va servir per poder-nos conèixer i apuntar-nos els números de telèfon, l’adreça del Facebook i correus electrònics.

\authorandplace{}{ 1er d'ESO }

\noindent\includegraphics[width=12cm,keepaspectratio]{eso/img/sant_gervasi5.jpg}


\end{news}
\definecolor{color}{cmyk}{0.5, 0, 1, 0.5}

\begin{news}
{2} %columnes
{Sant Jordi}
{\noindent\includegraphics[width=18cm,keepaspectratio]{eso/img/st_jordi_bates.jpg}}
{ESO}
{500} %pagesof

{\noindent\includegraphics[width=9cm,keepaspectratio]{eso/img/st_jordi_taulell.jpg}}

El dia de St.Jordi,  l'alumnat de 3r d'ESO ens vam encarregar de preparar-ho tot.
Alguns alumnes van venir una hora abans per muntar les parades abans que arribessin els pares per comprar alguns llibres. Al cap d'una hora van arribar els altres alumnes, i un grup de nosaltres se'n va anar a muntar les activitats per als més petits. 
A la parada, van anar venint tots els alumnes, primer els més petits i després els més grans, per comprar-nos molts llibres i polseres.

Els nens de parvulari s'ho van passar molt bé picant les pinyates que els vam preparar. Els vam tapar els ulls i els vam donar un pal perquè les trenquessin. 

{\noindent\includegraphics[width=9cm,keepaspectratio]{eso/img/st_jordi_pinguins.jpg}}

%{\noindent\includegraphics[width=9cm,keepaspectratio]{eso/img/st_jordi_carablanca.jpg}}

%{\noindent\includegraphics[width=9cm,keepaspectratio]{eso/img/st_jordi_taulell.jpg}}


Els nois i noies de 2n de primària també es van divertir molt amb les pinyates, però també fent la prova dels “filipinos”, que consisteix en passar un fil per un “filipino” i anar menjant el fil fins arribar al “filipino” i menjar-se'l.
Els alumnes de 3r i 4t de primària van fer molts jocs variats com el dels “filipinos”, llençar anelles a un pal, buscar un caramel amb la boca dins d'un bol amb farina, portar ous en una cullera,...

Volem agrair a tot l'alumnat, al professorat i a tots els pares i mares que van contribuir , comprant llibres,  al nostre futur viatge de 4t d’ESO   

\authorandplace{Artur Martínez}{3r ESO}
\end{news}
