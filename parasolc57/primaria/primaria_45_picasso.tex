\begin{news}
{2} %columnes
{Visita al museu Picasso}
{}
{Primaria}
{030} %pagesof


El dimarts 23 de març vam anar al Museu Picasso, que es troba al Barri Gòtic de Barcelona, al carrer Montcada, que és un carreró estret, antic i sense cotxes. Allà vam aprendre moltes coses d’aquest pintor malagueny.  

Quan hi vam entrar jo em vaig sentir com un veritable artista. Ens van explicar la vida de Pablo Picasso, la del seu pare (José Ruiz) i la de la seva mare (Maria Picasso). Ens van dir que Picasso va començar a dibuixar amb carbonet, que va presentar el quadre de “La dona malalta” a la universitat de Belles Arts de Madrid, que el seu pare era  professor en una universitat que es deia “Llotja”. També ens van ensenyar quatre estils de la seva pintura: carbonet, l’època blava, figures geomètriques i el puntillisme. 

Vam veure el quadre “La dona malalta” que, segons ens van explicar,  representava una dona molt malalta al llit amb un metge prenent-li el pols i una monja oferint-li un got d’aigua, aquesta portava un nadó a les mans. En realitat, el metge era el pare del Picasso, la dona malalta, una pidolaire, el nadó, el fill de la pidolaire i, la monja, era un amic de Picasso.

A més a més de veure els quadres més importants d’aquest pintor, em va agradar poder llegir la seva vida, que estava escrita a diverses parts del museu. 

\authorandplace{Óscar Redondo Mora}
				{6è de Primària}

\noindent\includegraphics[width=8cm,keepaspectratio]{primaria/img/picasso_julia.jpg}

\noindent\includegraphics[width=8cm,keepaspectratio]{primaria/img/picasso_raquel.jpg}

\end{news}
