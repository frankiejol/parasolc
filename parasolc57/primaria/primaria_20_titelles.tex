\begin{news}
{2} %columnes
{Elaboració dels titelles}
{Aquest curs, els nens de 5è de Primària estem preparant unes obres de titelles per representar-les al parvulari}
{Primaria}
{028} %pagesof




Primer havíem de portar un globus i un cilindre de cartró. Al voltant del globus enganxàvem paper amb cola. La cola feia que el cilindre s’ajuntés amb el globus. Quan es va eixugar el paper vam tornar a posar més capes de paper amb cola perquè ens quedés l’estructura més forta. Uns dies després, quan ja s’havia assecat, vam pintar el titella amb pintura blanca perquè no es veiés el diari. 

\columntitle{lines}
{Els titelles els fem nosaltres i les obres també les triem nosaltres}

Un altre dia el vam pintar amb els colors del personatge; per exemple, el meu  era un ós i el vaig pintar de color marró. Uns dies més tard, vam portar fil, agulla i tela per fer el vestit al titella. Tallàvem la tela en forma de vestit i després la cosíem. Fet això, el Ferran enganxava el vestit al cap amb belcro. Ja quedava fet el titella. També vam fer el guió de l’obra i un còmic per fer les fotografies. Aquestes ens serviran per fer un ``còmic digital" amb l’ordinador. Cap al juny representarem l’obra a parvulari i a la festa de final de curs podrem veure els ``còmics digitals".


M’agrada molt fer plàstica  sobretot, en aquesta ocasió, perquè hem treballat fent titelles i fent el ``còmic digital".

\authorandplace{Mar Garcia}
{5è de Primària}

\noindent\includegraphics[width=9cm,keepaspectratio]{primaria/img/titlles_513.jpg}

\end{news}
