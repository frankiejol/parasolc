\begin{news}
{2} %columnes
{L'ós}
{}
{Primaria}
{029} %pagesof

\noindent\includegraphics[width=9cm,keepaspectratio]{primaria/img/os.png}

Els óssos són omnívors, ara bé, l'ós polar, a causa de la manca d'altres fonts d'aliment, té una dieta quasi exclusiva de carn. Amb els seus pesats cossos i les seves poderoses mandíbules, els óssos es compten entre els majors carnívors que viuen a la Terra. Un mascle d'ós polar pot pesar més de 600 kg i pot aconseguir 1,60 m. d'altura quan es posa de quatre potes. Es desplacen amb uns moviments pesats tot  recolzant tota la planta dels peus. Posseeixen orelles curtes i cua rudimentària. Tenen una dieta variada i, tot i la seva temible dentadura, molts d'ells mengen fruits, arrels i insectes, a més de carn. 
Contràriament al que pot semblar a primera vista, ha estat molt debatut si els pandes són o no óssos, encara que els últims exàmens genètics suggereixen que els grans pandes sí que ho són.


L'ós era un animal sagrat al nord d'Europa, on era el rei del bosc. Se'l representa en els mites i llegendes com l'animal més semblant a l’home; fins i tot, arribava a tenir relacions amb dones humanes (fet pel qual l'església cristiana el va degradar com a associat al dimoni a partir de l'any mil). El nom propi Úrsula deriva directament de l'ós. Representa els països de Finlàndia i Rússia i és un emblema de la comunitat de Madrid.

\authorandplace{Arnau Ruiz}{6è de Primària}




\end{news}
