\begin{news}
{2} %columnes
{Conferència de la Josefina Piquet a l'Escola}
{\noindent\includegraphics[width=18cm,keepaspectratio]{primaria/img/dones_36_p4230014.jpg}}
{Primaria}
{019} %pagesof




La passada diada de Sant Jordi, la senyora Josefina Piquet, coordinadora de l'associació “Dones del 36” va visitar l'escola per compartir les seves vivències amb els nois i noies de 4t.

La sessió fou molt emotiva i profitosa, tant des del punt de vista de l'educació en valors com de l'educació en ciències socials.

La nostra escola sempre ha tingut com a una de les seves prioritats transmetre  la memòria històrica del nostre país. En aquest sentit, la valoració de les repercussions de la nostra guerra fa possible analitzar la seva influència, tant en les generacions que la van viure directament, com en la societat actual. D'altra banda, treballar sobre el fenomen de la guerra contribueix a avançar en el debat sobre la resolució pacífica dels conflictes.

Amb xerrades com la de la Josefina Piquet (per cert, quan ve a l'Escola Solc diu que “es troba com a casa”) intentem recuperar la història recent de les dones mitjançant la seva pròpia veu, la seva memòria individual i col·lectiva; destacar la importància i la complexitat de la història transmesa oralment, de la seva metodologia i dels seus resultats i analitzar la diversitat de protagonistes i d'escenaris en la història de la vida quotidiana, especialment en moments excepcionals com és en temps de guerra, exili i postguerra. 

\end{news}

\newssep
