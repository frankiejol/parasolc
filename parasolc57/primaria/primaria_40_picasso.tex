\begin{news}
{3} %columnes
{Les Menines}
{}
{Primaria}
{30} %pagesof

\noindent\includegraphics[width=6cm,keepaspectratio]{primaria/img/picasso_ariadna.jpg}


Velázquez va ser un gran pintor del segle XVII, tot i que avui en dia les seves obres d’art són molt conegudes. Ell
 va ser l’autor de “Les Menines” que està exposat al museu del Prado, a Madrid. Si observem bé el quadre podem veure-hi el seu autor; al fons hi ha un mirall on s’hi pot veure Felip IV i Mariana d’Àustria, al seu costat José Nieto, prop de Velázquez hi ha una menina, al mig hi ha la infanta Margarida, al seu costat hi té una altra menina, també hi ha una nana, un nan, un senyor i una monja. 


Pablo  Picasso, també un gran pintor del segle XX, va imitar  “Les Menines” a la seva
manera, és a dir,  amb formes geomètriques, amb punts, amb carbonet, va afegir elements, personatges, etc. Va fer unes 58 versions sobre aquest quadre. 


A la nostra escola, els alumnes de 5è i de 6è vàrem fer un treball sobre “Les Menines” abans de la visita al museu Picasso.
Vàrem observar l’original i algunes de les moltes versions de Picasso. Després de parlar-ne, vam dibuixar l’obra a la nostra manera. Per exemple, amb pals, estil modern, còmic, puntillisme,...  En fi, tothom  va fer el seu estil i van quedar  tots molt bonics.

\noindent\includegraphics[width=6cm,keepaspectratio]{primaria/img/picasso_conrad.jpg}

\authorandplace{Guillem Morales}{6è de Primària}

\end{news}

\newssep
