\definecolor{color}{rgb}{0.1 , 0.1 , 0.1}

\begin{news}
{2} %columnes
{El món de la Cèl·lula}
{El Dimecres 17 de març, els nens i nenes de 5è de primària de l’Escola Solc vam anar al CosmoCaixa a fer un taller de ciència que es diu “El món de la cèl·lula”}
{Primaria}
{5} %pagesof



Quan vam arribar al Museu vam conèixer a l’Eli i a la Irene, que van ser les monitores que ens van ajudar a fer les activitats. Hi havia dos tallers: un consistia en preparar mostres per observar pel microscopi i l’altre era mirar mostres ja preparades (orella de gos, testicle de rata,...). A la primera activitat, amb un tomàquet i una ceba havíem de fer les preparacions.
Després, un cop havíem fet tots dos tallers, vam estar una hora voltant pel CosmoCaixa. Jo vaig anar al Bosc Inundat, vaig tocar un bloc de gel gegant que hi havia, ...
M’ho vaig passar molt bé perquè el Museu de la Ciència és molt interessant.


\authorandplace{Albert Morales}
							{5è de Primària}

\noindent\includegraphics[width=9cm,keepaspectratio]{primaria/img/celula_DSC00655.JPG}

\end{news}

\newssep
\definecolor{color}{rgb}{0.1 , 0.1 , 0.1}

\begin{news}
{2} %columnes
{Elaboració dels titelles}
{Aquest curs, els nens de 5è de Primària estem preparant unes obres de titelles per representar-les al parvulari}
{Primaria}
{028} %pagesof




Primer havíem de portar un globus i un cilindre de cartró. Al voltant del globus enganxàvem paper amb cola. La cola feia que el cilindre s’ajuntés amb el globus. Quan es va eixugar el paper vam tornar a posar més capes de paper amb cola perquè ens quedés l’estructura més forta. Uns dies després, quan ja s’havia assecat, vam pintar el titella amb pintura blanca perquè no es veiés el diari. 

\columntitle{lines}
{Els titelles els fem nosaltres i les obres també les triem nosaltres}

Un altre dia el vam pintar amb els colors del personatge; per exemple, el meu  era un ós i el vaig pintar de color marró. Uns dies més tard, vam portar fil, agulla i tela per fer el vestit al titella. Tallàvem la tela en forma de vestit i després la cosíem. Fet això, el Ferran enganxava el vestit al cap amb belcro. Ja quedava fet el titella. També vam fer el guió de l’obra i un còmic per fer les fotografies. Aquestes ens serviran per fer un ``còmic digital" amb l’ordinador. Cap al juny representarem l’obra a parvulari i a la festa de final de curs podrem veure els ``còmics digitals".


M’agrada molt fer plàstica  sobretot, en aquesta ocasió, perquè hem treballat fent titelles i fent el ``còmic digital".

\authorandplace{Mar Garcia}
{5è de Primària}

\noindent\includegraphics[width=9cm,keepaspectratio]{primaria/img/titlles_513.jpg}

\end{news}
\definecolor{color}{rgb}{0.1 , 0.1 , 0.1}

\begin{news}
{2} %columnes
{Colònies de 1r i 2n de Primària}
%index: Colònies de Primària
{\noindent\includegraphics[width=18cm,keepaspectratio]{primaria/img/colonies_246.jpg}}
{Primaria}
{29} %pagesof

Dimarts 6 d’abril vam anar de colònies a una casa que es deia Can Putxet. Vam anar amb autocar  i era una mica lluny.

Quan vam arribar,  ens vam posar molt contents. Tenia un camp de futbol i una pista de tennis que semblava un camp de bàsquet. Vam estar jugant una estona, després vam dinar entrepans; un era de formatge i l’altre de pernil i, de postres, poma. Vam continuar jugant i va començar a ploure,. 

Vam entrar a la casa de dos en dos. A mi em va tocar amb el Joan, va voler dormir a dalt i jo vaig dormir a baix, tenia al Casi al davant i al costat l’Elena.

%\columntitle{lines}
%{També vam fer una excursió de tres quilòmetres i ens vam cansar molt}

Després vam fer unes activitats. La que més ens va agradar  va ser un joc d’equip. Els equips els vam fer segons els colors d’uns pitets  que ens van donar. Jo anava amb l’Alejandra, l’Elena, el Guillem T., el Gullem R., el Pol de primer, l’Aida i l’Oscar. Érem el color groc. Havíem de trobar 10 targetes d’animals. Només en vam trobar 8, perquè una se’ns va perdre i d’una altra ja  no en quedaven. Teníem un mapa per trobar els topants on buscar les targetes.

Per sopar hi  va haver sopa, truita de patates amb croquetes miniatura i per  postres iogurt. També vam fer una excursió de tres quilòmetres i ens vam cansar molt.

L’última nit vam fer un joc de nit, havíem de trobat tres claus: una al bosc, que era molt fàcil i la va trobar el Pol. L’ altra al camp, que va ser molt difícil i la va trobar l’Àlex. L’altra al riu, la va trobar l’Éric.

M’ho vaig passar molt Bé.

\authorandplace{Martina Tarrés Castellanos}{Primària}

\end{news}

\definecolor{color}{rgb}{0.1 , 0.1 , 0.1}

\begin{news}
{2} %columnes
{}
{}
{Primaria}
{029} %pagesof

%\vspace*{1cm}

El dimarts vam anar de colònies amb motxilles o maletes, les vam deixar al camp de futbol i vam començar a jugar. Després vam anar a la casa de can Putxet a veure-la per dintre. Vam començar per les habitacions, després pels lavabos. 

Vam tornar a les habitacions a fer el llit. Quan vam acabar, vam deixar la roba al calaix. Després vam jugar al camp de futbol un altre cop. Vam jugar a pilota. Després vam anar a passejar al camp, vam veure una riera.

Després vam tornar, quan vam arribar vam  anar a sopar. Després vam anar al llit a llegir i tot seguit a dormir. Quan ens vam llevar vam esmorzar i després vam fer una excursión que vam caminar tres quilòmetres. 

Vam berenar i, després, vam tornar a la casa de colònies, quan vam tornar vam sopar i després vam anar al llit, l’endemà marxàvem. Vam esmorzar, després arribava una altra escola, Vam jugar a una carrera de globus i també vam fer bombolles. 
Vam anar a dinar. I, després, vam marxar. 

\authorandplace{Màrio Ródenas Sistané}{Primària}


\end{news}
\definecolor{color}{rgb}{0.1 , 0.1 , 0.1}

\begin{news}
{3} %columnes
{Les Menines}
{}
{Primaria}
{30} %pagesof

\noindent\includegraphics[width=6cm,keepaspectratio]{primaria/img/picasso_ariadna.jpg}


Velázquez va ser un gran pintor del segle XVII, tot i que avui en dia les seves obres d’art són molt conegudes. Ell
 va ser l’autor de “Les Menines” que està exposat al museu del Prado, a Madrid. Si observem bé el quadre podem veure-hi el seu autor; al fons hi ha un mirall on s’hi pot veure Felip IV i Mariana d’Àustria, al seu costat José Nieto, prop de Velázquez hi ha una menina, al mig hi ha la infanta Margarida, al seu costat hi té una altra menina, també hi ha una nana, un nan, un senyor i una monja. 


Pablo  Picasso, també un gran pintor del segle XX, va imitar  “Les Menines” a la seva
manera, és a dir,  amb formes geomètriques, amb punts, amb carbonet, va afegir elements, personatges, etc. Va fer unes 58 versions sobre aquest quadre. 


A la nostra escola, els alumnes de 5è i de 6è vàrem fer un treball sobre “Les Menines” abans de la visita al museu Picasso.
Vàrem observar l’original i algunes de les moltes versions de Picasso. Després de parlar-ne, vam dibuixar l’obra a la nostra manera. Per exemple, amb pals, estil modern, còmic, puntillisme,...  En fi, tothom  va fer el seu estil i van quedar  tots molt bonics.

\noindent\includegraphics[width=6cm,keepaspectratio]{primaria/img/picasso_conrad.jpg}

\authorandplace{Guillem Morales}{6è de Primària}

\end{news}

\newssep
\definecolor{color}{rgb}{0.1 , 0.1 , 0.1}

\begin{news}
{2} %columnes
{Visita al museu Picasso}
{}
{Primaria}
{030} %pagesof


El dimarts 23 de març vam anar al Museu Picasso, que es troba al Barri Gòtic de Barcelona, al carrer Montcada, que és un carreró estret, antic i sense cotxes. Allà vam aprendre moltes coses d’aquest pintor malagueny.  

Quan hi vam entrar jo em vaig sentir com un veritable artista. Ens van explicar la vida de Pablo Picasso, la del seu pare (José Ruiz) i la de la seva mare (Maria Picasso). Ens van dir que Picasso va començar a dibuixar amb carbonet, que va presentar el quadre de “La dona malalta” a la universitat de Belles Arts de Madrid, que el seu pare era  professor en una universitat que es deia “Llotja”. També ens van ensenyar quatre estils de la seva pintura: carbonet, l’època blava, figures geomètriques i el puntillisme. 

Vam veure el quadre “La dona malalta” que, segons ens van explicar,  representava una dona molt malalta al llit amb un metge prenent-li el pols i una monja oferint-li un got d’aigua, aquesta portava un nadó a les mans. En realitat, el metge era el pare del Picasso, la dona malalta, una pidolaire, el nadó, el fill de la pidolaire i, la monja, era un amic de Picasso.

A més a més de veure els quadres més importants d’aquest pintor, em va agradar poder llegir la seva vida, que estava escrita a diverses parts del museu. 

\authorandplace{Óscar Redondo Mora}
				{6è de Primària}

\noindent\includegraphics[width=8cm,keepaspectratio]{primaria/img/picasso_julia.jpg}

\noindent\includegraphics[width=8cm,keepaspectratio]{primaria/img/picasso_raquel.jpg}

\end{news}
\definecolor{color}{rgb}{0.1 , 0.1 , 0.1}

\begin{news}
{2} %columnes
{La Bombeta}
{}
{Primaria}
{029} %pagesof

\noindent\includegraphics[width=9cm,keepaspectratio]{primaria/img/bombeta.png}

L’invent de la bombeta s’atribueix a Thomas Alva Edison que va presentar la patent  el 21 d’octubre de 1878. Està formada per un filament de wolframi molt primet situat a l’interior d’una “ampolla” de vidre que s’ha omplert posteriorment d’un gas inert o bé se li ha fet el buit. A la part de sota de la bombeta hi ha una part metàl·lica on s’hi troben totes les connexions elèctriques. Aquesta part metàl·lica, normalment té una una rosca que permet caragolar la bombeta al portalàmpades. La bombeta elèctrica també s’anomena làmpada incandescent.  De tota l’energia que consumeix la bombeta, només el 10\% es converteix en llum; la resta, es transforma en escalfor, llum ultraviolada i llum infraroja. 
La bombeta és un dels objectes més utilitzats des de la seva invenció, ja que tots en tenim, com a mínim, una a casa.

\authorandplace{Conrad Galli}{6è de Primària}

\end{news}

\newssep
\definecolor{color}{rgb}{0.1 , 0.1 , 0.1}

\begin{news}
{2} %columnes
{L'ós}
{}
{Primaria}
{029} %pagesof

\noindent\includegraphics[width=9cm,keepaspectratio]{primaria/img/os.png}

Els óssos són omnívors, ara bé, l'ós polar, a causa de la manca d'altres fonts d'aliment, té una dieta quasi exclusiva de carn. Amb els seus pesats cossos i les seves poderoses mandíbules, els óssos es compten entre els majors carnívors que viuen a la Terra. Un mascle d'ós polar pot pesar més de 600 kg i pot aconseguir 1,60 m. d'altura quan es posa de quatre potes. Es desplacen amb uns moviments pesats tot  recolzant tota la planta dels peus. Posseeixen orelles curtes i cua rudimentària. Tenen una dieta variada i, tot i la seva temible dentadura, molts d'ells mengen fruits, arrels i insectes, a més de carn. 
Contràriament al que pot semblar a primera vista, ha estat molt debatut si els pandes són o no óssos, encara que els últims exàmens genètics suggereixen que els grans pandes sí que ho són.


L'ós era un animal sagrat al nord d'Europa, on era el rei del bosc. Se'l representa en els mites i llegendes com l'animal més semblant a l’home; fins i tot, arribava a tenir relacions amb dones humanes (fet pel qual l'església cristiana el va degradar com a associat al dimoni a partir de l'any mil). El nom propi Úrsula deriva directament de l'ós. Representa els països de Finlàndia i Rússia i és un emblema de la comunitat de Madrid.

\authorandplace{Arnau Ruiz}{6è de Primària}




\end{news}
\definecolor{color}{rgb}{0.1 , 0.1 , 0.1}

\begin{news}
{4} %columnes
{L’Ot i els alumnes de 3r de Primària}
% index: L'Ot i 3r de Primària
{}
{Primaria}
{15} %pagesof


\noindent\includegraphics[width=4.5cm,keepaspectratio]{primaria/img/ot_imagen.jpg}

\bf L’Ot va a Egipte

\rm
Un dia l’Ot va anar amb l’escombra màgica de passeig i va arribar a Egipte, i mentre volava es va trobar a tres àrabs. Es veia que aquells senyors volien anar a  algun lloc. Finalment l’Ot va dir unes paraules màgiques: PLOP!! I, de sobte, l’escombra es va convertir en una catifa voladora. Així van poder pujar els tres àrabs que van quedar contents i resant cap al poble.

                                      Eloi Ibáñez           


\noindent\includegraphics[width=4.5cm,keepaspectratio]{primaria/img/ot_imagen_001.jpg}

\bf L’Ot i la senyora globus

\rm
Hi havia una vegada un bruixot que es deia Ot i també una princesa anomenada Berta. Un dia la Berta era en el seu castell i va aparèixer l’Ot i li va dir: 
-   Hola princeseta, estàs bé? 
Sí, va dir la Berta; i tu què fas per aquí?
Volia saludar-te i  veure com estaves.
Doncs estic molt bé.
Aleshores, l’Ot va agafar la Berta per les cames i la va inflar com un globus i va fer PLOP! Llavors va posar-li una corda i va agafar-la tot caminant... I així és la història de l’Ot i la Berta.
     
                                                           Adriana Bosch


\noindent\includegraphics[width=4.5cm,keepaspectratio]{primaria/img/ot_img025.jpg}

\bf L’Ot i les amigues de la Berta

\rm
Un dia l’Ot s’estava banyant i quan va acabar de banyar-se va agafar la tovallola i es va eixugar. Va obrir la porta i va trobar les amigues de la Berta. Aleshores, perquè no el veiessin, va fer ús de la seva màgia i va caminar pel sostre. Va anar a l’altra porta i les amigues de la Berta no el van veure.

                                                                Irene Cuadra

\noindent\includegraphics[width=4.5cm,keepaspectratio]{primaria/img/ot_img022.jpg}

\bf La gran barba

\rm

Una vegada hi havia un bruixot que es deia Ot i anava cap a la perruqueria. Llavors va dir: - no tinc cabells perquè me’ls tallin, doncs m’hauré de tallar el bigoti i això no m’agradaria, seria imperdonable. Aleshores, com que era un bruixot, va fer-se aparèixer una gran barba a la barbeta. Va entrar i no va sortir fins al dia següent, però no li van tallar la barba sinó el bigoti. Va sortir enfadat; en canvi,  de content , gens ni mica.

                                                             Ferran Corral 


\end{news}
\definecolor{color}{rgb}{0.1 , 0.1 , 0.1}

\begin{shortnews}
{3} %columnes
{Ressenyes de llibres}
{}
{Primaria}

\shortnewsitem{La marmota inventora}
{
Autor: Enric Larreula / Rita Culla.   Editorial: La Galera 

Hi havia una vegada una marmota molt llesta que un dia va fer un invent, una bufanda. La va tallar a trossets i la va regalar a totes les marmotes. La bufanda la va fer servir per dormir. La marmota vivia en unes muntanyes molt altes anomenades els Alps.

A mi m’ha agradat molt perquè tracta d’invents i de marmotes.
									
Biel Castro, 1r de Primària
}

\shortnewsitem{La campaneta de plata}
{
Autor: Conte popular.   Editorial: Susaeta

En una torre hi vivia la mare dels vents, que tenia tres fills: el vent del Nord, la Brisa i el vent de Llevant. Quan es van fer grans van viatjar per diferents mons, i van deixar una campaneta de plata a la mare per si els volia cridar.  Com que la mare es sentia sola, va tocar la campaneta per reunir els tres germans, però de tant vent, el poble es pensava que a la torre hi vivia una bruixa.

M’ha agradat aquest llibre perquè la mare reuneix els seus fills tocant la campaneta.
									
Aida Perez, 1r de Primària
}

\shortnewsitem{Pinotxo}
{
Autor: Carlo Collodi      
\noindent\includegraphics[width=4.5cm,keepaspectratio]{primaria/img/llibres_pinocho.jpg}

Un dia un fuster anomenat Gepetto va construir un titella, que va anomenar  “Pinotxo”. Va desitjar que  ell tingués vida. A la nit va venir una fada i va fer que el titella cobrés vida. Pinotxo es va sorprendre molt perquè podia parlar, caminar, menjar, anar a l’escola, etc. Pepito Grillo era la consciència de la marioneta, que el va ajudar a ser conscient del que feia. 

Em va agradar molt perquè la fada va fer màgia i a mi això m’agrada molt. Aquest llibre el recomano perquè és molt divertit, emocionant i hi passen moltes aventures.

Pol Hernández,  2n de Primària
}

\shortnewsitem{El mag Merlí}
{
Autor: Llegenda Popular.    Editorial: Susaeta  

Un nen anomenat Grill vivia en un castell a Alemanya. Un dia en Grill amb un amic que es diu Key van anar a caçar, però com que en Grill era novell  en la caça, va caure des d’un arbre a sobre d’en Key fent que llancés una fletxa al cel; en veure el que havia fet, en Grill li va dir que aniria a cercar la fletxa. En Grill anava cercant la fletxa quan, de sobte, va veure la fletxa en una branca, l’anava a agafar,  va relliscar i va caure en una cabana on hi vivia en Mag Merlí.  Li va dir que havia de  tornar al castell...    

A mi el que més m’ha agradat és quan llençava la fletxa. I,quan estan caçant, hi ha un dibuix que em va agradar molt.

Damià Rubió ,  2n de Primària
}

\shortnewsitem{La Xola i els lleons}
{
Autor: M. Dolors Alibés.   Editorial: Cruïlla

És una gossa que es diu Xola i que es pensa que es un lleó, llavors l’imita. El seu amo,  en Marc, té un amic que investiga la selva, llavors l’amic porta un llibre de lleons a casa la Xola i se’l deixa. La gosseta, com que està tan interessada en els lleons, l’agafa i el llegeix i descobreix que els lleons no són tan fantàstics com ella creia i ja no vol ser un lleó. I ara és una gosseta petonera com abans era.

M’ha agradat molt perquè és molt misteriós i no saps què passarà. També perquè mai  pots decidir ser un lleó si no saps res sobre ells.

Andrea Amador Alcaina,  2n de Primària
}

\shortnewsitem{Geronimo Stilton. Les entranyes de les rates pudents}
{
Autor: Geronimo Stilton.   Editorial: Destino

Aquest llibre parla que Ratalona comença a fer molta pudor. Una nit, de sobte,  de les clavegueres surten globus pudents. En Gerónimo baixa a les clavegueres per esbrinar què feia tanta pudor. Caminant per les clavegueres, arriba a Ratcity, la ciutat de les rates. Hi ha una rata que es diu Clavegueram que és una mica dolenta, i quan arriba a Ratcity s’enamora del Xafarot, l’amic d’en Geronimo i es vol casar amb ell. En Xafarot i en Geronimo agafen una moto d’aigua i s’escapen.

Ens ha agradat molt perquè aquest llibre és de misteri. També ens agraden molt les aventures d’en Gerónimo Stilton.

Eloi Banyes, Ferran Corral i Roger Guarro, 3r de Primària
}


\shortnewsitem{La Tanga i el gran lleopard}
{
Autor: Roberto Malo i Fco Javier Mateos.  Editorial: Comanegra

La Tanga és una noia que viu en un poblat al cor de la selva. En el poblat hi ha un mag que es diu el Gran Bruixot i que té poders màgics. Un bon dia, un lleopard ferotge i cruel va menjant-se tots els animals del poblat. A partir d’aquell succés van començar a patir fam, i el Gran Bruixot va intentar aturar e lleopard, però no va funcionar. A la nit es va posar una màscara per transmetre un missatge als ciutadans. I va dir a tot el poblat que el dia següent tothom que volgués enfrontar-se al lleopard es presentés, però no va aparèixer ningú, excepte la Tanga. Llavors el Gran Bruixot li va donar un ganivet de fusta de banús... 

Ens ha agradat molt perquè aquest llibre és una aventura molt divertida i perquè ens agrada molt la màgia i el Gran Bruixot en fa molta.

 Pablo de Quadras i Eduard Tenas,  3r de Primària
}


\shortnewsitem{El zoo d’en Pitus}
{
Autor: 	Sebastià Sorribas.  Editorial: La Galera

\noindent\includegraphics[width=4cm,keepaspectratio]{primaria/img/llibres_pitus.jpg}

El Pitus té una malaltia. Només el pot curar un metge molt famós de Suècia. Els seus amics, el Tanet, el Manelitus, en Cigró, en Juli, en Fleming i la Mariona van tenir una gran pensada: com que a en Pitus li agradaven molt els animals, van decidir fer un zoo per a ell.

M’ha agradat molt l’argument d’aquest  llibre.

Roser Pérez, 4t de Primària
}

\shortnewsitem{Tina superbruixa. L’aniversari d’en Pitus}
{
Autor: Knister   Editorial: Bruixola

\noindent\includegraphics[width=4cm,keepaspectratio]{primaria/img/llibres_fada.jpg}

Aquesta és una de les emocionants històries de la Tina. És l’aniversari d’en Pitus i està molt emocionat i vol fer una súper festa, però el seu pare està de viatge i la seva mare en un curs. Ve la seva tieta, que és una mica bleda  amb els nens. A la Tina i al Pitus els tracta com a nadons. Quan arriben els convidats, la tieta Elisa es queixa i es queixa fins que la Tina fa un dels seus encanteris i... Plaf! La tieta s’adorm.

M’ha agradat perquè és molt divertit, és una mica curt, però val la pena. És entretingut i el recomano a tothom.

Ariadna Vera, 4t de Primària
}

\shortnewsitem{Els rebels de la cabana}
{
Autor: David Nel·lo.   Editorial: Cruïlla

Va d’un grup de nens que tenen feta una cabana a dalt d’un arbre i l’alcalde vol construir un aparcament al Prat, tallaran l’arbre on hi ha la cabana. La Margot, que és la que mana, vol negociar amb ell perquè no tallin l’arbre, però l’alcalde diu que no. Aleshores, els nens van dir que si no negociaven, no baixarien de la cabana. 

És molt divertit perquè els nens són molt atrevits ja que es queden una nit de pluja a la cabana. Jo el recomano perquè és molt emocionant.

Aina Boronat, 5è de Primària
}

\shortnewsitem{La tribu de Camelot}
{
Autora: Gemma Lienes.    Editoria: Empúries

El Papagueno, el canari de la veïna de la Carlota ha desaparegut i la tribu passa moltes aventures amb un llibre de màgia que els ajuda a retornar en Papagueno a la veïna de la Carlota.  

Aquest  llibre m’ha agradat molt perquè hi ha moltes aventures i és molt intrigant per saber qui ha raptat Papagueno. M’entretenia a trobar els dracs que hi havia amagats a moltes pàgines, també eren molt divertides les olors i les tintes.  
Paula Claver, 5è de Primària
}

\shortnewsitem{Pocahontas}
{
Autor: Walt Disney.   Editorial: Beascoa

La història comença quan va arribar una expedició de navegants a la terra dels indis.
L’objectiu era trobar or i enriquir-se. Casualment, en John coneix la Pocahontas i se n’enamora. Al pare de la Pocahontas  això no li agrada ja que volia que es casés amb en Kocum. En una baralla per la Pocahontas, mor en Kocum i fan presoner en John. Els amics d’en John corren a salvar-lo amb les escopetes...

És una història d’amor i d’aventures i a mi m’ha agradat molt!

Eva Llovera, 5è de Primària.
}


\shortnewsitem{La meravellosa medicina d’en Jordi}
{
Autor: Roald Dahl.   Editorial: Empúries

En Jordi és un noi que  sempre ha d’estar cuidant la seva àvia i preparant-li els medicaments. Un dia, per experimentar,  li va preparar  una meravellosa medicina, quan l’àvia se la va prendre, es va fer molt alta i sobresortia pel terrat de la casa. Al seu pare li va agradar aquest experiment per donar-lo als animals, així serien més grans i tindrien més carn i els aprofitaria més. Va donar el medicament als animals i tots es van tornar  més grans. Al cap d’un temps, en va fer un altre i el va donar a l’àvia, llavors l’àvia es va tornar molt ...  

Aquest llibre és, al principi, una mica avorrit, fins que  comencen els experiments, però a mi el que més m’ha agradat són les coses que li passen a l’àvia.

Judit Molina, 6è de Primària
}

\shortnewsitem{Ales de foc}
{
Autor:Christopher Pike. Editorial: Edicions  agrupo  Z .

Ariel és un àngel que té una “protegida”que es diu Marla. Ariel acaba traïda per Marla, i és tancada a Gorilan, una presó “màgica”. Allà hi ha un rei que és un gripau molt intel·ligent. A Gorlian coneix un noi anomenat Brad, que acaba mort per l’exèrcit del rei gripau. Aleshores acaba sent rescatada per uns amics i viuen diferents aventures.

A mi aquest llibre m’ha agradat molt perquè és d’aventures i d’éssers màgics.

Dídac Benages , 6è de Primària
}

\shortnewsitem{Malsons}
{
Autor: Anne Fine. Editorial: Bromera.

Una nena anomenada Imogen canvia d’ escola, es fa  molt amiga de la Melanie. A les altres escoles que ha estat no tenia gaires amigues. La Melanie descobreix que Imogen té un secret, ple de màgia, de por, de fetilleria i ple d’encanteris. Aquest secret que li succeeix a Imogen serà bastant difícil de descobrir, només la Melanie ho podrà fer!

Aquest llibre, a mi m’ha agradat molt ja que és de misteri, d’aventures i a vegades semblen fets reals!!

Mariona Medrano, 6è de Primària
}

\end{shortnews}
\definecolor{color}{rgb}{0.1 , 0.1 , 0.1}

\begin{news}
{2} %columnes
{“Dins i fora”. Viure l’espai amb Beuys i Isozaki}
%index: L’espai amb Beuys i Isozaki
{}
{Primaria}
{22} %pagesof

%# 1.jpg  2.jpg  img003.jpg  img004.jpg
\noindent\includegraphics[width=8cm,keepaspectratio]{primaria/img/1.jpg}

\noindent\includegraphics[width=8cm,keepaspectratio]{primaria/img/img004.jpg}



El passat 5 de maig, els nens i nenes de primer de primària vàrem anar al Caixa Fòrum a veure dues obres, que es diuen instal.lacions, anomenades “Espai de dolor” i “El jardí secret”.

“Espai de dolor” és una instal.lació d’en Joseph Beuys. És un espai rectangular que té les parets i el sostre completament folrats de plom. Només es veu una bombeta pelada i dues anelles de plata que pengen al seu costat. En realitat és un espai hermètic en tots els sentits. Un cop dins la cambra, un es queda completament aïllat del món.
“El jardí secret” és una altra instal.lació; d’Arata Isozaki. És també un espai rectangular gairebé tancat, amb una obertura per entrar-hi, i sense sostre, que s’inscriu en un altre rectangle més gran que conforma tot el pati. No s’hi pot entrar però, si poguéssim, no tindríem la sensació d’estar tancats.
L’activitat dirigida que vam realitzar ens va permetre viure i tenir sensacions amb aquests dos espais. Heus aquí  l’opinió:


Vam anar al Caixa Fòrum i vam veure L’espai de dolor. Vam tocar les parets i estaven fredes i després vam anar al “Jardí secret” i vam fer soroll i ens arrossegàvem per les parets. M’ho vaig passar molt bé.

Jordi Montero


\noindent\includegraphics[width=8cm,keepaspectratio]{primaria/img/2.jpg}

\noindent\includegraphics[width=8cm,keepaspectratio]{primaria/img/img003.jpg}

El dia cinc de maig vam anar al Caixa Fòrum, vam veure dues instal.lacions; una es deia El jardí secret i l’altra Espai de dolor. També vam fer dos grups, un era el grup fosc i l’altra era el clar i, a mi, em va tocar el fosc, perquè tocava el que tocava i a dins de L’espai de dolor sentíem una veu i m’ho vaig passar molt bé.


Quima Lleonart

Vam anar a veure dues instal.lacions i una es deia L’espai de dolor i vam tocar les parets i les parets estaven fredes i eren de metall. Vam veure també el Jardí secret, i hi havia aigua i estava fet de pedres i ressonava tot quan fèiem un crit. Si hi havia silenci només sentíem el soroll de l’aigua.

Héctor Cuadra

Hem anat al Caixa Fòroum i vam mirar L’espai de dolor i el Jardí secret. A L’espai de dolor hi vam poder entrar i vam veure que al sostre hi havia dues anelles de plata, una més gran que l’altra. Representaven dos caps pensant. El cap d’una persona gran i el d’un nen. Dins l’espai ens vam arrossegar pel terra i vam tocar la paret i era fresca i tot estava molt fosc. Al Jardí secret podíem escoltar el silenci.

Eric Ramos




\end{news}
\definecolor{color}{rgb}{0.1 , 0.1 , 0.1}

\begin{news}
{2} %columnes
{Conferència de la Josefina Piquet a l'Escola}
{\noindent\includegraphics[width=18cm,keepaspectratio]{primaria/img/dones_36_p4230014.jpg}}
{Primaria}
{019} %pagesof




La passada diada de Sant Jordi, la senyora Josefina Piquet, coordinadora de l'associació “Dones del 36” va visitar l'escola per compartir les seves vivències amb els nois i noies de 4t.

La sessió fou molt emotiva i profitosa, tant des del punt de vista de l'educació en valors com de l'educació en ciències socials.

La nostra escola sempre ha tingut com a una de les seves prioritats transmetre  la memòria històrica del nostre país. En aquest sentit, la valoració de les repercussions de la nostra guerra fa possible analitzar la seva influència, tant en les generacions que la van viure directament, com en la societat actual. D'altra banda, treballar sobre el fenomen de la guerra contribueix a avançar en el debat sobre la resolució pacífica dels conflictes.

Amb xerrades com la de la Josefina Piquet (per cert, quan ve a l'Escola Solc diu que “es troba com a casa”) intentem recuperar la història recent de les dones mitjançant la seva pròpia veu, la seva memòria individual i col·lectiva; destacar la importància i la complexitat de la història transmesa oralment, de la seva metodologia i dels seus resultats i analitzar la diversitat de protagonistes i d'escenaris en la història de la vida quotidiana, especialment en moments excepcionals com és en temps de guerra, exili i postguerra. 

\end{news}

\newssep
\definecolor{color}{rgb}{0.1 , 0.1 , 0.1}

\begin{news}
{2} %columnes
{Fem un herbari!}
{}
{Primaria}
{029} %pagesof

%\noindent\includegraphics[width=9cm,keepaspectratio]{primaria/img/os.png}

Ja fa uns dies que vam anar al Jardí Botànic. Vam fotografiar els arbres que hi havia per poder fer després un PowerPoint a l’escola. Vam veure molts arbres i tots eren molt bonics. 
La Maria ens va dir que anéssim buscant fulles, durant uns quants dies, ja que faríem un herbari. També ens va dir que portéssim les fotos de la càmera per fer el PowerPoint. 

Per a l’herbari, la Maria ens va donar una bossa per posar-hi les fulles que primer havíem assecat posant-les entre fulls de diari, ben premsades. Llavors, en una cartolina on hi havia la descripció de l’arbre, hi enganxàvem les fulles i nosaltres havíem de fer un títol bonic amb el nom de l’arbre.

Jo he après moltes coses dels arbres. Ja sé com se’n diuen uns quants que no coneixia i com són les fulles i el tronc. M’ha agradat molt fer aquest herbari i, a més a més, m’ho he passat molt bé i ha estat molt divertit.

\authorandplace{}{4t de Primària}

\end{news}
\definecolor{color}{rgb}{0.1 , 0.1 , 0.1}

\begin{news}
{2} %columnes
{Sortida a Santa Fe del Montseny}
{\noindent\includegraphics[width=18cm,keepaspectratio]{primaria/img/montseny_DSC00744.JPG}}
{Primaria}
{19} %pagesof

{\noindent\includegraphics[width=9cm,keepaspectratio]{primaria/img/montseny_DSC00774.JPG}}

El dimecres 19 de maig,  la classe de 5è de Primària va anar al Montseny.

Quan vàrem arribar, vam fer un passeig per la muntanya mentre observàvem el paisatge. Però especialment vam observar un bosc de castanyers i un altre de faigs. A l’interior d’aquests boscos agafàvem un full, el col·locàvem sobre l’escorça d’un arbre i amb el llapis guixàvem i quedava calcada l’escorça. Això, també ho fèiem amb una fulla de l’arbre. Un cop fet, el dibuixàvem. Vàrem fer mitja volta al Pantà Santa Fe, és molt bonic i gran. Hi ha una gran resclosa per a retenir l’aigua. A la tornada ja podíem treure les càmeres per  fer fotografies. En acabar aquest itinerari, ens vam situar a ”Can Casades” , que és un lloc on hi ha una mena de masia. Allà, vàrem veure un audiovisual on sortia com canviava el Montseny en les diferents estacions de l’any. Un cop vist, vam dinar, vam jugar, i vam tornar al col·legi.  

\authorandplace{Marc Luna}
{5è de Primària}

\end{news}
