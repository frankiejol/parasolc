\begin{news}
{2} %columnes
{La Bombeta}
{}
{Primaria}
{029} %pagesof

\noindent\includegraphics[width=9cm,keepaspectratio]{primaria/img/bombeta.png}

L’invent de la bombeta s’atribueix a Thomas Alva Edison que va presentar la patent  el 21 d’octubre de 1878. Està formada per un filament de wolframi molt primet situat a l’interior d’una “ampolla” de vidre que s’ha omplert posteriorment d’un gas inert o bé se li ha fet el buit. A la part de sota de la bombeta hi ha una part metàl·lica on s’hi troben totes les connexions elèctriques. Aquesta part metàl·lica, normalment té una una rosca que permet caragolar la bombeta al portalàmpades. La bombeta elèctrica també s’anomena làmpada incandescent.  De tota l’energia que consumeix la bombeta, només el 10\% es converteix en llum; la resta, es transforma en escalfor, llum ultraviolada i llum infraroja. 
La bombeta és un dels objectes més utilitzats des de la seva invenció, ja que tots en tenim, com a mínim, una a casa.

\authorandplace{Conrad Galli}{6è de Primària}

\end{news}

\newssep
