\begin{news}
{2} %columnes
{Sortida a Santa Fe del Montseny}
{\noindent\includegraphics[width=18cm,keepaspectratio]{primaria/img/montseny_DSC00744.JPG}}
{Primaria}
{19} %pagesof

{\noindent\includegraphics[width=9cm,keepaspectratio]{primaria/img/montseny_DSC00774.JPG}}

El dimecres 19 de maig,  la classe de 5è de Primària va anar al Montseny.

Quan vàrem arribar, vam fer un passeig per la muntanya mentre observàvem el paisatge. Però especialment vam observar un bosc de castanyers i un altre de faigs. A l’interior d’aquests boscos agafàvem un full, el col·locàvem sobre l’escorça d’un arbre i amb el llapis guixàvem i quedava calcada l’escorça. Això, també ho fèiem amb una fulla de l’arbre. Un cop fet, el dibuixàvem. Vàrem fer mitja volta al Pantà Santa Fe, és molt bonic i gran. Hi ha una gran resclosa per a retenir l’aigua. A la tornada ja podíem treure les càmeres per  fer fotografies. En acabar aquest itinerari, ens vam situar a ”Can Casades” , que és un lloc on hi ha una mena de masia. Allà, vàrem veure un audiovisual on sortia com canviava el Montseny en les diferents estacions de l’any. Un cop vist, vam dinar, vam jugar, i vam tornar al col·legi.  

\authorandplace{Marc Luna}
{5è de Primària}

\end{news}
