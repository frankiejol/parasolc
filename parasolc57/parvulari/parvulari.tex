\definecolor{color}{cmyk}{0.5, 0, 1, 0.5}

\begin{news}
{2} %columnes
{El plaer de jugar i de compartir}
{El pati és un recurs fantàstic  per poder jugar, ja que considerem que el joc és una activitat fonamental pel desenvolupament dels nens i nenes.
}
{parvulari}
{3} %pagesof


El fet de sortir a l’aire lliure ens permet de jugar a un munt de coses, fer sorra fina i “ bollitos” (mena de boletes fetes de sorra i arrebossades amb sorra fina) jugar amb pales i galledes, a pilota .... I encara més,  a inventar altres jocs que ells mateixos van  creant.


\noindent\includegraphics[width=9cm,keepaspectratio]{parvulari/img/foto_pales.jpg}


\noindent\includegraphics[width=9cm,keepaspectratio]{parvulari/img/foto_pilota.jpg}

Per als cargols, el pati és una manera més de conèixer-se i de passar-s’ho bé a l’escola amb els seus amics i amigues de la classe.
Considerem a més a més que el pati és un lloc privilegiat d’observació, exploració i aprenentatge ; és un dels llocs on els cargolets aprenen a conèixer l’entorn i a adquirir hàbits de cura  i respecte amb el medi i amb els companys, no només els de la seva classe,  sinó amb els d’altres grups.


\noindent\includegraphics[width=9cm,keepaspectratio]{parvulari/img/foto_floretes.jpg}

Després de tants i tants dies de pluja i mal temps, el pati ens serveix per omplir-nos a tots nosaltres d’energia. 

\noindent\includegraphics[width=9cm,keepaspectratio]{parvulari/img/foto_tobogan.jpg}

\noindent\includegraphics[width=9cm,keepaspectratio]{parvulari/img/foto_sorra.jpg}


Així juguen i comparteixen els cargols les estones lúdiques del pati.


\authorandplace{Les mestres de parvulari}{Escola Solc}

\end{news}
\definecolor{color}{cmyk}{0.5, 0, 1, 0.5}

\begin{news}
{2} %columnes
{Pensar, fer i experimentar}
{Els pingüins i els elefants descobrim i experimentem amb diferents materials}
{Parvulari}
{011} %pagesof






\noindent\includegraphics[width=9cm,keepaspectratio]{parvulari/img/foto_materials.jpg}

L’experimentació en sentit científic s’entén com el mètode d’investigació que es fonamenta en crear una hipòtesi.

\noindent\includegraphics[width=9cm,keepaspectratio]{parvulari/img/foto_investigar1.jpg}


% Trec aquesta si no no m'hi cap l'article en una plana i surten dos nens tallats
%\noindent\includegraphics[width=9cm,keepaspectratio]{parvulari/img/foto_investigar2.jpg}

Aquests tipus d’activitats, impliquen manipulació d’elements i solen ser molt engrescadores per als nens.

\noindent\includegraphics[width=9cm,keepaspectratio]{parvulari/img/foto_manipulacio_elements1.jpg}

\noindent\includegraphics[width=9cm,keepaspectratio]{parvulari/img/foto_manipulacio_elements2.jpg}

A més a més de tocar i manipular, també ajuden a desenvolupar una  activitat mental. Amb les experimentacions els nens i nenes aprenen a pensar, a fer, a repensar i a refer ; per tant això ens aporta molts elements positius.
                         
\noindent\includegraphics[width=9cm,keepaspectratio]{parvulari/img/foto_fer_refer.jpg}

Aprenen a dialogar i a  raonar amb calma sobre el fet que experimentem  fins a arribar a aconseguir  trobar el sentit estètic de les coses. L’entenem com el focus d’iniciació a la recerca.

Amb aquestes experiències aconseguim  potenciar el treball en equip, i tot plegat ens ajuda a millorar els resultats educatius i el rendiment escolar.

Les mestres de parvulari intentem que els petits de l’escola vagin  descobrint tot allò que ens envolta per tal de desenvolupar tots els sentits d’una forma lúdica,  fent servir la seva imaginació.
                         
\authorandplace{Les mestres de parvulari}{Escola Solc}

\end{news}
