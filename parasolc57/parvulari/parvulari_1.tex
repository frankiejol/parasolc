\begin{news}
{2} %columnes
{El plaer de jugar i de compartir}
{El pati és un recurs fantàstic  per poder jugar, ja que considerem que el joc és una activitat fonamental pel desenvolupament dels nens i nenes.
}
{parvulari}
{3} %pagesof


El fet de sortir a l’aire lliure ens permet de jugar a un munt de coses, fer sorra fina i “ bollitos” (mena de boletes fetes de sorra i arrebossades amb sorra fina) jugar amb pales i galledes, a pilota .... I encara més,  a inventar altres jocs que ells mateixos van  creant.


\noindent\includegraphics[width=9cm,keepaspectratio]{parvulari/img/foto_pales.jpg}


\noindent\includegraphics[width=9cm,keepaspectratio]{parvulari/img/foto_pilota.jpg}

Per als cargols, el pati és una manera més de conèixer-se i de passar-s’ho bé a l’escola amb els seus amics i amigues de la classe.
Considerem a més a més que el pati és un lloc privilegiat d’observació, exploració i aprenentatge ; és un dels llocs on els cargolets aprenen a conèixer l’entorn i a adquirir hàbits de cura  i respecte amb el medi i amb els companys, no només els de la seva classe,  sinó amb els d’altres grups.


\noindent\includegraphics[width=9cm,keepaspectratio]{parvulari/img/foto_floretes.jpg}

Després de tants i tants dies de pluja i mal temps, el pati ens serveix per omplir-nos a tots nosaltres d’energia. 

\noindent\includegraphics[width=9cm,keepaspectratio]{parvulari/img/foto_tobogan.jpg}

\noindent\includegraphics[width=9cm,keepaspectratio]{parvulari/img/foto_sorra.jpg}


Així juguen i comparteixen els cargols les estones lúdiques del pati.


\authorandplace{Les mestres de parvulari}{Escola Solc}

\end{news}
