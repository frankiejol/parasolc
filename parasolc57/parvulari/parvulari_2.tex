\begin{news}
{2} %columnes
{Pensar, fer i experimentar}
{Els pingüins i els elefants descobrim i experimentem amb diferents materials}
{Parvulari}
{011} %pagesof






\noindent\includegraphics[width=9cm,keepaspectratio]{parvulari/img/foto_materials.jpg}

L’experimentació en sentit científic s’entén com el mètode d’investigació que es fonamenta en crear una hipòtesi.

\noindent\includegraphics[width=9cm,keepaspectratio]{parvulari/img/foto_investigar1.jpg}


% Trec aquesta si no no m'hi cap l'article en una plana i surten dos nens tallats
%\noindent\includegraphics[width=9cm,keepaspectratio]{parvulari/img/foto_investigar2.jpg}

Aquests tipus d’activitats, impliquen manipulació d’elements i solen ser molt engrescadores per als nens.

\noindent\includegraphics[width=9cm,keepaspectratio]{parvulari/img/foto_manipulacio_elements1.jpg}

\noindent\includegraphics[width=9cm,keepaspectratio]{parvulari/img/foto_manipulacio_elements2.jpg}

A més a més de tocar i manipular, també ajuden a desenvolupar una  activitat mental. Amb les experimentacions els nens i nenes aprenen a pensar, a fer, a repensar i a refer ; per tant això ens aporta molts elements positius.
                         
\noindent\includegraphics[width=9cm,keepaspectratio]{parvulari/img/foto_fer_refer.jpg}

Aprenen a dialogar i a  raonar amb calma sobre el fet que experimentem  fins a arribar a aconseguir  trobar el sentit estètic de les coses. L’entenem com el focus d’iniciació a la recerca.

Amb aquestes experiències aconseguim  potenciar el treball en equip, i tot plegat ens ajuda a millorar els resultats educatius i el rendiment escolar.

Les mestres de parvulari intentem que els petits de l’escola vagin  descobrint tot allò que ens envolta per tal de desenvolupar tots els sentits d’una forma lúdica,  fent servir la seva imaginació.
                         
\authorandplace{Les mestres de parvulari}{Escola Solc}

\end{news}
