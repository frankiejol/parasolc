% Copyright (C) 2007 by Ignacio Llopis <lloptor@gmail.com>
% -------------------------------------------------------
% 
% This file contains an example of what paperTeX class can do.
% Compile it using PDFLaTeX in order to get the best results.
% This file requires the img/ folder.
% There are many comments that the user can uncomment in order to see 
% how easy is to change the output style.
%
%
% This file may be distributed and/or modified under the
% conditions of the LaTeX Project Public License, either version 1.2
% of this license or (at your option) any later version.
% The latest version of this license is in:
%
%    http://www.latex-project.org/lppl.txt
%
% and version 1.2 or later is part of all distributions of LaTeX 
% version 1999/12/01 or later.
\documentclass[10pt,final,hyphenatedtitles]{papertex}
\pdfminorversion=5
\pdfobjcompresslevel=2
\pdfcompresslevel=5

\usepackage[utf8]{inputenc}
\usepackage[english,catalan]{babel}
%\usepackage[T1]{fontenc}
\usepackage{ulem}
%\usepackage{color}
%\usepackage{times}
\usepackage{fancybox}
\usepackage{wallpaper}
%\usepackage{draftwatermark}
%\SetWatermarkFontSize{5cm}

%\definecolor{color}{cmyk}{1, 0,0, 0.5}

%\renewcommand{\indexEntryFormat}{\large\rmfamily}
%\renewcommand{\indexEntryPageTxt}{page}
%\renewcommand{\timestampSeparator}{$\Rightarrow$}
%\renewcommand{\innerTextFinalMark}{$\spadesuit$}


\renewcommand{\logo}{\vspace*{-1.5cm} \mylogo{\noindent{\fontsize{35mm}{14mm} \usefont{T1}{pag}{bx}{n} \textcolor{black}{paraSOLC}}\\[3pt]}}
%\renewcommand{\editionFormat}{\LARGE}
%\renewcommand{\indexEntryFormat}{\normalsize\rmfamily}

\renewcommand{\date}{Juny 2010}
\renewcommand{\issue}{57}
\author{AMPA Solc}
\edition{Revista de l'Escola Solc}

%\setlength{\columnsep}{2cm}


%new "ragged text" feature
\minraggedcols=3

\begin{document}

\selectlanguage{catalan}

\begin{frontpage}

\ThisCenterWallPaper{1}{portada/img/portada.png}

\begin{indexblock}{}
\indexitem{Banc d'Aliments}{101}
\indexitem{El documental com a estratègia educativa}{102}
\indexitem{Què és el primer que treballem quan arribem a l’escola ?}{21}
\indexitem{Els castanyers i les castanyeres}{216}
\indexitem{Notícies molt i molt importants a 1r de Primària}{51}
\indexitem{Cinema}{33}
\indexitem{El sabor dolç}{34}
\indexitem{Santa Fe del Montseny}{304}
\indexitem{La digestió}{305}
\indexitem{Quan tingui 30 anys}{306}
\indexitem{La palmera del pati}{37}
\indexitem{Ens visita un astrònom}{35}
\indexitem{A famous person}{52}
\indexitem{Aprendre Democràcia}{18}
\indexitem{L'aulamòbil de les professions}{11}
\indexitem{Sortida al CaixaForum}{12}
\indexitem{Sortida Pipilotti Rist}{13}
\indexitem{Xerrada sobre el Banc dels Aliments}{404}
\indexitem{Projecte Apadrinament Lector}{16}
\end{indexblock}

%titol de les seccions en blanc {}
\begin{weatherblock}{}
  \weatheritem{fem_escola/img/index.png}{fem escola}{ pàg. \indexEntryPageTxt{}~101}
\weatheritem{parvulari/img/index.png}{parvulari}{ pàg. 21}
\weatheritem{primaria/img/index.png}{primaria}{ pàg. 51}
\weatheritem{eso/img/index.png}{eso}{ pàg. 18}
\end{weatherblock}



%\begin{weatherblock}{}
\weatheritem{fem_escola/img/index.png}{fem escola}{ pàg. 2}
\weatheritem{parvulari/img/index.png}{parvulari}{ pàg. 3}
\weatheritem{primaria/img/index.png}{primaria}{ pàg. 5}
\weatheritem{eso/img/index.png}{eso}{ pàg. 18}
\end{weatherblock}




\begin{authorblock}
\textbf{}

AMPA Solc\\
\href{mailto:parasolc@gmail.com}{\texttt{parasolc@gmail.com}}\\[5pt]
\href{http://escolasolc.com}{www.escolasolc.com}\\

\end{authorblock}

\end{frontpage}


\newsection{Fem Escola}

%doc: Revista 3/La castanyada/ELS CASTANYERS I LES CASTANYERES.docx
\newpage

\begin{news}
{2} %columnes
{Els castanyers i les castanyeres}
{Els nois i noies de 1r d’ESO,  van fer de castanyers i castanyeres als nens i nenes de parvulari i primària}
{ESO}
{216} %pagesof


\noindent\includegraphics[width=8.5cm,keepaspectratio]{parvulari/img/castanyada_fiotos012.jpg}
El divendres  29 d’octubre,  sis nois i noies de 1r d’ESO,  l’Oriol, la Laia, l’Òscar, el Joel, el Dídac i l’Ariadna,  van anar a fer una representació als nens i nenes de parvulari i primària. Van fer de castanyers i castanyeres. Ho van fer molt bé, tenien moltes ganes de poder-la representar, els feia molta il.lusió.

Els nens i nenes petits es van quedar bocabadats, els va agradar molt, feien una cara de sorpresos, que em sembla que tots s’ho van creure!

El vestuari que portaven estava molt ben trobat; em penso que per a ells no va ser fàcil fer la representació, no només s’havien de disfressar, sinó que també van haver de fer una veu especial, moure els braços i les cames com els vells, i, sobretot, no tenir gens de vergonya! 

A tots els va agradar molt la història que van explicar de la castanya daurada, però també els va agradar moltíssim la cançó que van ballar al final.

Van ser unes castanyeres i castanyers molt valents, simpàtics i gens vergonyosos!!!

\noindent\includegraphics[width=8.5cm,keepaspectratio]{parvulari/img/castanyada_DSC00666.JPG}

\noindent\includegraphics[width=8.5cm,keepaspectratio]{parvulari/img/castanyada_DSC01046.JPG}

\authorandplace{Mariona Medrano}{1r d’ESO}
						
\end{news}


%doc: Revista 3/Res Llibres/ressenyes llibres.docx
\begin{news}
{2} %columnes
{Ressenyes de Llibres}
{}
{Fem Escola}
{03} %pagesof



\subsection*{La Castanyera}

\noindent\fbox{\includegraphics[width=7.5cm,keepaspectratio]{fem_escola/img/ressenya_img002.jpg}}

\emph{Autor: Anna Grau.  Editorial: Combel}

Una castanyera era molt amable, tant, que als nens que no portaven diners els donava castanyes i els deia que portessin els diners un altre dia. Però una altra castanyera, que era dolenta, com que no li compraven castanyes, va pensar robar les castanyes a l’altra. Llavors els nens li van donar castanyes a la castanyera amable, perquè en pogués continuar venent. Al final les dues es van fer amigues i van vendre juntes.

Aquest conte m’ha agradat perquè tracta de l’ amistat.

\authorandplace{Arnau Garcés}{1r de Primària}

\newpage

\subsection*{Els dos núvols amics}
\emph{Autor: Enric Larreula Editorial: Teide }

\noindent\fbox{\includegraphics[width=5cm,keepaspectratio]{fem_escola/img/ressenya_img004.jpg}}

Els núvols són molt amics, viatjaven pel temps perquè els agradava estar molt junts. Un núvol estava amb els ocells i els altres estaven solets, els tiraven globus i un dia van caure a terra. Els agradava estar a les muntanyes, a vegades canviaven els colors: un era rosa i l’altre era lila. Un dia va fer molt vent i es van separar una mica, es va posar molt negre el cel i va ploure. Més tard va sortir el sol i els núvols van quedar enterrats.

He triat aquest llibre per l’amistat que demostren entre ells, pels jocs, pels viatges,...

\authorandplace{Hàlia Iborra}{1r de Primària}

\subsection*{La història de l’Ernest}
\emph{Autor: Mercè Company   Editorial: Cruïlla}


\noindent\fbox{\includegraphics[width=5cm,keepaspectratio]{fem_escola/img/ressenya_img007.jpg}}

Un nen que es diu Xavier tenia un gat marró molt “xulo”, però un dia es va posar histèric.

M’ha agradat aquest conte perquè el dia de l’aniversari de l’Ernest, en Xavier li regala un gat. 

\authorandplace{Sara Colomer}{2n de Primària}


\subsection*{Feu-me cas!}
\emph{Autor: Anke de Vries     Editorial: Cruïlla}

En Kees no estimava el seu germà, en Ron, i la seva mare li va portar un regal que era una pilota i en Kees va estimar el seu germà.

M’ha agradat perquè han fet els dibuixos molt bé.

\authorandplace{Èric Ramos}{ 2n de Primària}

\subsection*{El vell mariner}
\emph{Autor: Geronimo Stilton  Editorial: Destino}

\noindent\fbox{\includegraphics[width=5cm,keepaspectratio]{fem_escola/img/ressenya_img008.jpg}}

Aquest llibre tracta d’un mariner que un dia va a pescar i pesca una sirena.  Se l’emporta a casa seva i li recorda una senyora que anava amb cadira de rodes....

M’ha agradat molt perquè m’agraden els vaixells

\authorandplace{Elena Canosa i Damià Rubio}{3r de Primària}


\subsection*{Llegeix-me si us plau!}
\emph{Autor: Frank Sales.  Editorial: Cruïlla}

\noindent\fbox{\includegraphics[width=5cm,keepaspectratio]{fem_escola/img/ressenya_img006.jpg}}

Aquesta història parla d’un llibre en un prestatge d’una biblioteca. Aquest llibre té màgia: les lletres parlen! Demanen a la gent que llegeixin perquè no se’ls emportin de la biblioteca. Un dia el llibre cau a sobre de la Laia i comença llegir el conte del lloro Pepet.

M’ha agradat perquè el lloro Pepet comptava d’una manera molt estranya.

\authorandplace{Albert Garcia i Roger Ibars}{ 3r de Primària}


\subsection*{Las aventuras de Ulises}
\emph{Autor: Geronimo Stilton    Editorial: Destino.}

Geronimo Stilton va a una illa, allà troba uns ratolins molt rics i els agrada fer aventures. Entre aquests hi ha el gran capità que no para de parlar d’ Ulisses, que si domina el mar, que si és molt poderós,... Aleshores, Geronimo Stilton va junt amb el capità en un vaixell. De sobte, l’Ulisses neda fins on és el vaixell i amb el seu poder els el trenca. Aquells van en busca de l’Ulisses,... 
	
Aquest llibre m’ha agradat molt per les aventures.

\authorandplace{Axel Santos}{4t de Primària}

\subsection*{Els casos de l’inspector Formiga}
\emph{Autor: Joan de Déu Prats    Editorial: Marge}

\noindent\fbox{\includegraphics[width=5cm,keepaspectratio]{fem_escola/img/ressenya_img003.jpg}}

És una formiga inspectora que resol casos dels insectes que són assassinats i dels desapareguts. El cas és d’un escarabat que se’l troba mort al carrer i comença a investigar; troba petjades, un ganivet,...

A mi aquest llibre m’ha agradat perquè va d’espionatge i de misteri. El recomano als alumnes de Primària.

\authorandplace{Júlia Carmona}{ 4t de Primària}

\subsection*{Arcanus}
\emph{Autor: Care Santos   Editorial: Planeta}

La història d’aquest llibre ens parla de nens que tenen uns dons especials i que han de cooperar entre ells per salvar el món del malvat Maghul. Aquest llibre forma part d’una col·lecció de dotze títols. A cada un es presenta un dels nens i , a més a més, hi surten bèsties com per exemple un Fènix o un Hipògrif. 

A mi m’ha agradat molt. Us el recomano.

\authorandplace{Oriol Fossas}{ 5è de primària}

\subsection*{Escombres voladores}
\emph{Autor: Ann Jungman    Editorial: Vicens Vives}

Aquest llibre tracta de tres bruixes que a les nits de lluna plena fan encanteris dolents. Dues de les bruixes decideixen no tornar a fer mai més de bruixa. La bruixa que queda s’enfada amb les altres i les insulta. Al cap d’un temps una de les bruixes demana ajuda a les altres. L’ajuden encara que les hagués tractat de mala manera. Finalment la bruixa que no s’havia portat bé es casa gràcies a les seves companyes.

\authorandplace{Carla Salas}{5è de Primària}

\subsection*{T’escriuré}
\emph{Autor: Glòria Llobet.   Editorial: Baula}

L’Anna vivia a Gandia (València). Allà va tenir moltes amistats, especialment  un noi,en Guillem. Un dia els seus pares li comuniquen que se n’han d’anar a viure a Manlleu (Barcelona). Allà haurà de fer noves amistats i superar la seva tristesa.

Aquest llibre m’agrada perquè aquesta noia ha d’anar passant “obstacles” en la seva vida i no saps com reaccionarà. També perquè es una història d’amistat i de comèdia.  

\authorandplace{Marc Luna}{6è de Primària}

\subsection*{Mutació fatal}
\emph{Autor: R.L.Stine  Editorial: Edicions primera plana S.A.}


\noindent\fbox{\includegraphics[width=5cm,keepaspectratio]{fem_escola/img/ressenya_img005.jpg}}

Gary Lutz és un nen que no suporta la seva vida, té por de les abelles i se seny empipat per tothom, fins i tot la seva germana es posa amb ell. El senyor Andretti, el seu veí, té un rusc d’abelles. Al nen li fan por les abelles i el senyor Andretti li fa bromes amb les abelles perquè les controla. Un dia ,el nen rep un paper d’una empresa que tenia una màquina que podia canviar les vides en un temps determinat. El nen no s’ho pensa dues vegades i va a l’edifici de l’empresa. A l’edifici, degut a una infiltració d’una abella, Gary Lutz es convertirà en l’insecte que menys li agrada.


\authorandplace{Àlex Fuentes}{6è de Primària}



\subsection*{Lior}
\emph{Autor; Pradas, Núria. Ed.}
 
Us recomano aquest llibre perquè és molt entretingut. Parla d’un món del futur on no hi ha sentiments ni llibertat, només val l’esport i la competició. En Lior descobrirà que hi ha altres habitants d’aquesta societat que pensen de forma diferent i lluitarà per aconseguir un món diferent.

\authorandplace{Joan Colldecarrera}{1r ESO}


\subsection*{Cartas de invierno}

\emph{Autor; 	Fernández Paz, Agustín. Ed.SM}

El pintor Adrián Novoa, aconsejado por su amigo Xavier, compra una casa en una aldea de Galicia. Es la casa de sus sueños. En ella quiere encontrar la paz necessaria para encontrar nueva inspiración en su pintura. Pero la casa esconde un profundo y oscuro misterio que atrapa a Novoa como un imán.

Recomiendo este libro porque te engancha desde el principio, es decir, no puedes parar de leer, porque cada vez que acabas un capítulo quieres leer el siguiente. És un libro de misterio y algo de terror.

\authorandplace{Sergi Cervera}{1r d’ESO}



\subsection*{Fi de curs a Bucarest}
\emph{Autor; Pinyol, Joan. Ed. Baula}

Radu Petrescu és un noi que ha de marxar de casa perquè el seu pare l’ha obligat a robar i el persegueix la policia.

Radu se’n va de Cosbuc, el seu poble, cap a Bucarest, la capital de Romania.

Durant el viatge cap a Bucarest coneix  uns nois ( Vasile, Moisei ,l'Ovidi i en Juliu) que pertanyen a una organització que no vol permetre que Romania entri a La Unió Europea. En Radu s’uneix a ells ja que no té diners.

Un grup  d‘estudiants catalans arriba a Bucarest de viatge de fi de curs de 4t   d’ESO, entre ells un noi anomenat Raül. 

El dia de l’arribada dels catalans el grup d’en Radu i d’en Raül es troben en  un parc  i es barallen. Durant aquest enfrontament els amics d’en Radu li roben la càmera a en Raül.

En Raül havia de recuperar la càmera fos com fos, ja que era del seu pare.  

En Radu li promet que  l’ajudarà a recuperar la càmera. Uns dies més tard en Radu desapareix.

En Raül i la Tatiana, una amiga romanesa, presenten una denúncia per la desaparició d’en Radu Petrescu. ...

 I la història continua..............

\authorandplace{Carla Jiménez}{2n ESO}

\subsection*{Sis històries al voltant d’en Màrius}
\emph{Autor: Sierra i Fabra, Jordi. Ed Planeta	}

Aquest llibre tracta de la vida d’un noi, en Màrius, que relata la seva vida des que neix fins als divuit anys; no obstant, ell no n’és el narrador sinó les diferents persones que ens l’expliquen, gent del seu entorn, família i amics.

Cada història és un any o dos de la seva vida, i cada persona explica el que va viure amb en Màrius

Si el llegiu veureu que en Màrius acaba tenint problemes amb les drogues i que no li serà gens fàcil sortir-se’n.

Aquest llibre l’he trobat entranyable i fascinant, és d’aquells que el llegiries més d’una vegada

\authorandplace{Maria Garcia}{ 3r ESO}



\subsection*{Pots comptar els estels?}
\emph{Autor: Lowry, Lois. Ed La Magrana}

Som a l’any 1943 i, per a l’Annemarie , viure a Copenhaguen és una barreja de rutina, d’escassetat de menjar i d’opressió per la presència constant de les tropes nazis.
La seva família ajuda els seus veïns a escapar de Dinamarca quan els nazis comencen a endur-se els jueus, i amaguen l’Ellen, la millor amiga de l’Annemarie, tot i córrer el risc de ser descoberts.

\authorandplace{Davínia W. Vilagrasa}{ 3r d’ESO}

\subsection*{Bitllet d’anada i tornada}

\emph{Autor: Lienas, Gemma. Ed. Estrella Polar}

Bitllet d’anada i tornada és el testimoni d’una noia que pateix anorèxia i és ingressada a l’hospital on coneix a altres noies que tenen el mateix problema. Però la Marta, la protagonista, no vol engreixar-se de cap de les maneres ja que té por de patir bulímia. Allà recorda com era la seva vida, abans d’ingressar a l’hospital, amb la família, amb el Rcky, la seva parella, i amb la seva millor amiga, la Clàudia.
Aquesta novel.la la trobo molt recomanable ja que és una història sobre la realitat de molts joves d’avui en dia, és un llibre que et mentalitza. Crec que l’autora sap com explicar i parlar d’aquest tema tan delicat, ens fa veure que l’anorèxia no és cap broma ni cap diversió. És una malaltia molt greu. 

\authorandplace{Raquel Madrenas}{4t ESO}


\subsection*{Memòries d’Idhum}

\emph{Autora: Gallego, Laura. Ed. Cruïlla}

És un llibre que combina molt bé la història d’amor amb la fantasia i un conjunt d’éssers fantàstics
La trama es du a terme a la terra i a Idhum, un món màgic on un tirà anomenat Ashran és al poder. En Jack, la Victòria i els seus amics, els anomenats “la resistència”,  lluitaran per aconseguir la llibertat de la seva terra i la seva i també pròpia.. Escenaris com boscos que creixen a grans velocitats, deserts mortífers, muntanyes canviants i paisatges gelats on hi viuen gegants..., tot plegat és una explosió de fantasia i màgia
\authorandplace{Esteve Pérez}{4t ESO}

\end{news}


\newpage
%doc: Revista 3/PresentacioRB/presentacio.docx
\begin{news}
{2} %columnes
{Presentació del llibre: El documental com a estratègia educativa}
%index: El documental com a estratègia educativa
{El Ramon Breu, professor  de l’escola i company nostre, el dia 20 d’octubre va presentar el seu darrer llibre : EL DOCUMENTAL COMO ESTRATEGIA EDUCATIVA, a la Biblioteca plaça d'Europa}
{Fem Escola}
{250} %pagesof

\noindent\includegraphics[width=9cm,keepaspectratio]{fem_escola/img/llibre_PA200015b.jpg}

%Amadeu Torner, 57 de l’Hospitalet de Llobregat.

Participaren de la presentació  David Urrea, director de la Biblioteca Plaça d'Europa, Cinta Vidal, directora d'Edicions de l'Editorial Graó i Alba Ambròs, professora de la Facultat d'Educació de la Universitat de Barcelona.

Va ser un acte entranyable i íntim;  en Ramon va estar molt ben acompanyat per la família, els companys, els  amics i per professionals diversos. 
A més, la biblioteca és un lloc molt significatiu i proper per en Ramon ja que està ubicada en  l’entorn on ell ha passat la infància i bona part de la seva vida.

Com a professionals de l’ensenyament valorem aquest llibre com una bona eina didàctica per a l’educació en valors i per a l’estudi i anàlisi del documental a les aules. Estem molt orgullosos de la il·lusió, dedicació i esperit de renovació que el nostre company és capaç d’aportar a l’ensenyament tot elaborant diferents materials didàctics i de reflexió que permeten millorar la qualitat pedagògica de les escoles. 

\authorandplace{Per molts anys, Ramón !}{}

\end{news}


%doc: Revista 3/Banc dels aliments - Arnau Bilbao (3r d'ESO).docx

\begin{news}
{2} %columnes
{Xerrada sobre el Banc dels Aliments}
{El divendres 12 de novembre,  un dels 92 voluntaris del banc dels aliments va venir a l’escola Solc per fer-nos una xerrada als nois i noies de 3r d’ESO sobre el Banc dels Aliments}
{ESO}
{404} %pagesof


Va ser una xerrada molt instructiva i alhora molt divertida ja que el senyor que va venir ens va fer la xerrada d’una manera diferent a altres xerrades que havia escoltat jo anteriorment. Ens va explicar moltes coses: què era el banc dels aliments, de què s’encarregava, què o qui el formaven...

Segons ens va explicar el voluntari, el banc dels aliments és una entitat que s’encarrega de recollir aliments per tota Catalunya, en instituts i escoles que col·laboren en la recollida d’aliments que es produeix un cop cada any durant una setmana. També recullen excedents de marques de menjar (Tarradellas, Ferrero Rocher, etc...). Tot aquest menjar s’aconsegueix gratis. Un cop tenen el menjar el guarden uns dies fins que gent pobra d’aquí, a Catalunya, en necessita per alimentar-se,  ja que no tenen prou diners per aconseguir-ne. El menjar es reparteix amb 4 furgonetes que tenen entre tots els voluntaris. A vegades al Banc dels Aliments els falta algun tipus determinat de menjar, llavors fan una mena de subhasta per comprovar quina marca d’aquell aliment els el donen més barat i llavors el compren. En el Banc dels Aliments no s’accepten ni begudes amb alcohol ni queviures quasi caducats;  tot i així aquests aliments sí que els accepten perquè no els hagin de portar a l’abocador.

Tot això ens  ho va explicar, també,  amb un “powerpoint”. En el “powerpoint” s’hi explicava què era el banc dels aliments (bé, ens explicava el mateix que el voluntari). Ens vam divertir molt en aquesta xerrada perquè el voluntari era molt divertit i quan algú feia alguna pregunta interessant li donava un caramel! A part d’això també va ser una xerrada molt instructiva.

Ens ho vam passar d’allò més bé fent feina normal i corrent.

També ens va dir que, si volíem, podíem anar al Banc dels Aliments a col·laborar. Un company de la nostra classe ho va fer fa pocs dies, i tant ell com jo us animem a presentar-vos, si voleu, com a voluntaris del Banc dels Aliments, una cosa que s’agrairia molt ja que allà tenen molta feina per fer i poca gent per fer-la. Podeu trobar més informació a www.bancdelsaliments.org.

\authorandplace{Arnau Bilbao}{3r d’ESO}

\end{news}



\newsection{Parvulari}

%doc: Parvulari/que_es_el_primer.doc
\begin{news}
{2} %columnes
{Què és el primer que treballem quan arribem a l’escola ?}
{\noindent\includegraphics[width=18cm,keepaspectratio]{parvulari/img/foto3b.jpg}}
{Parvulari}
{21} %pagesof

Entenem com a hàbits aquelles situacions que es repeteixen de forma sistemàtica i ens ajuden a agafar seguretat i a poder preveure els fets.

També ens ajuden a estructurar-nos, a orientar-nos i a evolucionar millor.
Els bons hàbits també possibiliten una relació de convivència positiva amb tots i amb tot.

Quan més estructurats estiguem, més preparats estarem per la nostra autonomia personal. 

Els hàbits s’han de treballar a casa i a l’escola. A mida que anem assolint els diferents hàbits ens sentirem més segurs, tranquils i amb ganes d’aprendre.
Aprendre a observar els petits progressos que fa cadascú, dia a dia,  i saber valorar-los és per a la persona  una motivació important i necessària per continuar avançant.

Aquests  són alguns dels hàbits que aprenem a la classe de cargols, a la classe de pingüins i a la dels elefants.

Els cargols expliquen: hàbits d’autonomia
La Metis ens ensenya a rentar-nos les mans, abans ho fèiem  malament i no ens quedaven netes , ara ja ho fem bé perquè som grans .


\noindent\includegraphics[width=9cm,keepaspectratio]{parvulari/img/foto1b.jpg}

També hem après a treure’ns la jaqueta i posar-nos la bata nosaltres solets i, quan no podem, ens ajuda la Metis i el Martí


\noindent\includegraphics[width=9cm,keepaspectratio]{parvulari/img/foto2b.jpg}



Els pingüins expliquen: hàbits socials

A la classe dels pingüins,  la Rosa ens ensenya que quan volem dir una cosa  hem d’aixecar la mà i,  si hi ha moltes mans alçades, ens hem de saber esperar i no parlar tots alhora.



%\noindent\includegraphics[width=9cm,keepaspectratio]{parvulari/img/foto3b.jpg}

Ara hem après a compartir les joguines,  a endreçar-les i a recollir les coses a poc a poc i respectant el material



\noindent\includegraphics[width=9cm,keepaspectratio]{parvulari/img/foto4.jpg}

Els elefants expliquen: hàbits de treball

Nosaltres aprenem a escriure i a fer frases i la Toni ens posa deures per aprendre molt

\noindent\includegraphics[width=9cm,keepaspectratio]{parvulari/img/foto5b.jpg}

Quan ens enfadem ara ja no ens barallem, parlem entre nosaltres perquè ja som molt grans

Fem quadern de matemàtiques, anem a l’aula d’informàtica i treballem amb la P.D.I.


\noindent\includegraphics[width=9cm,keepaspectratio]{parvulari/img/foto6b.jpg}


Ara ja som grans i abans de contestar,  primer hem de pensar.


Tots aquests hàbits són  necessaris per a l’adquisició dels nous aprenentatges que assolirem al llarg de tota la nostra escolaritat.


\authorandplace{Les mestres de Parvulari}{Escola Solc}

\end{news}

\noindent\includegraphics[width=12.5cm,keepaspectratio]{parvulari/img/foto7.jpg}




\newsection{Primària}
\definecolor{color}{rgb}{0.1 , 0.1 , 0.1}

\begin{news}
{2} %columnes
{El món de la Cèl·lula}
{El Dimecres 17 de març, els nens i nenes de 5è de primària de l’Escola Solc vam anar al CosmoCaixa a fer un taller de ciència que es diu “El món de la cèl·lula”}
{Primaria}
{5} %pagesof



Quan vam arribar al Museu vam conèixer a l’Eli i a la Irene, que van ser les monitores que ens van ajudar a fer les activitats. Hi havia dos tallers: un consistia en preparar mostres per observar pel microscopi i l’altre era mirar mostres ja preparades (orella de gos, testicle de rata,...). A la primera activitat, amb un tomàquet i una ceba havíem de fer les preparacions.
Després, un cop havíem fet tots dos tallers, vam estar una hora voltant pel CosmoCaixa. Jo vaig anar al Bosc Inundat, vaig tocar un bloc de gel gegant que hi havia, ...
M’ho vaig passar molt bé perquè el Museu de la Ciència és molt interessant.


\authorandplace{Albert Morales}
							{5è de Primària}

\noindent\includegraphics[width=9cm,keepaspectratio]{primaria/img/celula_DSC00655.JPG}

\end{news}

\newssep
\definecolor{color}{rgb}{0.1 , 0.1 , 0.1}

\begin{news}
{2} %columnes
{Elaboració dels titelles}
{Aquest curs, els nens de 5è de Primària estem preparant unes obres de titelles per representar-les al parvulari}
{Primaria}
{028} %pagesof




Primer havíem de portar un globus i un cilindre de cartró. Al voltant del globus enganxàvem paper amb cola. La cola feia que el cilindre s’ajuntés amb el globus. Quan es va eixugar el paper vam tornar a posar més capes de paper amb cola perquè ens quedés l’estructura més forta. Uns dies després, quan ja s’havia assecat, vam pintar el titella amb pintura blanca perquè no es veiés el diari. 

\columntitle{lines}
{Els titelles els fem nosaltres i les obres també les triem nosaltres}

Un altre dia el vam pintar amb els colors del personatge; per exemple, el meu  era un ós i el vaig pintar de color marró. Uns dies més tard, vam portar fil, agulla i tela per fer el vestit al titella. Tallàvem la tela en forma de vestit i després la cosíem. Fet això, el Ferran enganxava el vestit al cap amb belcro. Ja quedava fet el titella. També vam fer el guió de l’obra i un còmic per fer les fotografies. Aquestes ens serviran per fer un ``còmic digital" amb l’ordinador. Cap al juny representarem l’obra a parvulari i a la festa de final de curs podrem veure els ``còmics digitals".


M’agrada molt fer plàstica  sobretot, en aquesta ocasió, perquè hem treballat fent titelles i fent el ``còmic digital".

\authorandplace{Mar Garcia}
{5è de Primària}

\noindent\includegraphics[width=9cm,keepaspectratio]{primaria/img/titlles_513.jpg}

\end{news}
\definecolor{color}{rgb}{0.1 , 0.1 , 0.1}

\begin{news}
{2} %columnes
{Colònies de 1r i 2n de Primària}
%index: Colònies de Primària
{\noindent\includegraphics[width=18cm,keepaspectratio]{primaria/img/colonies_246.jpg}}
{Primaria}
{29} %pagesof

Dimarts 6 d’abril vam anar de colònies a una casa que es deia Can Putxet. Vam anar amb autocar  i era una mica lluny.

Quan vam arribar,  ens vam posar molt contents. Tenia un camp de futbol i una pista de tennis que semblava un camp de bàsquet. Vam estar jugant una estona, després vam dinar entrepans; un era de formatge i l’altre de pernil i, de postres, poma. Vam continuar jugant i va començar a ploure,. 

Vam entrar a la casa de dos en dos. A mi em va tocar amb el Joan, va voler dormir a dalt i jo vaig dormir a baix, tenia al Casi al davant i al costat l’Elena.

%\columntitle{lines}
%{També vam fer una excursió de tres quilòmetres i ens vam cansar molt}

Després vam fer unes activitats. La que més ens va agradar  va ser un joc d’equip. Els equips els vam fer segons els colors d’uns pitets  que ens van donar. Jo anava amb l’Alejandra, l’Elena, el Guillem T., el Gullem R., el Pol de primer, l’Aida i l’Oscar. Érem el color groc. Havíem de trobar 10 targetes d’animals. Només en vam trobar 8, perquè una se’ns va perdre i d’una altra ja  no en quedaven. Teníem un mapa per trobar els topants on buscar les targetes.

Per sopar hi  va haver sopa, truita de patates amb croquetes miniatura i per  postres iogurt. També vam fer una excursió de tres quilòmetres i ens vam cansar molt.

L’última nit vam fer un joc de nit, havíem de trobat tres claus: una al bosc, que era molt fàcil i la va trobar el Pol. L’ altra al camp, que va ser molt difícil i la va trobar l’Àlex. L’altra al riu, la va trobar l’Éric.

M’ho vaig passar molt Bé.

\authorandplace{Martina Tarrés Castellanos}{Primària}

\end{news}

\definecolor{color}{rgb}{0.1 , 0.1 , 0.1}

\begin{news}
{2} %columnes
{}
{}
{Primaria}
{029} %pagesof

%\vspace*{1cm}

El dimarts vam anar de colònies amb motxilles o maletes, les vam deixar al camp de futbol i vam començar a jugar. Després vam anar a la casa de can Putxet a veure-la per dintre. Vam començar per les habitacions, després pels lavabos. 

Vam tornar a les habitacions a fer el llit. Quan vam acabar, vam deixar la roba al calaix. Després vam jugar al camp de futbol un altre cop. Vam jugar a pilota. Després vam anar a passejar al camp, vam veure una riera.

Després vam tornar, quan vam arribar vam  anar a sopar. Després vam anar al llit a llegir i tot seguit a dormir. Quan ens vam llevar vam esmorzar i després vam fer una excursión que vam caminar tres quilòmetres. 

Vam berenar i, després, vam tornar a la casa de colònies, quan vam tornar vam sopar i després vam anar al llit, l’endemà marxàvem. Vam esmorzar, després arribava una altra escola, Vam jugar a una carrera de globus i també vam fer bombolles. 
Vam anar a dinar. I, després, vam marxar. 

\authorandplace{Màrio Ródenas Sistané}{Primària}


\end{news}
\definecolor{color}{rgb}{0.1 , 0.1 , 0.1}

\begin{news}
{3} %columnes
{Les Menines}
{}
{Primaria}
{30} %pagesof

\noindent\includegraphics[width=6cm,keepaspectratio]{primaria/img/picasso_ariadna.jpg}


Velázquez va ser un gran pintor del segle XVII, tot i que avui en dia les seves obres d’art són molt conegudes. Ell
 va ser l’autor de “Les Menines” que està exposat al museu del Prado, a Madrid. Si observem bé el quadre podem veure-hi el seu autor; al fons hi ha un mirall on s’hi pot veure Felip IV i Mariana d’Àustria, al seu costat José Nieto, prop de Velázquez hi ha una menina, al mig hi ha la infanta Margarida, al seu costat hi té una altra menina, també hi ha una nana, un nan, un senyor i una monja. 


Pablo  Picasso, també un gran pintor del segle XX, va imitar  “Les Menines” a la seva
manera, és a dir,  amb formes geomètriques, amb punts, amb carbonet, va afegir elements, personatges, etc. Va fer unes 58 versions sobre aquest quadre. 


A la nostra escola, els alumnes de 5è i de 6è vàrem fer un treball sobre “Les Menines” abans de la visita al museu Picasso.
Vàrem observar l’original i algunes de les moltes versions de Picasso. Després de parlar-ne, vam dibuixar l’obra a la nostra manera. Per exemple, amb pals, estil modern, còmic, puntillisme,...  En fi, tothom  va fer el seu estil i van quedar  tots molt bonics.

\noindent\includegraphics[width=6cm,keepaspectratio]{primaria/img/picasso_conrad.jpg}

\authorandplace{Guillem Morales}{6è de Primària}

\end{news}

\newssep
\definecolor{color}{rgb}{0.1 , 0.1 , 0.1}

\begin{news}
{2} %columnes
{Visita al museu Picasso}
{}
{Primaria}
{030} %pagesof


El dimarts 23 de març vam anar al Museu Picasso, que es troba al Barri Gòtic de Barcelona, al carrer Montcada, que és un carreró estret, antic i sense cotxes. Allà vam aprendre moltes coses d’aquest pintor malagueny.  

Quan hi vam entrar jo em vaig sentir com un veritable artista. Ens van explicar la vida de Pablo Picasso, la del seu pare (José Ruiz) i la de la seva mare (Maria Picasso). Ens van dir que Picasso va començar a dibuixar amb carbonet, que va presentar el quadre de “La dona malalta” a la universitat de Belles Arts de Madrid, que el seu pare era  professor en una universitat que es deia “Llotja”. També ens van ensenyar quatre estils de la seva pintura: carbonet, l’època blava, figures geomètriques i el puntillisme. 

Vam veure el quadre “La dona malalta” que, segons ens van explicar,  representava una dona molt malalta al llit amb un metge prenent-li el pols i una monja oferint-li un got d’aigua, aquesta portava un nadó a les mans. En realitat, el metge era el pare del Picasso, la dona malalta, una pidolaire, el nadó, el fill de la pidolaire i, la monja, era un amic de Picasso.

A més a més de veure els quadres més importants d’aquest pintor, em va agradar poder llegir la seva vida, que estava escrita a diverses parts del museu. 

\authorandplace{Óscar Redondo Mora}
				{6è de Primària}

\noindent\includegraphics[width=8cm,keepaspectratio]{primaria/img/picasso_julia.jpg}

\noindent\includegraphics[width=8cm,keepaspectratio]{primaria/img/picasso_raquel.jpg}

\end{news}
\definecolor{color}{rgb}{0.1 , 0.1 , 0.1}

\begin{news}
{2} %columnes
{La Bombeta}
{}
{Primaria}
{029} %pagesof

\noindent\includegraphics[width=9cm,keepaspectratio]{primaria/img/bombeta.png}

L’invent de la bombeta s’atribueix a Thomas Alva Edison que va presentar la patent  el 21 d’octubre de 1878. Està formada per un filament de wolframi molt primet situat a l’interior d’una “ampolla” de vidre que s’ha omplert posteriorment d’un gas inert o bé se li ha fet el buit. A la part de sota de la bombeta hi ha una part metàl·lica on s’hi troben totes les connexions elèctriques. Aquesta part metàl·lica, normalment té una una rosca que permet caragolar la bombeta al portalàmpades. La bombeta elèctrica també s’anomena làmpada incandescent.  De tota l’energia que consumeix la bombeta, només el 10\% es converteix en llum; la resta, es transforma en escalfor, llum ultraviolada i llum infraroja. 
La bombeta és un dels objectes més utilitzats des de la seva invenció, ja que tots en tenim, com a mínim, una a casa.

\authorandplace{Conrad Galli}{6è de Primària}

\end{news}

\newssep
\definecolor{color}{rgb}{0.1 , 0.1 , 0.1}

\begin{news}
{2} %columnes
{L'ós}
{}
{Primaria}
{029} %pagesof

\noindent\includegraphics[width=9cm,keepaspectratio]{primaria/img/os.png}

Els óssos són omnívors, ara bé, l'ós polar, a causa de la manca d'altres fonts d'aliment, té una dieta quasi exclusiva de carn. Amb els seus pesats cossos i les seves poderoses mandíbules, els óssos es compten entre els majors carnívors que viuen a la Terra. Un mascle d'ós polar pot pesar més de 600 kg i pot aconseguir 1,60 m. d'altura quan es posa de quatre potes. Es desplacen amb uns moviments pesats tot  recolzant tota la planta dels peus. Posseeixen orelles curtes i cua rudimentària. Tenen una dieta variada i, tot i la seva temible dentadura, molts d'ells mengen fruits, arrels i insectes, a més de carn. 
Contràriament al que pot semblar a primera vista, ha estat molt debatut si els pandes són o no óssos, encara que els últims exàmens genètics suggereixen que els grans pandes sí que ho són.


L'ós era un animal sagrat al nord d'Europa, on era el rei del bosc. Se'l representa en els mites i llegendes com l'animal més semblant a l’home; fins i tot, arribava a tenir relacions amb dones humanes (fet pel qual l'església cristiana el va degradar com a associat al dimoni a partir de l'any mil). El nom propi Úrsula deriva directament de l'ós. Representa els països de Finlàndia i Rússia i és un emblema de la comunitat de Madrid.

\authorandplace{Arnau Ruiz}{6è de Primària}




\end{news}
\definecolor{color}{rgb}{0.1 , 0.1 , 0.1}

\begin{news}
{4} %columnes
{L’Ot i els alumnes de 3r de Primària}
% index: L'Ot i 3r de Primària
{}
{Primaria}
{15} %pagesof


\noindent\includegraphics[width=4.5cm,keepaspectratio]{primaria/img/ot_imagen.jpg}

\bf L’Ot va a Egipte

\rm
Un dia l’Ot va anar amb l’escombra màgica de passeig i va arribar a Egipte, i mentre volava es va trobar a tres àrabs. Es veia que aquells senyors volien anar a  algun lloc. Finalment l’Ot va dir unes paraules màgiques: PLOP!! I, de sobte, l’escombra es va convertir en una catifa voladora. Així van poder pujar els tres àrabs que van quedar contents i resant cap al poble.

                                      Eloi Ibáñez           


\noindent\includegraphics[width=4.5cm,keepaspectratio]{primaria/img/ot_imagen_001.jpg}

\bf L’Ot i la senyora globus

\rm
Hi havia una vegada un bruixot que es deia Ot i també una princesa anomenada Berta. Un dia la Berta era en el seu castell i va aparèixer l’Ot i li va dir: 
-   Hola princeseta, estàs bé? 
Sí, va dir la Berta; i tu què fas per aquí?
Volia saludar-te i  veure com estaves.
Doncs estic molt bé.
Aleshores, l’Ot va agafar la Berta per les cames i la va inflar com un globus i va fer PLOP! Llavors va posar-li una corda i va agafar-la tot caminant... I així és la història de l’Ot i la Berta.
     
                                                           Adriana Bosch


\noindent\includegraphics[width=4.5cm,keepaspectratio]{primaria/img/ot_img025.jpg}

\bf L’Ot i les amigues de la Berta

\rm
Un dia l’Ot s’estava banyant i quan va acabar de banyar-se va agafar la tovallola i es va eixugar. Va obrir la porta i va trobar les amigues de la Berta. Aleshores, perquè no el veiessin, va fer ús de la seva màgia i va caminar pel sostre. Va anar a l’altra porta i les amigues de la Berta no el van veure.

                                                                Irene Cuadra

\noindent\includegraphics[width=4.5cm,keepaspectratio]{primaria/img/ot_img022.jpg}

\bf La gran barba

\rm

Una vegada hi havia un bruixot que es deia Ot i anava cap a la perruqueria. Llavors va dir: - no tinc cabells perquè me’ls tallin, doncs m’hauré de tallar el bigoti i això no m’agradaria, seria imperdonable. Aleshores, com que era un bruixot, va fer-se aparèixer una gran barba a la barbeta. Va entrar i no va sortir fins al dia següent, però no li van tallar la barba sinó el bigoti. Va sortir enfadat; en canvi,  de content , gens ni mica.

                                                             Ferran Corral 


\end{news}
\definecolor{color}{rgb}{0.1 , 0.1 , 0.1}

\begin{shortnews}
{3} %columnes
{Ressenyes de llibres}
{}
{Primaria}

\shortnewsitem{La marmota inventora}
{
Autor: Enric Larreula / Rita Culla.   Editorial: La Galera 

Hi havia una vegada una marmota molt llesta que un dia va fer un invent, una bufanda. La va tallar a trossets i la va regalar a totes les marmotes. La bufanda la va fer servir per dormir. La marmota vivia en unes muntanyes molt altes anomenades els Alps.

A mi m’ha agradat molt perquè tracta d’invents i de marmotes.
									
Biel Castro, 1r de Primària
}

\shortnewsitem{La campaneta de plata}
{
Autor: Conte popular.   Editorial: Susaeta

En una torre hi vivia la mare dels vents, que tenia tres fills: el vent del Nord, la Brisa i el vent de Llevant. Quan es van fer grans van viatjar per diferents mons, i van deixar una campaneta de plata a la mare per si els volia cridar.  Com que la mare es sentia sola, va tocar la campaneta per reunir els tres germans, però de tant vent, el poble es pensava que a la torre hi vivia una bruixa.

M’ha agradat aquest llibre perquè la mare reuneix els seus fills tocant la campaneta.
									
Aida Perez, 1r de Primària
}

\shortnewsitem{Pinotxo}
{
Autor: Carlo Collodi      
\noindent\includegraphics[width=4.5cm,keepaspectratio]{primaria/img/llibres_pinocho.jpg}

Un dia un fuster anomenat Gepetto va construir un titella, que va anomenar  “Pinotxo”. Va desitjar que  ell tingués vida. A la nit va venir una fada i va fer que el titella cobrés vida. Pinotxo es va sorprendre molt perquè podia parlar, caminar, menjar, anar a l’escola, etc. Pepito Grillo era la consciència de la marioneta, que el va ajudar a ser conscient del que feia. 

Em va agradar molt perquè la fada va fer màgia i a mi això m’agrada molt. Aquest llibre el recomano perquè és molt divertit, emocionant i hi passen moltes aventures.

Pol Hernández,  2n de Primària
}

\shortnewsitem{El mag Merlí}
{
Autor: Llegenda Popular.    Editorial: Susaeta  

Un nen anomenat Grill vivia en un castell a Alemanya. Un dia en Grill amb un amic que es diu Key van anar a caçar, però com que en Grill era novell  en la caça, va caure des d’un arbre a sobre d’en Key fent que llancés una fletxa al cel; en veure el que havia fet, en Grill li va dir que aniria a cercar la fletxa. En Grill anava cercant la fletxa quan, de sobte, va veure la fletxa en una branca, l’anava a agafar,  va relliscar i va caure en una cabana on hi vivia en Mag Merlí.  Li va dir que havia de  tornar al castell...    

A mi el que més m’ha agradat és quan llençava la fletxa. I,quan estan caçant, hi ha un dibuix que em va agradar molt.

Damià Rubió ,  2n de Primària
}

\shortnewsitem{La Xola i els lleons}
{
Autor: M. Dolors Alibés.   Editorial: Cruïlla

És una gossa que es diu Xola i que es pensa que es un lleó, llavors l’imita. El seu amo,  en Marc, té un amic que investiga la selva, llavors l’amic porta un llibre de lleons a casa la Xola i se’l deixa. La gosseta, com que està tan interessada en els lleons, l’agafa i el llegeix i descobreix que els lleons no són tan fantàstics com ella creia i ja no vol ser un lleó. I ara és una gosseta petonera com abans era.

M’ha agradat molt perquè és molt misteriós i no saps què passarà. També perquè mai  pots decidir ser un lleó si no saps res sobre ells.

Andrea Amador Alcaina,  2n de Primària
}

\shortnewsitem{Geronimo Stilton. Les entranyes de les rates pudents}
{
Autor: Geronimo Stilton.   Editorial: Destino

Aquest llibre parla que Ratalona comença a fer molta pudor. Una nit, de sobte,  de les clavegueres surten globus pudents. En Gerónimo baixa a les clavegueres per esbrinar què feia tanta pudor. Caminant per les clavegueres, arriba a Ratcity, la ciutat de les rates. Hi ha una rata que es diu Clavegueram que és una mica dolenta, i quan arriba a Ratcity s’enamora del Xafarot, l’amic d’en Geronimo i es vol casar amb ell. En Xafarot i en Geronimo agafen una moto d’aigua i s’escapen.

Ens ha agradat molt perquè aquest llibre és de misteri. També ens agraden molt les aventures d’en Gerónimo Stilton.

Eloi Banyes, Ferran Corral i Roger Guarro, 3r de Primària
}


\shortnewsitem{La Tanga i el gran lleopard}
{
Autor: Roberto Malo i Fco Javier Mateos.  Editorial: Comanegra

La Tanga és una noia que viu en un poblat al cor de la selva. En el poblat hi ha un mag que es diu el Gran Bruixot i que té poders màgics. Un bon dia, un lleopard ferotge i cruel va menjant-se tots els animals del poblat. A partir d’aquell succés van començar a patir fam, i el Gran Bruixot va intentar aturar e lleopard, però no va funcionar. A la nit es va posar una màscara per transmetre un missatge als ciutadans. I va dir a tot el poblat que el dia següent tothom que volgués enfrontar-se al lleopard es presentés, però no va aparèixer ningú, excepte la Tanga. Llavors el Gran Bruixot li va donar un ganivet de fusta de banús... 

Ens ha agradat molt perquè aquest llibre és una aventura molt divertida i perquè ens agrada molt la màgia i el Gran Bruixot en fa molta.

 Pablo de Quadras i Eduard Tenas,  3r de Primària
}


\shortnewsitem{El zoo d’en Pitus}
{
Autor: 	Sebastià Sorribas.  Editorial: La Galera

\noindent\includegraphics[width=4cm,keepaspectratio]{primaria/img/llibres_pitus.jpg}

El Pitus té una malaltia. Només el pot curar un metge molt famós de Suècia. Els seus amics, el Tanet, el Manelitus, en Cigró, en Juli, en Fleming i la Mariona van tenir una gran pensada: com que a en Pitus li agradaven molt els animals, van decidir fer un zoo per a ell.

M’ha agradat molt l’argument d’aquest  llibre.

Roser Pérez, 4t de Primària
}

\shortnewsitem{Tina superbruixa. L’aniversari d’en Pitus}
{
Autor: Knister   Editorial: Bruixola

\noindent\includegraphics[width=4cm,keepaspectratio]{primaria/img/llibres_fada.jpg}

Aquesta és una de les emocionants històries de la Tina. És l’aniversari d’en Pitus i està molt emocionat i vol fer una súper festa, però el seu pare està de viatge i la seva mare en un curs. Ve la seva tieta, que és una mica bleda  amb els nens. A la Tina i al Pitus els tracta com a nadons. Quan arriben els convidats, la tieta Elisa es queixa i es queixa fins que la Tina fa un dels seus encanteris i... Plaf! La tieta s’adorm.

M’ha agradat perquè és molt divertit, és una mica curt, però val la pena. És entretingut i el recomano a tothom.

Ariadna Vera, 4t de Primària
}

\shortnewsitem{Els rebels de la cabana}
{
Autor: David Nel·lo.   Editorial: Cruïlla

Va d’un grup de nens que tenen feta una cabana a dalt d’un arbre i l’alcalde vol construir un aparcament al Prat, tallaran l’arbre on hi ha la cabana. La Margot, que és la que mana, vol negociar amb ell perquè no tallin l’arbre, però l’alcalde diu que no. Aleshores, els nens van dir que si no negociaven, no baixarien de la cabana. 

És molt divertit perquè els nens són molt atrevits ja que es queden una nit de pluja a la cabana. Jo el recomano perquè és molt emocionant.

Aina Boronat, 5è de Primària
}

\shortnewsitem{La tribu de Camelot}
{
Autora: Gemma Lienes.    Editoria: Empúries

El Papagueno, el canari de la veïna de la Carlota ha desaparegut i la tribu passa moltes aventures amb un llibre de màgia que els ajuda a retornar en Papagueno a la veïna de la Carlota.  

Aquest  llibre m’ha agradat molt perquè hi ha moltes aventures i és molt intrigant per saber qui ha raptat Papagueno. M’entretenia a trobar els dracs que hi havia amagats a moltes pàgines, també eren molt divertides les olors i les tintes.  
Paula Claver, 5è de Primària
}

\shortnewsitem{Pocahontas}
{
Autor: Walt Disney.   Editorial: Beascoa

La història comença quan va arribar una expedició de navegants a la terra dels indis.
L’objectiu era trobar or i enriquir-se. Casualment, en John coneix la Pocahontas i se n’enamora. Al pare de la Pocahontas  això no li agrada ja que volia que es casés amb en Kocum. En una baralla per la Pocahontas, mor en Kocum i fan presoner en John. Els amics d’en John corren a salvar-lo amb les escopetes...

És una història d’amor i d’aventures i a mi m’ha agradat molt!

Eva Llovera, 5è de Primària.
}


\shortnewsitem{La meravellosa medicina d’en Jordi}
{
Autor: Roald Dahl.   Editorial: Empúries

En Jordi és un noi que  sempre ha d’estar cuidant la seva àvia i preparant-li els medicaments. Un dia, per experimentar,  li va preparar  una meravellosa medicina, quan l’àvia se la va prendre, es va fer molt alta i sobresortia pel terrat de la casa. Al seu pare li va agradar aquest experiment per donar-lo als animals, així serien més grans i tindrien més carn i els aprofitaria més. Va donar el medicament als animals i tots es van tornar  més grans. Al cap d’un temps, en va fer un altre i el va donar a l’àvia, llavors l’àvia es va tornar molt ...  

Aquest llibre és, al principi, una mica avorrit, fins que  comencen els experiments, però a mi el que més m’ha agradat són les coses que li passen a l’àvia.

Judit Molina, 6è de Primària
}

\shortnewsitem{Ales de foc}
{
Autor:Christopher Pike. Editorial: Edicions  agrupo  Z .

Ariel és un àngel que té una “protegida”que es diu Marla. Ariel acaba traïda per Marla, i és tancada a Gorilan, una presó “màgica”. Allà hi ha un rei que és un gripau molt intel·ligent. A Gorlian coneix un noi anomenat Brad, que acaba mort per l’exèrcit del rei gripau. Aleshores acaba sent rescatada per uns amics i viuen diferents aventures.

A mi aquest llibre m’ha agradat molt perquè és d’aventures i d’éssers màgics.

Dídac Benages , 6è de Primària
}

\shortnewsitem{Malsons}
{
Autor: Anne Fine. Editorial: Bromera.

Una nena anomenada Imogen canvia d’ escola, es fa  molt amiga de la Melanie. A les altres escoles que ha estat no tenia gaires amigues. La Melanie descobreix que Imogen té un secret, ple de màgia, de por, de fetilleria i ple d’encanteris. Aquest secret que li succeeix a Imogen serà bastant difícil de descobrir, només la Melanie ho podrà fer!

Aquest llibre, a mi m’ha agradat molt ja que és de misteri, d’aventures i a vegades semblen fets reals!!

Mariona Medrano, 6è de Primària
}

\end{shortnews}
\definecolor{color}{rgb}{0.1 , 0.1 , 0.1}

\begin{news}
{2} %columnes
{“Dins i fora”. Viure l’espai amb Beuys i Isozaki}
%index: L’espai amb Beuys i Isozaki
{}
{Primaria}
{22} %pagesof

%# 1.jpg  2.jpg  img003.jpg  img004.jpg
\noindent\includegraphics[width=8cm,keepaspectratio]{primaria/img/1.jpg}

\noindent\includegraphics[width=8cm,keepaspectratio]{primaria/img/img004.jpg}



El passat 5 de maig, els nens i nenes de primer de primària vàrem anar al Caixa Fòrum a veure dues obres, que es diuen instal.lacions, anomenades “Espai de dolor” i “El jardí secret”.

“Espai de dolor” és una instal.lació d’en Joseph Beuys. És un espai rectangular que té les parets i el sostre completament folrats de plom. Només es veu una bombeta pelada i dues anelles de plata que pengen al seu costat. En realitat és un espai hermètic en tots els sentits. Un cop dins la cambra, un es queda completament aïllat del món.
“El jardí secret” és una altra instal.lació; d’Arata Isozaki. És també un espai rectangular gairebé tancat, amb una obertura per entrar-hi, i sense sostre, que s’inscriu en un altre rectangle més gran que conforma tot el pati. No s’hi pot entrar però, si poguéssim, no tindríem la sensació d’estar tancats.
L’activitat dirigida que vam realitzar ens va permetre viure i tenir sensacions amb aquests dos espais. Heus aquí  l’opinió:


Vam anar al Caixa Fòrum i vam veure L’espai de dolor. Vam tocar les parets i estaven fredes i després vam anar al “Jardí secret” i vam fer soroll i ens arrossegàvem per les parets. M’ho vaig passar molt bé.

Jordi Montero


\noindent\includegraphics[width=8cm,keepaspectratio]{primaria/img/2.jpg}

\noindent\includegraphics[width=8cm,keepaspectratio]{primaria/img/img003.jpg}

El dia cinc de maig vam anar al Caixa Fòrum, vam veure dues instal.lacions; una es deia El jardí secret i l’altra Espai de dolor. També vam fer dos grups, un era el grup fosc i l’altra era el clar i, a mi, em va tocar el fosc, perquè tocava el que tocava i a dins de L’espai de dolor sentíem una veu i m’ho vaig passar molt bé.


Quima Lleonart

Vam anar a veure dues instal.lacions i una es deia L’espai de dolor i vam tocar les parets i les parets estaven fredes i eren de metall. Vam veure també el Jardí secret, i hi havia aigua i estava fet de pedres i ressonava tot quan fèiem un crit. Si hi havia silenci només sentíem el soroll de l’aigua.

Héctor Cuadra

Hem anat al Caixa Fòroum i vam mirar L’espai de dolor i el Jardí secret. A L’espai de dolor hi vam poder entrar i vam veure que al sostre hi havia dues anelles de plata, una més gran que l’altra. Representaven dos caps pensant. El cap d’una persona gran i el d’un nen. Dins l’espai ens vam arrossegar pel terra i vam tocar la paret i era fresca i tot estava molt fosc. Al Jardí secret podíem escoltar el silenci.

Eric Ramos




\end{news}
\definecolor{color}{rgb}{0.1 , 0.1 , 0.1}

\begin{news}
{2} %columnes
{Conferència de la Josefina Piquet a l'Escola}
{\noindent\includegraphics[width=18cm,keepaspectratio]{primaria/img/dones_36_p4230014.jpg}}
{Primaria}
{019} %pagesof




La passada diada de Sant Jordi, la senyora Josefina Piquet, coordinadora de l'associació “Dones del 36” va visitar l'escola per compartir les seves vivències amb els nois i noies de 4t.

La sessió fou molt emotiva i profitosa, tant des del punt de vista de l'educació en valors com de l'educació en ciències socials.

La nostra escola sempre ha tingut com a una de les seves prioritats transmetre  la memòria històrica del nostre país. En aquest sentit, la valoració de les repercussions de la nostra guerra fa possible analitzar la seva influència, tant en les generacions que la van viure directament, com en la societat actual. D'altra banda, treballar sobre el fenomen de la guerra contribueix a avançar en el debat sobre la resolució pacífica dels conflictes.

Amb xerrades com la de la Josefina Piquet (per cert, quan ve a l'Escola Solc diu que “es troba com a casa”) intentem recuperar la història recent de les dones mitjançant la seva pròpia veu, la seva memòria individual i col·lectiva; destacar la importància i la complexitat de la història transmesa oralment, de la seva metodologia i dels seus resultats i analitzar la diversitat de protagonistes i d'escenaris en la història de la vida quotidiana, especialment en moments excepcionals com és en temps de guerra, exili i postguerra. 

\end{news}

\newssep
\definecolor{color}{rgb}{0.1 , 0.1 , 0.1}

\begin{news}
{2} %columnes
{Fem un herbari!}
{}
{Primaria}
{029} %pagesof

%\noindent\includegraphics[width=9cm,keepaspectratio]{primaria/img/os.png}

Ja fa uns dies que vam anar al Jardí Botànic. Vam fotografiar els arbres que hi havia per poder fer després un PowerPoint a l’escola. Vam veure molts arbres i tots eren molt bonics. 
La Maria ens va dir que anéssim buscant fulles, durant uns quants dies, ja que faríem un herbari. També ens va dir que portéssim les fotos de la càmera per fer el PowerPoint. 

Per a l’herbari, la Maria ens va donar una bossa per posar-hi les fulles que primer havíem assecat posant-les entre fulls de diari, ben premsades. Llavors, en una cartolina on hi havia la descripció de l’arbre, hi enganxàvem les fulles i nosaltres havíem de fer un títol bonic amb el nom de l’arbre.

Jo he après moltes coses dels arbres. Ja sé com se’n diuen uns quants que no coneixia i com són les fulles i el tronc. M’ha agradat molt fer aquest herbari i, a més a més, m’ho he passat molt bé i ha estat molt divertit.

\authorandplace{}{4t de Primària}

\end{news}
\definecolor{color}{rgb}{0.1 , 0.1 , 0.1}

\begin{news}
{2} %columnes
{Sortida a Santa Fe del Montseny}
{\noindent\includegraphics[width=18cm,keepaspectratio]{primaria/img/montseny_DSC00744.JPG}}
{Primaria}
{19} %pagesof

{\noindent\includegraphics[width=9cm,keepaspectratio]{primaria/img/montseny_DSC00774.JPG}}

El dimecres 19 de maig,  la classe de 5è de Primària va anar al Montseny.

Quan vàrem arribar, vam fer un passeig per la muntanya mentre observàvem el paisatge. Però especialment vam observar un bosc de castanyers i un altre de faigs. A l’interior d’aquests boscos agafàvem un full, el col·locàvem sobre l’escorça d’un arbre i amb el llapis guixàvem i quedava calcada l’escorça. Això, també ho fèiem amb una fulla de l’arbre. Un cop fet, el dibuixàvem. Vàrem fer mitja volta al Pantà Santa Fe, és molt bonic i gran. Hi ha una gran resclosa per a retenir l’aigua. A la tornada ja podíem treure les càmeres per  fer fotografies. En acabar aquest itinerari, ens vam situar a ”Can Casades” , que és un lloc on hi ha una mena de masia. Allà, vàrem veure un audiovisual on sortia com canviava el Montseny en les diferents estacions de l’any. Un cop vist, vam dinar, vam jugar, i vam tornar al col·legi.  

\authorandplace{Marc Luna}
{5è de Primària}

\end{news}


\selectlanguage{english}
\newsection{Anglès}
\input{angles/esteve}


\begin{news}
{2} %columnes
{My Life}
{}
{Anglès}
{0505} %pagesof

%doc: Revista 3/writting unit 2 ariadna.doc

\subsection*{Ariadna Peiró}

I was born in Barcelona on the 22nd of February 1995. When I was young, we lived in a different house and when my sister was born in 1999 we moved to a bigger house, but we continued living in the same neighborhood.

My earliest memory is very clear. All the summers we used to go to my village in my grandparent’s house, and now I have a lot of friends that I met when I was little. Sometimes we remember the things that we did and we laugh a lot! I also remember my first day in Solc school, I remember that I was one of the few children that didn’t cry, because when I was younger I liked so much going to school but now is different because I go to school to study and do a lot of homework and in the past I used to go to school to play with my friends!

I’ve been in the same school and I’m in ESO. My friends have been the same but of course I’ve met new friends! I’ve changed since I was young. I think that I’m shyer with new people but the thing that hasn’t changed is my personality: I’m very cheerful!

When I finish secondary school, I’ll probably study higher secondary school on history and then I’ll probably go to university and I’ll study teaching or journalism. I’d like to study in Germany and meet new people, but I think that I will never forget the people that have been my friends in this school, because they’ve proved me that they’ve become my family!


\authorandplace{Ariadna Peiró}{4th ESO}



\input{angles/judit}

\end{news}


\begin{news}
{1} % columnes
{A famous person}
{}
{Anglès}
{452} %pagesof

\input{angles/nadal}

\subsection*{Shakira is a singer and dancer}

She was born in Colombia. Her favourite food is arabic food, seafood and chocolate; her favourite colour is black. As a hobby she likes making pottery. She can play the guitar and the harmonica. She usually dresses casually.


\authorandplace{Anna Bàsquet}{6è PRI}



\end{news}


%doc: 2a tramesa revista/escrits_angles_PRIMARIA/maria i jordi.doc
\begin{news}
{2} % columnes
{Spike’s web page}
{}
{Anglès}
{053} %pagesof

\subsection*{My school}

In our classroom we’ve got chairs,tables ,a board and a PDI .
In our gym we’ve got balls, hoops and a basketball net. We can run  jump, play basketball and play football.
In our computer room we’ve got computers a board and PDI. We can learn how to do some things with computers.

\authorandplace{Maria Morella}{4t Primària}



\input{angles/jordi}

\end{news}


\input{angles/portrait}



\selectlanguage{catalan}
\newsection{ESO}
\definecolor{color}{rgb}{0.1 , 0.1 , 0.1}

\begin{news}
{2} %columnes
{Redacció sobre les colònies a Fenals}
%index: Colònies d'ESO: Fenals i Camprodón
{Les colònies de tercer d’ESO són un mite entre tots els alumnes de la secundària; i és que només començar-la ja penses en elles, tothom parla que són uns dies únics i genials; doncs bé, a la fi aquest any hem arribat  tercer!}
{ESO}
{18} %pagesof

\noindent\includegraphics[width=9cm,keepaspectratio]{eso/img/colonies_1.jpg}

Després de tot l’any esperant,  arribà el dia quatre de maig i amb pluja inclosa vam marxar a Fenals. Des del primer dia que el temps no ens va acompanyar, però això no va resultar cap problema i és que vam realitzar sense cap entrebanc la majoria d’activitats programades.  Colònies de platja passades per aigua, encara que ha de quedar clar que van ser igual o més divertides. Van passar volant i vam gaudir al màxim cada dia que anava passant. Quatre dies que van marcar la unió entre nosaltres; i en definitiva, les colònies de tercer són les millors! Què més dóna que faci fred mentre fem submarinisme, que més dóna que plogui si tenim el pòquer i el pictionary! Si la cosa més important i fonamental de les colònies és la unió que es crea entre nosaltres!

Les nostres últimes colònies a la Solc han estat perfectes; un bon hotel, bona gent i activitats divertides... Ha estat una estada diferent de totes les que hem fet fins ara i és que l’objectiu final d’aquestes era crear un gran vincle d’amistat entre nosaltres i alhora descobrir diferents activitats marítimes, la majoria, per passar quatre dies desconnectats del món exterior.

\authorandplace{Anna Peiró}{3r ESO}

\end{news}

\newssep
\definecolor{color}{rgb}{0.1 , 0.1 , 0.1}

\begin{news}
{2} %columnes
{Les colònies de 1r i 2n d'ESO a Camprodon}
{El dia 4 de maig els nens i les nenes de 1r i 2n d'ESO vam marxar de colònies a Camprodon}
{ESO}
{033} %pagesof

%\noindent\includegraphics[width=9cm,keepaspectratio]{eso/img/JACOB_DAHLGREN.jpg}


El primer dia ens vam dividir: els nens i les nenes de 2n van anar a visitar el monestir de Ripoll, i els de primer vam anar a visitar el Molí i el Monestir de St. Joan de les Abadesses.

Allà ens van explicar que el molí, en el segle VII aproximadament, era molt important, ja que bàsicament menjaven pa. Aquest molí era hidràulic,  funcionava amb la força de l'aigua, per tant estava contruït a prop d'un riu. Com que no sempre el riu baixava amb la mateixa força, desviaven l'aigua per una canal fins a una bassa. Aquesta aigua, quan la necessitaven, la feien baixar per una mena de pendent que feia voltar una pedra enorme que triturava el blat i el convertia en farina, aquesta farina s'havia de dur a les abadesses del monestir.

Després vam anar a visitar el Monestir, que el va fer construir Guifré el Pilós al segle IX. Allà hi vivien les monges i l' abadessa que era la filla de Guifré el Pilós, i es deia Emma.  

Pel fet de ser allà el monestir, es van repoblar les  terres dels voltants.
 El nom que té el monestir té l'origen  que uns segles més tard el Comte de Tallaferro va intentar prendre a l' abadessa Inguilverra el monestir, i com que no ho va aconseguir, va anar a parlar amb el Papa i les va acusar de molts “pecats”, encara que tot i així no es  va sortir amb la seva.

Llavors va fer un tracte amb el Papa : ell es quedava amb el monestir,  però el Papa rebia una part de les riqueses. Aleshores oficialment van fer fora   les monges i  les abadesses amb l'argument que  eren unes Meretrius de Venus. Per aquest motiu el nom que té el monestir és com una mena d'insult. 

Durant la Guerra civil el monestir el van fer servir de polvorí. 

Al cap de molts segles hi va haver terratrèmol que va destruir una gran part del Monestir. 
El van tornar a reconstruir a l'època romànica, incorporant motius decoratius propis de l'art romànic i també del gòtic.

Aquesta visita m`ha semblat molt interessant ja que fins ara, quan veia un monestir, no em podia imaginar que tingués tanta història al seu darrera. Ara imagino les persones, les guerres, els interessos..., i no tan sols un simple convent on vivien monges. Darrera de tota edificació històrica, tal com diu el seu nom: HI HA UNA HISTÒRIA.


\authorandplace{Marta Ortega}{1r ESO}

\end{news}
\definecolor{color}{rgb}{0.1 , 0.1 , 0.1}

\begin{news}
{2} %columnes
{Camprodon' 10}
{}
{ESO}
{033} %pagesof


Quan vaig arribar em van venir al cap molts records, bons i dolents, i és que l'any anterior ja havíem anat a aquella mateixa casa. Els monitors eren els mateixos menys una noia, que era nova. L'Ignasi ens va rebre molt bé. Per començar vam anar a deixar les motxilles a la sala polivalent, després vam anar als porxos a escoltar les normes de la casa. Encara que ja les sabíem de l'any anterior, les vam escoltar molt atents, i després vam tornar  a la sala polivalent a buscar les motxilles per distribuir-nos  les habitacions.

 Jo vaig estar molt bé amb la gent que em va tocar a l'habitació, ja que són persones amb les quals m'hi trobo bé. A nosaltres ens va tocar una habitació que estava al passadís de tots els nens de la nostra classe. Ens van deixar un temps per fer-nos el llit, i després l'Ignasi va xiular amb el seu xiulet. Això volia dir que havíem d'anar tots a la sala polivalent, i així ho vam fer. 

\columntitle{lines}
{Quan vaig arribar em van venir al cap molts records, bons i dolents, i és que l'any anterior ja havíem anat a aquella mateixa casa}

Un cop els 63 vam ser allà dins, l'Ignasi ens va proposar un joc molt divertit: un es tapava els ulls amb un mocador, llavors una persona (tant de primer com de segon d' ESO) sortia i el dels ulls tapats havia d'endevinar qui era aquella persona només tocant-li la cara i els peus. Després vam fer un altre joc: dues persones (normalment un era el més alt de la classe i l'altre el més baixet) havien de portar una patata amb el front fins a l'altra banda de la sala. Quan vam acabar aquests jocs, va arribar l'hora d'anar a sopar. 


El menjar era molt bo. Cada dia parava i desparava taula un grup de gent diferent, perquè així el treball quedava més ben repartit. 

El dimecres al matí, l'Ignasi i els tres monitors ens van explicar el que faríem durant aquell dia. Faríem diferents activitats;  tir amb arc, fer un recorregut amb bicicleta, rocòdrom i jocs d'enginy. A mi la que em va agradar menys va ser el rocòdrom. Aquestes quatre activitats les vam fer durant tot el dia.

{\noindent\includegraphics[width=9cm,keepaspectratio]{eso/img/Camprodon.JPG}}

A la nit no vam poder fer el joc que tenien preparat ja que nevava. Però per substituir-lo ens van posar una pel·lícula. 

Al dia següent va tocar l'excursió per la muntanya. Vam fer 17quilòmetres en total 8 hores caminant. Gràcies a aquesta excursió em vaig adonar que tots tenim més resistència del que ens pensem. Quan vam arribar a la casa, la Davínia i l'Àlex van jugar a pedra, paper o tisora per decidir qui anava primer a dutxar-se, si les noies o els nois, i vam guanyar les noies. 

Les noies vam anar a berenar xocolata calenta amb melindros, i els nois se'n van anar a dutxar. 

Quan tots vam estar dutxats, vam anar a sopar per agafar forces; a la nit hi havia discoteca! Quan va començar la disco em vaig quedar molt sorpresa perquè, després de la caminada que havíem fet,  la gent ballava com boja. Allà va ser on vaig saber com és la gent de veritat. Hi ha gent que em va sorprendre molt, perquè mai no m'hauria imaginat que arribarien a ballar tant. 

La veritat és que jo també em vaig sorprendre  perquè mai havia ballat tant com aquella nit. 

I al dia següent vam tornar cap a casa.

\authorandplace{Andrea Ibáñez}{2n ESO}

\end{news}

\newssep
\definecolor{color}{rgb}{0.1 , 0.1 , 0.1}

\begin{news}
{3} %columnes
{Jornada Esportiva}
{El dissabte dia 15 de maig,  es va fer una jornada esportiva en què es jugaven una sèrie de partits de futbol entre diferents escoles}
{ESO}
{039} %pagesof


La classe de 3r d’ESO, a més, vam organitzar un servei de mini-bar per finançar el viatge de fi de curs  ( quan acabem l'escola) i fer més amè aquell dia. La nostra oferta no era molt extensa però sí de qualitat, ja que els productes els havíem fet nosaltres, intentant que el menjar ajudés a suavitzar i gaudir d’un dia calorós d’escola. També, per facilitar la compra d’aquests productes, vam sortir a la pista a oferir-los . Hi havia : entrepans d'embotit i de formatge, refrescos diversos , cafès i tallats.

El servei de mini-bar el teníem instal.lat a l’entrada del menjador; a més, disposàvem de la cuina per fer els entrepans. Per tal d’organitzar-nos millor, vam fer grups de la classe que havíem de cobrir torns d’una hora i quart. D’aquesta manera, cadascú venia al torn que li tocava.

Des de la classe de 3r d’ESO de l’Escola Solc, us donem les gràcies per la vostra col·laboració i comprensió. 

\authorandplace{Raquel Madrenas}{3r ESO}


\end{news}
\definecolor{color}{rgb}{0.1 , 0.1 , 0.1}

\begin{news}
{3} %columnes
{Sortida a la fundació Miró.  Exposició MURALS}
%index: Fundació Miró: Exposició Murals
{ \noindent\includegraphics[width=9cm,keepaspectratio]{eso/img/JACOB_DAHLGREN.jpg}
\noindent\includegraphics[width=9cm,keepaspectratio]{eso/img/UTR_CREW.jpg}}
{ESO}
{31} %pagesof

Els nois i noies de 2n d’ESO,el dia 9 d’abril, vam anar a la Fundació Joan Miró per visitar l’exposició ‘MURALS’.

En arribar vam esmorzar i després ens van separar en dos grups. Cada grup tenia un guia, el del segon grup es deia David.

Abans de començar,  en David ens va explicar que els primers murals que es van pintar van ser els dibuixos de les coves, a la prehistòria; i que aquests ens han permès estudiar la vida d’aquella època.
També ens va explicar que Joan Miró va pintar murals, i que durant el gòtic el nombre de murals va disminuir a causa de la construcció d’edificis amb grans vitralls.

El primer mural que vàrem veure va ser pintat per unes dones de Diadji Bine Gan de Ga. Allà les dones són les que ho controlen tot i pinten per decorar les seves cases.
Els colors que utilitzen són els colors típics de Marraqueix (grocs, ocres, marrons...), també pinten amb colors vius i ho fan de manera senzilla i detallada.
Pinten els murals amb les mans i els dits,i fabriquen els colors barrejant aigua amb determinats pigments.

El segon mural treballava la geometria i la precisió. A diferència de l’anterior, aquest no era improvisat, cada figura havia estat mesurada i feta utilitzant el regle i el nivell.
El pintor pinta amb una precisió perfecte un tema abstracte.
Cada figura mesura la seva alçada. Amb aquesta obra Lothar Götz, vol representar un record de la seva infantesa.

El tercer mural ens va donar la impressió que allò no era un tipus d’art. Era un mural fet amb plantes (heures). L’artista mexicà només treballa amb plantes com a protesta ecologista.
L’obra s’anomena ‘Contemplant la invasió’, on l’espectador assegut en un banc i a través d’un vidre, contempla la invasió d’heura, que ocupa tot el vidre i que anirà enfilant-se per l’edifici.
L’obra s’havia començat a preparar amb un any d’antelació.

El quart mural interpretava un petit temple budista. L’artista va demanar a la Fundació que pintessin la sala de color vermell i que tapessin el terra amb moqueta vermella. Ell va crear l’obra utilitzant unes plantilles amb forats i pols d’arròs. (Utilitzant una tècnica semblant al puntillisme).

La cinquena obra que ens van ensenyar era un graffiti tradicional. En David ens va explicar que els primera graffitis eren signatures (TAG) i que el pintor és de Singapur, on valoren molt el civisme, per tant si enxampen algú fent un graffiti el penalitzen amb una multa.

El sisè mural era un homenatge al graffiti, cada dibuix representa diferents sentiments, emocions, opinions...
El dibuix d'un puny representa l'esforç de fer un graffiti. Unes càmeres de vigilància dibuixades a un racó representen la vigilància de la ciutat, i el cor al centre és l'emoció i la por que sent l'artista quan pinta un graffiti per la ciutat.

El següent mural era un dibuix fet amb pintura blanca sobre un fons pintat prèviament en negre (com si fos una pissarra). A un costat hi havia representades les pors del pintor i a l'altra banda dibuixos de platets voladors i abduccions alienígenes, una por molt americana.

A continuació Núria i Tono. Són els artistes del vuitè mural.
Representen el significat del graffiti.
Els seus graffitis s'integren a les superfícies on els pinten. Aquests dos artistes han creat també graffitis lumínics. Pinten de forma legal, amb permís. I normalment la gent els conserva, els graffitis. El seu treball dins l'exposició representa figures geomètriques pintades amb pinzell amb diferents gradacions de color.

L'avantpenúltim mural és un paisatge tan en el fons com en el primer pal. Una dent de lleó. L'artista l'ha dibuixat en blanc i negre amb la intenció  que cadascú mentalment li posi els colors que vulgui.

El desè mural és una cobra de collage realitzada per Ludovica Gioscia. Per fer aquesta obra , l'artista va haver de calcular-ho tot. La mida dels papers, roba, colors, ...
És un treball molt elaborat i fet de capes.

La darrera obra estava formada per dianes penjades que omplien tot el pany de paret.
L'artista ha volgut jugar amb la norma de no poder tocar els murals. El fet de posar-hi dianes ens convida a atacar l'obra d'art amb dards.

La meva opinió sobre aquesta exposició és que és molt original, ja que sovint no trobem que el graffiti sigui entès com un tipus d'art dins d'un museu.

							Davínia Vilagrassa 2n ESO


\end{news}
\definecolor{color}{cmyk}{0.5, 0, 1, 0.5}

\begin{news}
{3} %columnes
{Viatge 4T D’ESO 2010}
%index: 4t d'ESO a Polònia
{
\noindent\includegraphics[width=18cm,keepaspectratio]{eso/img/polonia_P5050116.JPG}
}
{ESO}
{036} %pagesof


Diumenge 2 de maig al matí, per no dir la matinada, ens trobàvem arrossegant les maletes i els nervis per l’aeroport del Prat

%\noindent\includegraphics[width=9cm,keepaspectratio]{eso/img/polonia_P5050106.JPG}
A poc a poc deixàvem enrere la bella Barcelona i les no tan belles matemàtiques, literatures i d’altres, i fèiem camí cap a Polònia; però abans vam parar a Düsseldorf, una ciutat alemanya amb un aeroport preciós on ens vam acomiadar dels euros per quatre dies. Havent dinat vam embarcar per segona vegada, ara ja sí cap a Cracòvia, una ciutat grisa i castigada encara que preciosa.

Era dilluns però no ens feia mandra llevar-nos a quarts de vuit; no teníem per endavant un dia laboral ni d’escola, sinó una visita a la ciutat on ens allotjàvem: el centre, Kazimierz (el barri jueu), l’antic Ghetto, el pujolet de Babel (que no tenia res a veure amb la mítica Torre de Babel) on per sobre la ciutat regnava la Catedral, la Universitat on es formà Copèrnic, etc. Vam dinar pel centre, uns asseguraven un bon àpat a alguna pizzeria (el que fa la globalització) i d’altres s’atrevien amb la gastronomia de la zona. 

A la tarda vàrem visitar les mines de sal de Wieliczka, a 15km de Cracòvia, on hi vam arribar amb l’autocar. Vam fer un recorregut que ens portava per diferents cambres, fins i tot hi havia una capella, la de Sta.Kinga, feta de la mateixa sal de la mina que els propis miners havien construït.

Eren quarts de vuit quan arribàvem a l’hotel, havíem de ser puntuals, el sopar se servia a les vuit. 

L’endemà al matí l’autocar ens duia fins a Zakopane, la capital hivernal de Polònia, un lloc molt turístic però tot i això força rural. Pràcticament vam passar el dia  allà, vam aprofitar per comprar alguns souvenirs i la majoria vam tastar els famosos formatges fumats de la zona, molts ens vam sorprendre, no eren deliciosos com ens els havien  pintat però eren força bons. 

Dimecres 5 ens vam llevar més d’hora, vam canviar la molla Cracòvia per la freda Auschwitz. Una visita guiada ens portava pels blocs i carrers de la ciutat d’extermini nazi, passant pels crematoris que encara continuaven drets i les cambres de gas. 

Vam estar a Auschwitz I i a Auschwitz II-Birkenau, de 90 i 147 hectàrees respectivament. 

Vam tornar a Cracòvia a l’hora de dinar i a la tarda vam estar passejant i fent les últimes compres, els últims berenars o cafès amb zlotis. 

De sobte, es va fer dijous i havíem de passar altre cop per l’aeroport de Düsseldorf i acabar de gastar les últimes monedes poloneses; així que vam tornar als entrepans de l’avió i a les revistes alemanyes indesxifrables. A quarts de vuit del vespre arribàvem a Barcelona deixant enrere Polònia, els seus carrers i places, la fredor general de la gent, la delicada i valenta Cracòvia, i el nostre primer i últim viatge com a classe de 4t d’ESO. 


\authorandplace{Júlia Oliver}{4t d’ESO}

\end{news}
\definecolor{color}{cmyk}{0.5, 0, 1, 0.5}

\begin{news}
{2} %columnes
{Cracòvia 2010}
{}
{ESO}
{035} %pagesof

\noindent\includegraphics[width=9cm,keepaspectratio]{eso/img/polonia_P5030075.JPG}

Tots sabíem on anàvem. Si més no, ho havíem decidit nosaltres. Polska, Kraków, Polònia, Cracòvia. Aquesta era la nostra destinació, un món ple d’homes i de dones rossos amb els ulls blaus, un món plujós, poc acostumat a veure un dia totalment assolellat i de viure les caloroses temperatures mediterrànies, un idioma impossible d’entendre, una moneda estranya. Sabíem que hauríem de fer quatre hores d’avió més l’espera a l’aeroport, i també que valdria la pena perquè ens esperava un viatge inoblidable. 

Vam arribar a la nostra destinació, cansats però amb ganes d’instal·lar-nos a l’hotel i començar a viure aquells cinc dies del viatge de fi de curs amb intensitat.

Durant aquests cinc dies, vam passejar per els carrers humits de Cracòvia, vam observar encuriosits els racons més especials i bonics d’aquella petita ciutat, vam provar el menjar típic, com per exemple el formatge fumat, per alguns molt bo, per altres no tant. També visitàrem Zakopane un petit poble hivernal, amb unes vistes precioses. Vam viure l’experiència esgarrifant de endinsar-nos en el Camp de Concentració d’Auschwitz, visitar unes mines de sal sota terra, on vam quedar totalment bocabadats al veure que hi havia escultures totalment de sal a 135 m de profunditat. 


Tampoc podrem oblidar les nits perdudes pel barri jueu on es trobava el nostre hotel, anant a un parc d’atraccions o simplement passejant, i tornant cap a l’hotel, pels carrers que dia a dia ja s’anaven fent nostres, mentre que la pluja ens mullava (factor que ens va acompanyar durant tot el viatge, com era previst), però ens era igual, perquè l’important era que estàvem junts, i volíem aprofitar tots els minuts d’aquells dies per fer d’aquell viatge que tancaria el nostre cicle a l’Escola Solc, un record inesborrable.

\authorandplace{Rosa Gras}{4t ESO}

\end{news}

\newssep
\definecolor{color}{cmyk}{0.5, 0, 1, 0.5}

\begin{news}
{3} %columnes
{Aprenem a votar}
{La nostra escola ha passat a formar part del col·lectiu de 40 escoles i instituts que a tot Catalunya portaran a terme l’experiència d’innovació pedagògica “Aprenem a votar”. 
}
{ESO}
{33} %pagesof

\noindent\includegraphics[width=5.5cm,keepaspectratio]{eso/img/aprenem_votar.jpg}

El programa es desenvoluparà durant el primer trimestre del curs 2010-2011 i té com a finalitat que l’alumnat de 4t d’ESO aprengui a exercir lliurement el seu dret al vot de cara al futur, a partir d’un procés d’informació de la mecànica electoral i d’una simulació de les eleccions.

El projecte està promogut pel Departament de Didàctica de les Ciències Socials i té el suport del Departament d’Educació de la Generalitat de Catalunya.


\bf Estructura del projecte  

\rm
El projecte es divideix en cinc fases:

Unitat 1. Elegim. Es tractarà de conèixer els diferents elements del sistema democràtic català, i del Parlament en particular, i descriure’n el funcionament general.

Unitat 2. Ens informem. Es tractarà de reconèixer i utilitzar de manera crítica les diverses fonts d’informació que es posaran a disposició dels nois i noies durant la campanya electoral.

Unitat 3. Opinem.  L’alumnat podrà opinar sobre els diferents partits i coalicions catalans, tenint en compte les preocupacions i les opcions que presentaran durant les eleccions.

Unitat 4. Participem. Incrementarà la informació sobre les diferents maneres de participar en un sistema democràtic.


Unitat 5. Votem. Es tractarà d’aprendre a organitzar una votació tot respectant els criteris que fan possible la celebració d’unes eleccions lliures i democràtiques. L’alumnat participarà en una simulació el divendres anterior al dia de les eleccions al Parlament de Catalunya. 


En una segona part d’aquest programa, es reflexionarà sobre els resultats oficials de les eleccions al Parlament en comparació amb el vot estudiantil de quart d’ESO

Departament de Ciències Socials – Educació Secundària Obligatòria

\end{news}
\definecolor{color}{cmyk}{0.5, 0, 1, 0.5}

\begin{news}
{2} %columnes
{VideoTrams}
{El grup de guionistes, format per quatre persones i reforçat per part de l'equip de direcció, van desenvolupar una història basada en el tema principal triat per la classe:“Embarassos a l'adolescència” }
{ESO}
{39} %pagesof


Tik, Tok, s'ha acabat el temps, el nostre vídeo ja està enviat a l'altra escola (és un treball que fem conjuntament amb les escoles de TRAMS). Ha estat un treball en grup complicat que alhora hem sabut tirar endavant. Aquest vídeo ha estat un treball transversal, ja que en fer-lo  hem treballat temes de les àrees de tutoria i ciutadania, d'educació audiovisual, d'informàtica i música. 

En els primers moments,  vam dividir-nos en grups segons la funció que volíem realitzar en el videotrams. El grup de guionistes, format per quatre persones i reforçat per part de l'equip de direcció, van desenvolupar una història basada en el tema principal triat per la classe: “Embarassos a l'adolescència”. Un cop exercida la feina dels guionistes, van començar els càstings per triar els actors i les actrius que portarien a terme els papers principals de la història.



Els guionistes, juntament amb l'equip directiu,  van donar el seu punt de vista sobre la història perquè els actors i actrius l'enfoquessin com ells creien. Després d'assajos, va arribar l'hora que el grup de decorats pensés com poder adaptar les diferents aules de l'escola d'acord amb l'escenari. Un cop tot adaptat i amb el guió après va ser l'hora de posar-se a gravar. 

\noindent\includegraphics[width=9cm,keepaspectratio]{eso/img/videotrams_3.jpg}

Els càmeres se les van empescar per introduir diferents plans  al llarg de tota la història.  Quan va estar gravat, l'equip d'edició de vídeo amb la col·laboració de la Lluïsa  van editar-lo i enviar-lo a l'escola corresponent. El resultat? Un vídeo que mostra un treball excepcional de grup en una activitat més lúdica  del que és habitual.

\authorandplace{Esteve Pérez i Ariadna Peiró}{3r d'ESO}

\end{news}

\newssep
\definecolor{color}{cmyk}{0.5, 0, 1, 0.5}

\begin{news}
{3} %columnes
{Concert a l'Escola Sant Gervasi}
{Els nois i noies de 1r d’ESO hem anat a l’escola Sant Gervasi de Mollet del Vallès a fer un concert conjuntament amb els de 1r d’ESO d’aquesta escola}
{ESO}
{60} %pagesof


Quan vàrem arribar, els nois i noies de l’escola Sant Gervasi ens van donar la benvinguda i vàrem esmorzar junts al seu pati.

Després d’esmorzar vàrem anar a la sala d’actes, on cada grup va assajar la seva part per al concert.

Més tard varen venir nens i nenes de 6è de primària, per poder escoltar i participar del concert.

Un cop acabat el concert, vàrem anar a dinar junts i a jugar una estona. Aquesta estona junts va servir per poder-nos conèixer i apuntar-nos els números de telèfon, l’adreça del Facebook i correus electrònics.

\authorandplace{}{ 1er d'ESO }

\noindent\includegraphics[width=12cm,keepaspectratio]{eso/img/sant_gervasi5.jpg}


\end{news}
\definecolor{color}{cmyk}{0.5, 0, 1, 0.5}

\begin{news}
{2} %columnes
{Sant Jordi}
{\noindent\includegraphics[width=18cm,keepaspectratio]{eso/img/st_jordi_bates.jpg}}
{ESO}
{500} %pagesof

{\noindent\includegraphics[width=9cm,keepaspectratio]{eso/img/st_jordi_taulell.jpg}}

El dia de St.Jordi,  l'alumnat de 3r d'ESO ens vam encarregar de preparar-ho tot.
Alguns alumnes van venir una hora abans per muntar les parades abans que arribessin els pares per comprar alguns llibres. Al cap d'una hora van arribar els altres alumnes, i un grup de nosaltres se'n va anar a muntar les activitats per als més petits. 
A la parada, van anar venint tots els alumnes, primer els més petits i després els més grans, per comprar-nos molts llibres i polseres.

Els nens de parvulari s'ho van passar molt bé picant les pinyates que els vam preparar. Els vam tapar els ulls i els vam donar un pal perquè les trenquessin. 

{\noindent\includegraphics[width=9cm,keepaspectratio]{eso/img/st_jordi_pinguins.jpg}}

%{\noindent\includegraphics[width=9cm,keepaspectratio]{eso/img/st_jordi_carablanca.jpg}}

%{\noindent\includegraphics[width=9cm,keepaspectratio]{eso/img/st_jordi_taulell.jpg}}


Els nois i noies de 2n de primària també es van divertir molt amb les pinyates, però també fent la prova dels “filipinos”, que consisteix en passar un fil per un “filipino” i anar menjant el fil fins arribar al “filipino” i menjar-se'l.
Els alumnes de 3r i 4t de primària van fer molts jocs variats com el dels “filipinos”, llençar anelles a un pal, buscar un caramel amb la boca dins d'un bol amb farina, portar ous en una cullera,...

Volem agrair a tot l'alumnat, al professorat i a tots els pares i mares que van contribuir , comprant llibres,  al nostre futur viatge de 4t d’ESO   

\authorandplace{Artur Martínez}{3r ESO}
\end{news}



\end{document}
