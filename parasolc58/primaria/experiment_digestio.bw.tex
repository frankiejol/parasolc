%%%%%%%%%%%%%%%%%%%%%%%%%%%%%%%%%%%%%%%%%%%%%%%%%%%%%%%%%%%%%%%
%
% Arxiu pregenerat.
% NO EDITAR
%
%%%%%%%%%%%%%%%%%%%%%%%%%%%%%%%%%%%%%%%%%%%%%%%%%%%%%%%%%%%%%%%
%doc: 2a tramesa revista/Revista_5_1r_trim/Experiment digestio/Experiment_digestio.doc
\begin{news}
{2} %columnes
{EXPERIMENT SOBRE LA DIGESTIÓ}
{En acabar el primer tema de Medi Natural, “Els éssers vius”, hem fet un experiment on hem après els tres processos de la digestió. Ara us explicaré què vam fer i el  material que vam utilitzar}
{Primaria}
{34}

\section{Digestió bucal}

	\begin{itemize}
		\item Menjar: pa
		\item Boca: bol
		\item Dents: dits
		\item Llengua: cullera
		\item Saliva: aigua
	\end{itemize}

	\begin{description}
	\item[a] Masticació: desfem el menjar (esmicolem el pa amb els dits).
	\item[b] Salivació: humitegem el menjar (afegim aigua i remenem amb la cullera).
	\item[c] Deglució:empassem el menjar.
	\end{description}

\emph{Formació bol alimentari}


\section{Digestió estomacal}

	\begin{itemize}
        \item Menjar: bol alimentari.
        \item Estómac:bol.
        \item Moviments estomacals: cullera.
        \item Àcids gàstrics: vinagre.
	\end{itemize}

	Els àcids gàstrics(vinagre) desfan encara més el bol alimentari.

	\emph{Formació del quim.}



\section{Digestió intestinal}

	\begin{itemize}
		\item Menjar: quim.
		\item Intestí: paper absorbent.
	\end{itemize}

L’intestí absorbeix les substàncies nutritives del quim a través dels vasos sanguinis. Aquet procés s’anomena absorció intestinal.

	\emph{Formació de la matèria fecal}

\end{news}
