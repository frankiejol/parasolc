%doc: 2a tramesa revista/Revista_5_1r_trim/Santa Fe del Montseny/Montseny.doc

\begin{news}
{2} %columnes
{Sortida a Santa Fe del Montseny}
%index: Santa Fe del Montseny
{El dimarts 26 d'octubre,  els nens i nenes de 5è vam anar a Santa Fe del Montseny}
{Primaria}
{306}

\noindent\fbox{\includegraphics[width=8cm,keepaspectratio]{primaria/img/dsc00900.jpg}}


 Quan vam arribar vam veure uns arbres supergrans que es diuen sequoies. Ens agafàvem sis nens de la mà i envoltàvem tot el tronc, era impressionant. Més tard vam veure un audiovisual de com és el Montseny a les quatre estacions de l'any.

 Tenia efectes especials i era superdivertit. Més tard vam començar a caminar cap a l'embassament de Santa Fe. Pel camí vam fer qutre parades. En una de les quatre vam calcar la fulla i l'escorça del faig i, en una altra parada vam fer el mateix amb el castanyer. Durant el camí també  vam agafar castanyes. Després de dinar el Ferran ens va deixar fer fotos. Quan tornàvem,  ens van explicar que el boix grèvol per  baix té les fulles amb punxes i per dalt les té sense punxes. Quan vam arribar a la plana vam pujar a l'autocar. Ens esparava hora i mitja de viatge de tornada.


Aquesta sortida va ser molt divertida.


\authorandplace{Laia Lloreda}
							{5è de primària}

\end{news}

