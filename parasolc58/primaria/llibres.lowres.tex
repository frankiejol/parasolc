%%%%%%%%%%%%%%%%%%%%%%%%%%%%%%%%%%%%%%%%%%%%%%%%%%%%%%%%%%%%%%%
%
% Arxiu pregenerat.
% NO EDITAR
%
%%%%%%%%%%%%%%%%%%%%%%%%%%%%%%%%%%%%%%%%%%%%%%%%%%%%%%%%%%%%%%%
\begin{shortnews}
{3} %columnes
{Ressenyes de llibres}
{}
{Primaria}

\shortnewsitem{La marmota inventora}
{
Autor: Enric Larreula / Rita Culla.   Editorial: La Galera 

Hi havia una vegada una marmota molt llesta que un dia va fer un invent, una bufanda. La va tallar a trossets i la va regalar a totes les marmotes. La bufanda la va fer servir per dormir. La marmota vivia en unes muntanyes molt altes anomenades els Alps.

A mi m’ha agradat molt perquè tracta d’invents i de marmotes.
									
Biel Castro, 1r de Primària
}

\shortnewsitem{La campaneta de plata}
{
Autor: Conte popular.   Editorial: Susaeta

En una torre hi vivia la mare dels vents, que tenia tres fills: el vent del Nord, la Brisa i el vent de Llevant. Quan es van fer grans van viatjar per diferents mons, i van deixar una campaneta de plata a la mare per si els volia cridar.  Com que la mare es sentia sola, va tocar la campaneta per reunir els tres germans, però de tant vent, el poble es pensava que a la torre hi vivia una bruixa.

M’ha agradat aquest llibre perquè la mare reuneix els seus fills tocant la campaneta.
									
Aida Perez, 1r de Primària
}

\shortnewsitem{Pinotxo}
{
Autor: Carlo Collodi      

Un dia un fuster anomenat Gepetto va construir un titella, que va anomenar  “Pinotxo”. Va desitjar que  ell tingués vida. A la nit va venir una fada i va fer que el titella cobrés vida. Pinotxo es va sorprendre molt perquè podia parlar, caminar, menjar, anar a l’escola, etc. Pepito Grillo era la consciència de la marioneta, que el va ajudar a ser conscient del que feia. 

Em va agradar molt perquè la fada va fer màgia i a mi això m’agrada molt. Aquest llibre el recomano perquè és molt divertit, emocionant i hi passen moltes aventures.

Pol Hernández,  2n de Primària
}

\shortnewsitem{El mag Merlí}
{
Autor: Llegenda Popular.    Editorial: Susaeta  

Un nen anomenat Grill vivia en un castell a Alemanya. Un dia en Grill amb un amic que es diu Key van anar a caçar, però com que en Grill era novell  en la caça, va caure des d’un arbre a sobre d’en Key fent que llancés una fletxa al cel; en veure el que havia fet, en Grill li va dir que aniria a cercar la fletxa. En Grill anava cercant la fletxa quan, de sobte, va veure la fletxa en una branca, l’anava a agafar,  va relliscar i va caure en una cabana on hi vivia en Mag Merlí.  Li va dir que havia de  tornar al castell...    

A mi el que més m’ha agradat és quan llençava la fletxa. I,quan estan caçant, hi ha un dibuix que em va agradar molt.

Damià Rubió ,  2n de Primària
}

\shortnewsitem{La Xola i els lleons}
{
Autor: M. Dolors Alibés.   Editorial: Cruïlla

És una gossa que es diu Xola i que es pensa que es un lleó, llavors l’imita. El seu amo,  en Marc, té un amic que investiga la selva, llavors l’amic porta un llibre de lleons a casa la Xola i se’l deixa. La gosseta, com que està tan interessada en els lleons, l’agafa i el llegeix i descobreix que els lleons no són tan fantàstics com ella creia i ja no vol ser un lleó. I ara és una gosseta petonera com abans era.

M’ha agradat molt perquè és molt misteriós i no saps què passarà. També perquè mai  pots decidir ser un lleó si no saps res sobre ells.

Andrea Amador Alcaina,  2n de Primària
}

\shortnewsitem{Geronimo Stilton. Les entranyes de les rates pudents}
{
Autor: Geronimo Stilton.   Editorial: Destino

Aquest llibre parla que Ratalona comença a fer molta pudor. Una nit, de sobte,  de les clavegueres surten globus pudents. En Gerónimo baixa a les clavegueres per esbrinar què feia tanta pudor. Caminant per les clavegueres, arriba a Ratcity, la ciutat de les rates. Hi ha una rata que es diu Clavegueram que és una mica dolenta, i quan arriba a Ratcity s’enamora del Xafarot, l’amic d’en Geronimo i es vol casar amb ell. En Xafarot i en Geronimo agafen una moto d’aigua i s’escapen.

Ens ha agradat molt perquè aquest llibre és de misteri. També ens agraden molt les aventures d’en Gerónimo Stilton.

Eloi Banyes, Ferran Corral i Roger Guarro, 3r de Primària
}


\shortnewsitem{La Tanga i el gran lleopard}
{
Autor: Roberto Malo i Fco Javier Mateos.  Editorial: Comanegra

La Tanga és una noia que viu en un poblat al cor de la selva. En el poblat hi ha un mag que es diu el Gran Bruixot i que té poders màgics. Un bon dia, un lleopard ferotge i cruel va menjant-se tots els animals del poblat. A partir d’aquell succés van començar a patir fam, i el Gran Bruixot va intentar aturar e lleopard, però no va funcionar. A la nit es va posar una màscara per transmetre un missatge als ciutadans. I va dir a tot el poblat que el dia següent tothom que volgués enfrontar-se al lleopard es presentés, però no va aparèixer ningú, excepte la Tanga. Llavors el Gran Bruixot li va donar un ganivet de fusta de banús... 

Ens ha agradat molt perquè aquest llibre és una aventura molt divertida i perquè ens agrada molt la màgia i el Gran Bruixot en fa molta.

 Pablo de Quadras i Eduard Tenas,  3r de Primària
}


\shortnewsitem{El zoo d’en Pitus}
{
Autor: 	Sebastià Sorribas.  Editorial: La Galera


El Pitus té una malaltia. Només el pot curar un metge molt famós de Suècia. Els seus amics, el Tanet, el Manelitus, en Cigró, en Juli, en Fleming i la Mariona van tenir una gran pensada: com que a en Pitus li agradaven molt els animals, van decidir fer un zoo per a ell.

M’ha agradat molt l’argument d’aquest  llibre.

Roser Pérez, 4t de Primària
}

\shortnewsitem{Tina superbruixa. L’aniversari d’en Pitus}
{
Autor: Knister   Editorial: Bruixola


Aquesta és una de les emocionants històries de la Tina. És l’aniversari d’en Pitus i està molt emocionat i vol fer una súper festa, però el seu pare està de viatge i la seva mare en un curs. Ve la seva tieta, que és una mica bleda  amb els nens. A la Tina i al Pitus els tracta com a nadons. Quan arriben els convidats, la tieta Elisa es queixa i es queixa fins que la Tina fa un dels seus encanteris i... Plaf! La tieta s’adorm.

M’ha agradat perquè és molt divertit, és una mica curt, però val la pena. És entretingut i el recomano a tothom.

Ariadna Vera, 4t de Primària
}

\shortnewsitem{Els rebels de la cabana}
{
Autor: David Nel·lo.   Editorial: Cruïlla

Va d’un grup de nens que tenen feta una cabana a dalt d’un arbre i l’alcalde vol construir un aparcament al Prat, tallaran l’arbre on hi ha la cabana. La Margot, que és la que mana, vol negociar amb ell perquè no tallin l’arbre, però l’alcalde diu que no. Aleshores, els nens van dir que si no negociaven, no baixarien de la cabana. 

És molt divertit perquè els nens són molt atrevits ja que es queden una nit de pluja a la cabana. Jo el recomano perquè és molt emocionant.

Aina Boronat, 5è de Primària
}

\shortnewsitem{La tribu de Camelot}
{
Autora: Gemma Lienes.    Editoria: Empúries

El Papagueno, el canari de la veïna de la Carlota ha desaparegut i la tribu passa moltes aventures amb un llibre de màgia que els ajuda a retornar en Papagueno a la veïna de la Carlota.  

Aquest  llibre m’ha agradat molt perquè hi ha moltes aventures i és molt intrigant per saber qui ha raptat Papagueno. M’entretenia a trobar els dracs que hi havia amagats a moltes pàgines, també eren molt divertides les olors i les tintes.  
Paula Claver, 5è de Primària
}

\shortnewsitem{Pocahontas}
{
Autor: Walt Disney.   Editorial: Beascoa

La història comença quan va arribar una expedició de navegants a la terra dels indis.
L’objectiu era trobar or i enriquir-se. Casualment, en John coneix la Pocahontas i se n’enamora. Al pare de la Pocahontas  això no li agrada ja que volia que es casés amb en Kocum. En una baralla per la Pocahontas, mor en Kocum i fan presoner en John. Els amics d’en John corren a salvar-lo amb les escopetes...

És una història d’amor i d’aventures i a mi m’ha agradat molt!

Eva Llovera, 5è de Primària.
}


\shortnewsitem{La meravellosa medicina d’en Jordi}
{
Autor: Roald Dahl.   Editorial: Empúries

En Jordi és un noi que  sempre ha d’estar cuidant la seva àvia i preparant-li els medicaments. Un dia, per experimentar,  li va preparar  una meravellosa medicina, quan l’àvia se la va prendre, es va fer molt alta i sobresortia pel terrat de la casa. Al seu pare li va agradar aquest experiment per donar-lo als animals, així serien més grans i tindrien més carn i els aprofitaria més. Va donar el medicament als animals i tots es van tornar  més grans. Al cap d’un temps, en va fer un altre i el va donar a l’àvia, llavors l’àvia es va tornar molt ...  

Aquest llibre és, al principi, una mica avorrit, fins que  comencen els experiments, però a mi el que més m’ha agradat són les coses que li passen a l’àvia.

Judit Molina, 6è de Primària
}

\shortnewsitem{Ales de foc}
{
Autor:Christopher Pike. Editorial: Edicions  agrupo  Z .

Ariel és un àngel que té una “protegida”que es diu Marla. Ariel acaba traïda per Marla, i és tancada a Gorilan, una presó “màgica”. Allà hi ha un rei que és un gripau molt intel·ligent. A Gorlian coneix un noi anomenat Brad, que acaba mort per l’exèrcit del rei gripau. Aleshores acaba sent rescatada per uns amics i viuen diferents aventures.

A mi aquest llibre m’ha agradat molt perquè és d’aventures i d’éssers màgics.

Dídac Benages , 6è de Primària
}

\shortnewsitem{Malsons}
{
Autor: Anne Fine. Editorial: Bromera.

Una nena anomenada Imogen canvia d’ escola, es fa  molt amiga de la Melanie. A les altres escoles que ha estat no tenia gaires amigues. La Melanie descobreix que Imogen té un secret, ple de màgia, de por, de fetilleria i ple d’encanteris. Aquest secret que li succeeix a Imogen serà bastant difícil de descobrir, només la Melanie ho podrà fer!

Aquest llibre, a mi m’ha agradat molt ja que és de misteri, d’aventures i a vegades semblen fets reals!!

Mariona Medrano, 6è de Primària
}

\end{shortnews}
