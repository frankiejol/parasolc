%doc: 2a tramesa revista/Revista_5_1r_trim/Experiment digestio/Experiment_digestio.doc
\begin{news}
{2} %columnes
{Experiment sobre la digestió}
{}
{Primaria}
{234}

En acabar el primer tema de Medi Natural, “Els éssers vius”, hem fet un experiment on hem après els tres processos de la digestió. Ara us explicaré què vam fer i el  material que vam utilitzar
\paragraph{ 1 Digestió bucal}

		Menjar: pa , Boca: bol , Dents: dits ,  Llengua: cullera , Saliva: aigua

	\begin{description}
	\item[a] Masticació: desfem el menjar (esmicolem el pa amb els dits).
	\item[b] Salivació: humitegem el menjar (afegim aigua i remenem amb la cullera).
	\item[c] Deglució:empassem el menjar.
	\end{description}

\emph{Formació bol alimentari}


\paragraph{ 2 Digestió estomacal}

        Menjar: bol alimentari , Estómac:bol , Moviments estomacals: cullera , 
             Àcids gàstrics: vinagre

	Els àcids gàstrics(vinagre) desfan encara més el bol alimentari.

	\emph{Formació del quim.}



\paragraph{ 3 Digestió intestinal}

		Menjar: quim. ,  Intestí: paper absorbent

L’intestí absorbeix les substàncies nutritives del quim a través dels vasos sanguinis. Aquet procés s’anomena absorció intestinal.

	\emph{Formació de la matèria fecal}

\end{news}
