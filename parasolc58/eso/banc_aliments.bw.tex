%%%%%%%%%%%%%%%%%%%%%%%%%%%%%%%%%%%%%%%%%%%%%%%%%%%%%%%%%%%%%%%
%
% Arxiu pregenerat.
% NO EDITAR
%
%%%%%%%%%%%%%%%%%%%%%%%%%%%%%%%%%%%%%%%%%%%%%%%%%%%%%%%%%%%%%%%
%doc: Revista 3/Banc dels aliments - Arnau Bilbao (3r d'ESO).docx

\begin{news}
{2} %columnes
{Xerrada sobre el Banc dels Aliments}
{El divendres 12 de novembre,  un dels 92 voluntaris del banc dels aliments va venir a l’escola Solc per fer-nos una xerrada als nois i noies de 3r d’ESO sobre el Banc dels Aliments}
{ESO}
{13} %pagesof


Va ser una xerrada molt instructiva i alhora molt divertida ja que el senyor que va venir ens va fer la xerrada d’una manera diferent a altres xerrades que havia escoltat jo anteriorment. Ens va explicar moltes coses: què era el banc dels aliments, de què s’encarregava, què o qui el formaven...

Segons ens va explicar el voluntari, el banc dels aliments és una entitat que s’encarrega de recollir aliments per tota Catalunya, en instituts i escoles que col·laboren en la recollida d’aliments que es produeix un cop cada any durant una setmana. També recullen excedents de marques de menjar (Tarradellas, Ferrero Rocher, etc...). Tot aquest menjar s’aconsegueix gratis. Un cop tenen el menjar el guarden uns dies fins que gent pobra d’aquí, a Catalunya, en necessita per alimentar-se,  ja que no tenen prou diners per aconseguir-ne. El menjar es reparteix amb 4 furgonetes que tenen entre tots els voluntaris. A vegades al Banc dels Aliments els falta algun tipus determinat de menjar, llavors fan una mena de subhasta per comprovar quina marca d’aquell aliment els el donen més barat i llavors el compren. En el Banc dels Aliments no s’accepten ni begudes amb alcohol ni queviures quasi caducats;  tot i així aquests aliments sí que els accepten perquè no els hagin de portar a l’abocador.

Tot això ens  ho va explicar, també,  amb un “powerpoint”. En el “powerpoint” s’hi explicava què era el banc dels aliments (bé, ens explicava el mateix que el voluntari). Ens vam divertir molt en aquesta xerrada perquè el voluntari era molt divertit i quan algú feia alguna pregunta interessant li donava un caramel! A part d’això també va ser una xerrada molt instructiva.

Ens ho vam passar d’allò més bé fent feina normal i corrent.

També ens va dir que, si volíem, podíem anar al Banc dels Aliments a col·laborar. Un company de la nostra classe ho va fer fa pocs dies, i tant ell com jo us animem a presentar-vos, si voleu, com a voluntaris del Banc dels Aliments, una cosa que s’agrairia molt ja que allà tenen molta feina per fer i poca gent per fer-la. Podeu trobar més informació a www.bancdelsaliments.org.

\authorandplace{Arnau Bilbao}{3r d’ESO}

\end{news}
