
%doc: 2a tramesa revista/SORTIDA PIPPILOTI RIST/Pipilotti Rist- Ccalvo i Mromero.doc

La Pipilotti Rist és una artista que treballa sobretot amb vídeo, d’aquesta manera ella vol intentar que entris en les seves obres art, cosa que no es pot fer amb els quadres. 


Vam entrar  a quatre sales. En la primera sala, que es diu “ Xarrupa el meu oceà” parla del desig profund d’entendre’ns els uns amb els altres. En aquesta sala l’artista juga amb dues pantalles fent cantonada i  les imatges que surten alhora, surten simètriques l’una i altra.  Aquest vídeo és una queixa personal per part de l’artista, que diu que està sola  ja que, a l’ haver-hi dues pantalles, ens vol dir que aquell univers està format per dues persones, però només hi és  ella.

La segona sala consta de dues obres. Una obra està formada  per una làpida, unes fulles i un  vídeo d’ella mateixa, d’aquesta manera ens vol fer veure que solament el nom de la persona a la làpida no diu res, que seria millor que hi hagués un vídeo de la vida de la persona morta. La segona obra està formada per un vídeo de dos nens petits jugant en una piscina buida, d’aquesta manera vol relacionar la fragilitat de l’estructura òssia dels nens petits amb la fragilitat del món actual, tant en la política com en l’economia...

La tercera sala està formada per una sala blanca amb una cuina amb armaris que arriben fins al sostre, fent de pantalla per la projecció d’un vídeo en el qual surt ella sota la pluja, i  la pluja representa el seu plor en sentir-se sola a la cuina treballant pels altres.

I la quarta sala està formada per tres pantalles, dues paral·leles perpendiculars a una altra de frontal. Aquesta obra s’anomena “L’òvul pulmonar” I tracta de la nostra autocensura diària. Això l’artista ho aconsegueix amb la seva imitació cap a la natura. És a dir,  en una pantalla es veu un porc senglar i a l’altra se la veu a ella imitant-lo.


Generalment les seves obres parlen d’ella mateixa i de la seva posició com a dona.

\authorandplace{Clàudia Calvo i Marina Romero}
{3r d’ESO}

