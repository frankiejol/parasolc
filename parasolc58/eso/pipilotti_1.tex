%doc: 2a tramesa revista/SORTIDA PIPPILOTI RIST/pipilotti.doc

Quan vam arribar, vàrem deixar les motxilles en unes  taquilles i vam començar la visita.

Vam veure diferents vídeos. Primer vam entrar en una sala i ens vam estirar a terra. L'autora explicava els seus sentiments d'amor; es veia el fons del mar i se sentia una cançó d'amor  que cantava ella.

Després vam anar a una altra sala, en aquesta ens havíem d'estirar sobre uns coixins  perquè el vídeo es projectava al sostre.

Hi havia una altra obra que era una escultura, a dins d'una de les peces hi havia el projector. Representava un mòbil, en el qual hi havia una bola de metall a un costat i , a l'altre,  una peça de roba que era una gota.

Seguidament, vam veure un vídeo que estava projectat per tres càmeres on es veien imatges diferents, però que estaven relacionades entre elles.

Per acabar hi havia una sala que representava una cuina on es veia projectat un vídeo que explicava els sentiments d'una dona maltractada.

En acabar la visita, vàrem anar a dinar a fora. Després vam tornar a l'escola.

Aquesta sortida ens va agradar molt a tots, ja que era un tipus d'art més actual, que nosaltres tenim més present.

\authorandplace{Marta Ortega
Marina Jiménez}{ESO}
