%%%%%%%%%%%%%%%%%%%%%%%%%%%%%%%%%%%%%%%%%%%%%%%%%%%%%%%%%%%%%%%
%
% Arxiu pregenerat.
% NO EDITAR
%
%%%%%%%%%%%%%%%%%%%%%%%%%%%%%%%%%%%%%%%%%%%%%%%%%%%%%%%%%%%%%%%
%doc: Revista 3/projecte lector.doc
\begin{news}
{2} %columnes
{Projecte Apadrinament Lector}
{}
{ESO}
{16} %pagesof

 Els nens de segon d'ESO, un cop per setmana, ens trobem amb el nostre fillol/a que és un nen/a de segon de primària. La nostra tasca és ensenyar-los a llegir. Durant l'estona que estem llegint amb ells anem anotant  si millora, si ha de millorar; la seva evolució. Sovint, nosaltres, els de secundària, els llegim un fragment del llibre perquè vegin com han d'entonar, com s'han de respectar els punts, les comes, etc. 
 A final de curs totes les observacions que hem escrit aniran a l'àlbum dels nens.

 Aquest projecte és una nova experiència per a mi. M'agrada moltíssim el fet d'ajudar a millorar un nen més petit

\authorandplace{Roger Aylagas}{Torres 2n ESO}

\section*{}

 És una idea molt bona ja que els alumnes de segon de primària encara no dominen la lectura. Crec que els pot anar molt bé ja que estan llegint amb gent que fa uns anys fèiem els mateixos errors i hem après a rectificar-los.
És una iniciativa que s'hauria de continuar durant molt de temps.

\authorandplace{Àlex Gascó Orgilles}{2n ESO}

\section*{}

 Jo crec que és una idea molt bona perquè ells aprenen a llegir bé i nosaltres sabem com ho fèiem a la seva edat. A més, nosaltres aprenem a ensenyar i ells a rectificar els errors que feien abans que nosaltres el corregíssim.
Quan estic amb l'Eric (el meu fillol- lector) em sento com un professor que ensenya a llegir al seu alumne. És un nen molt simpàtic. I a la millor algun dia serà un bon escriptor.

\authorandplace{Paco Martín Cárcoba}{2n ESO }
\end{news}
