%%%%%%%%%%%%%%%%%%%%%%%%%%%%%%%%%%%%%%%%%%%%%%%%%%%%%%%%%%%%%%%
%
% Arxiu pregenerat.
% NO EDITAR
%
%%%%%%%%%%%%%%%%%%%%%%%%%%%%%%%%%%%%%%%%%%%%%%%%%%%%%%%%%%%%%%%
%doc: Revista 3/Expressio.doc
\begin{news}
{2} %columnes
{Si no fos qui sóc qui o què m'agradaria ser...}
{Expressió escrita}
{ESO}
{13} %pagesof

\section*{Si no fos qui sóc m'agradaria ser... el Tortell Poltrona}

Des que jo era petita m'ha encantat el Tortell Poltrona, sempre l'he admirat. Els meus pares ens portaven, a vegades, al meu germà i a mi,  al circ per poder veure el nostre pallasso preferit.

Quan era petita, només l'admirava perquè m'encantaven les seves pallassades i les tonteries que feia.

Però ara, que ja sóc més gran, també l'admiro com a persona. M'encanta que pertanyi a la fundació i ONG de “ Pallassos sense Fronteres”.

Sempre me'n recordaré d'una vegada que els meus pares em van portar al CIRC CRIC. Aquell dia per  mi va ser perfecte; més o menys jo tenia 8 anys i aquella tarda vaig veure un espectacle de Tortell Poltrona que em va agradar molt. El gag consistia en què un pallasso havia de treure els pantalons al Tortell Poltrona. Quan va aconseguir baixar-los-hi, el Tortell portava uns calçotets grandíssims i amb cors de color lila. 

Vaig riure molt. 

\authorandplace{Júlia Guarro}{  4t ESO}

\section*{Tan de bo pogués ser un núvol!!}

Jo recordo que fa uns dos anys vaig tenir una conversa d’aquest estil amb el meu cosí gran. Ell em deia convençut que voldria ser una persona xinesa; era estrany d’entendre’l, però tenia els seus motius i se’l veia molt segur d’aquella idea, i després d’un moment de pausa vaig respondre-li que jo voldria ser o m’hagués agradat ser un núvol. El meu cosí em va mirar amb una cara més estranya que la meva quan moments abans ell m’havia dit que voldria ser xinès; em va preguntar el perquè de la meva resposta i només se’m va acudir una paraula: tranquil.litat. Poder anar tranquil.lament per sobre de tothom mentre dónes la volta al món, hauria de ser espectacular, a part que no hi hauria crisis, ni baralles ni res del que vivim al nostre dia a dia. Aquesta idea la continuo pensant  actualment i diria que ja l’he explicada diverses vegades després de la xerrada que vaig tenir amb el meu cosí.


\authorandplace{Marc Morales}{4t ESO}

\end{news}
